 \documentclass[a4paper,12pt]{article}
\usepackage[a4paper,top=1.3cm,bottom=2cm,left=1.5cm,right=1.5cm,marginparwidth=0.75cm]{geometry}
\usepackage{setspace}
\usepackage{cmap}					
\usepackage{mathtext} 				
\usepackage[T2A]{fontenc}			
\usepackage[utf8]{inputenc}			
\usepackage[english,russian]{babel}
\usepackage{multirow}
\usepackage{graphicx}
\usepackage{wrapfig}
\usepackage{tabularx}
\usepackage{float}
\usepackage{longtable}
\usepackage{hyperref}
\hypersetup{colorlinks=true,urlcolor=blue}
\usepackage[rgb]{xcolor}
\usepackage{amsmath,amsfonts,amssymb,amsthm,mathtools} 
\usepackage{icomma} 
\mathtoolsset{showonlyrefs=true}
\usepackage{euscript}
\usepackage{mathrsfs}

\DeclareMathOperator{\sgn}{\mathop{sgn}}
\newcommand*{\hm}[1]{#1\nobreak\discretionary{}
	{\hbox{$\mathsurround=0pt #1$}}{}}


\title{\textbf{Определение моментов инерции твердых тел с помощью трифилярного подвеса. (1.2.3)}}
\author{Лавыгин Кирилл}
\date{10 ноября 2022}


\begin{document}
	
	\maketitle
	
	\section{Аннотация}
	В ходе работы будет измерен момент инерции различных тел, проверена предложенная теоретическая зависимость. Так-же будет проверенна аддитивность моментов инерции и теорема Гюйгенса-Штейнера. Используемое оборудование: трифилярный подвес; секундомер; счетчик числа колебаний; набор тел, момент инерции которых нужно измерить.
	
	\section {Экспериментальная установка}
	
	\begin{wrapfigure}{l}{7cm}
		\includegraphics[width=0.95\linewidth]{1.2.3 ustan.jpg}
		\caption{Трифилярный подвес}\label{risunok}
	\end{wrapfigure}
	
	Для наших целей удобно использовать устройство, показанное на Рис. \ref{risunok} и называемое трифилярным подвесом. Оно состоит из укрепленной на некоторой высоте неподвижной платформы $P$ и подвешенной к ней на трех симметрично расположенных нитях $AA'$, $BB'$ и $CC'$, вращающейся платформы $P'$. 
	
	Чтобы не вызывать дополнительных раскачиваний, лучше поворачивать верхнюю платформу, укрепленную на неподвижной оси. После поворота верхняя платформа остается неподвижной в течение всего процесса колебаний. После того, как нижняя платформа $P'$ оказывается повернутой на угол $\varphi$ относительно верхней платформы $P$, возникает момент сил, стремящийся вернуть нижнюю платформу в положение равновесия, при котором относительный поворот платформ отсутствует. В результате платформа совершает крутильные колебания.
	
	\section{Теоретические сведения}
	
	\par Инерционность при вращении тела относительно оси определяется моментом инерции тела относительно этой оси. Момент инерции твердого тела относительно неподвижной оси вращения вычисляется по формуле:
	
	\begin{equation}
		I = \int r^2 dm
	\end{equation}
	
	Здесь $r$ -- расстояние элемента массы тела $dm$ от оси вращения. Интегрирование проводится по всей массе тела $m$.
	
	Если пренебречь потерями энергии на трение о воздух и крепление нитей, то уравнение сохранения энергии при колебаниях можно записать следующим образом:
	
	\begin{equation}\label{moment}
		\frac{I \dot{\varphi^2}}{2} + mg(z_0-z) = E
	\end{equation}
	
	Здесь $I$ -- момент инерции платформы вместе с исследуемым телом, $m$ -- масса платформы с телом, $\varphi$ -- угол поворота платформы от положения равновесия системы, $z_0$ -- координата по вертикали центра нижней платформы $O'$  при равновесии ($\varphi = 0$), $z$ -- координата той же точки при некотором угле поворота $\varphi$. Первый член в левой части уравнения -- кинетическая энергия вращения, второй член -- потенциальная энергия в поле тяжести, $E$ -- полная энергия системы (платформы с телом).
	
	Воспользуемся системой координат $x, y, z$, связанной с верхней платформой, как показано на Рис. \ref{risunok}. Координаты верхнего конца одной из нитей подвеса точки $C$ в этой системе -- $(r, 0, 0)$. Нижний конец данной нити $C'$, находящийся на нижней платформе, при равновесии имеет координаты $(R, 0, z_0)$, а при повороте платформы на угол $\varphi$ эта точка переходит в $C''$ с координатами $(Rcos\varphi, Rsin\varphi, z)$. расстояние между точками $C$ и $C''$ равно длине нити, поэтому, после некоторых преобразований, получаем: 
	
	\begin{center}
		\begin{spacing}{1.6}
			$ (R\cos\phi - r)^2 + R^2\sin^2\phi + z^2 = L^2 $
			
			$ z^2 = L^2 - R^2 - r^2 + 2Rr\cos\phi \approx z^2_{0} - 2Rr(1 - \cos\phi) \approx z^2_{0} - Rr\phi^2 $
			
			$ z = \sqrt{z^2_{0} - Rr\phi^2} \approx z_{0} - \frac{Rr\phi^2}{2z_{0}} $
		\end{spacing}
	\end{center}
	
	Подставляя $z$ в уравнение \eqref{moment}, получаем:
	
	\begin{equation}
		\frac{1}{2}I\dot{\varphi^2} + mg \frac{Rr}{2z_0}\varphi^2 = E
	\end{equation}
	
	Дифференцируя по времени и сокращая на $\dot\varphi$, находим уравнение крутильных колебаний системы:
	
	\begin{equation}
		I\ddot\varphi^2 + mg\frac{Rr}{2z_0}\varphi^2 = 0
	\end{equation}
	
	Производная по времени от $E$ равна нулю, так как потерями на трение, как уже было сказано выше, пренебрегаем.
	
	Решение этого уравнения имеет вид:
	
	\begin{equation}
		\varphi = \varphi_0 sin \left(\sqrt{\frac{mgRr}{Iz_0}}t + \theta\right)
	\end{equation}
	
	Здесь амплитуда $\varphi_0$ и фаза $\theta$ колебаний определяются начальными условиями. Период крутильных колебаний нашей системы равен:
	
	\begin{equation}
		T = 2\pi \sqrt{\frac{Iz_0}{mgRr}}
	\end{equation}
	
	Из формулы для периода получаем:
	
	\begin{equation}\label{momin}
		I = \frac{mgRrT^2}{4 \pi^2z_0} = kmT^2
	\end{equation}
	\noindent где $k = \frac{gRr}{4\pi^2z_0}$ -- величина, постоянная для данной установки.
	\section{Задание}
	\subsection{Проверка установки}
	При возбуждении крутильных колебаний маятникообразных движений платформы не наблюдается -- устройство функционирует нормально.
	
	При выводе формул мы предполагали, что потери энергии, связанные с трением, малы, то есть мало затухание колебаний. Это значит, что теоретические вычисления будут верны, если выполняется условие:
	
	\begin{equation}
		\tau \gg T
	\end{equation}
	
	Проверим данное условие. На нашей установке амплитуда колебаний уменьшилась в $2$ раза за $31$ колебание. Соотношение выполняется -- установка пригодна для проведение эксперимента. Начальное отклоненеи было выбрано $\alpha \approx 15^\circ$.
	
	\subsection{Параметры установки и коэффицент $k$}

	\begin{table}[H]
		\begin{center}
			\begin{tabular}{|c|c|c|c|c|}
				\hline
				$m \text{, г}$  & $R\text{, мм}$ & $r\text{, мм}$ & $L\text{, см}$ & $z_0\text{, см}$\\
				\hline
				1012.5 & 114.5 & 30.5 & 215 & 215\\
				\hline
				\hline
				$\sigma_m \text{, г}$  & $\sigma_R\text{, мм}$ & $\sigma_r\text{, мм}$ & $\sigma_L\text{, см}$ & $\sigma_{z_0}\text{, см}$\\
				\hline
				0.5 & 0.5 & 0.3 & 1 & 1\\
				\hline 
			\end{tabular}
			\caption{Парметры установки}
			\label{param}
		\end{center}
	\end{table}
	
	\noindent где $\sigma_m$, $\sigma_R$, $\sigma_r$, $\sigma_L$, $\sigma_{z_0}$ -- погрешности соответствующих величин.
	
	\bigskip
	
	По полученным данным вычислим постоянную для конструкции №3:
	
	\begin{equation}
		k = \frac{gRr}{4\pi^2z_0} \approx 4.040\cdot 10^{-4} \frac{\text{м}^2}{\text{с}^2}
	\end{equation}
	
	Погрешность же $k$ будет равна:
	
	\begin{equation}
		\sigma_k = k \cdot \sqrt{\left( \frac{\sigma_R}{R}\right)^2 + \left( \frac{\sigma_r}{r}\right)^2 + \left( \frac{\sigma_{z_0}}{z_0}\right)^2} \approx 0,05 \cdot 10^{-4} \frac{\text{м}^2}{\text{с}^2}
	\end{equation}
	
	\subsection{Момент инерции платформы}
	
	Определить момент инерции платформы можно по формуле \eqref{momin}. Для этого нам необходимо определить период колебаний ненагруженной платформы. Измеряем преиод, получаем:
	
	\begin{table}[!h]
		\begin{center}
			\begin{tabular}{|r|r|r|r|}
				\hline
				 Амплитуда $A, ^\circ$ & Время колебаний         $t, c$ &  Число колебаний $n$ &        Период колебаний $T$ \\
				 \hline
				20 & 136.7 & 31 & 4.410 \\
				\hline
				15 & 136.2 & 31 & 4.394 \\
				\hline
				10 & 131.6 & 30 & 4.387 \\
				\hline
				30 & 133.7 & 30 & 4.457 \\
				\hline
				28 & 132.5 & 30 & 4.416 \\
				\hline
				30 & 132.3 & 30 & 4.411 \\
				\hline
				30 & 132.7 & 30 & 4.427 \\
				\hline			\end{tabular}
		\end{center}
	\end{table}
	
	Видно, что во время четвертого опыта что-то произошло и показания в нём сильно отличаются от средних, поэтому учитывать его в дальнейшем мы не будем
	
	Тогда, средний период колебания платформы будет: \[T_\text{ср} \approx 4,407\text{ с}\]
	
	Рассчитаем его погрешность: 
	
	\begin{equation}
		\sigma_T^{\text{сист}} = 0,003\text{ с}
	\end{equation}
	\begin{equation}
		\sigma_T^{\text{случ}} = \sigma_\text{случ}=\sqrt{\frac{1}{  N_\text{изм} \left( N_\text{изм} - 1 \right)}\sum_{i=1}^{N_\text{изм}}\left( T_\text{ср} - T_i \right)^2 } \approx 0,014\text{, c}
	\end{equation}
	\begin{equation}
		\sigma_T = \sqrt{\sigma_\text{случ}^{2} + \sigma_{\text{сист}}^{2}} \approx 0,014\text{, с}
	\end{equation}
	Значит $T_\text{ср} = \left(4,407 \pm 0,014\right)\text{, с}$. Теперь мы можем определить момен инерции платформы:
	
	\begin{equation}
		I_\text{пл} = kmT^2 \approx 7.944\cdot 10^{-3}  \text{  кг $\cdot$ $\text{м}^2$}  
	\end{equation}
	
	Найдем погрешность найденного нами момента инерции платформы:
	
	\begin{equation}
		\varepsilon_I = \sqrt{ \left(\frac{\sigma_k}{k}\right)^2 +\left(\frac{\sigma_m}{m}\right)^2 + \left(2\frac{\sigma_T}{T}\right)^2} \approx 0.016
	\end{equation}
	\begin{equation}
		\sigma_{I_\text{пл}} = \varepsilon_I \cdot I_\text{пл} \approx 0,13 \cdot 10^{-3} \text{  кг $\cdot$ $\text{м}^2$ }
	\end{equation}
	Получаем, что с помощью данной конструкции мы можем определять момент инерции тела с погрешностью 1.6\%, и $I_\text{пл} = \left(7,944 \pm 0,13\right)\cdot 10^{-3} \text{ кг $\cdot$ $\text{м}^2$}$
	
	\subsection{Определение моментов инерции различных тел. Аддитивность моментов инерции}
	
	Измерим периоды колебаний платформы с различными телами таким же образом, как и для ненагруженной платформы, $n$ -- число измерений:
	
	\begin{table}[!h]
		\begin{center}
			\begin{tabular}{|l|r|r|r|r|r|}
				\hline
				Вращающиеся тела & $t, c$ & $n$ & Период $T$ & Масса установки $m$, г &  $I\cdot10^3, \text{ кг} \cdot \text{м}^2 $ \\ \hline
				Только платформа &                         135.95 &                  31 &                4.385 & 1012.5 &  7.87 \\ \hline
				Обод             &                         136.46 &                  32 &                4.264 & 1596.9 & 11.73 \\ \hline
				Диск             &                         130.21 &                  33 &                3.946 & 1750.2 & 11.01 \\ \hline
				Обод+Диск        &                         119.68 &                  30 &                3.989 & 2334.6 & 15.01 \\ \hline
			\end{tabular}
			\caption{Моменты инерции платформы с различными телами}
			\label{momtel}
		\end{center}
	\end{table}
	
	Для подтверждения аддитивности необходимо показать,что выполняются условия:
	
	
	\begin{equation} \label{plc}
		I_\text{пл+об} = I_\text{пл} + I_\text{ц}
	\end{equation}
	\begin{equation}\label{plk}
		I_\text{пл+д} = I_\text{пл} + I_\text{д}
	\end{equation}
	\begin{equation}
		I_\text{пл+д+об} = I_\text{пл} + I_\text{ц} + I_\text{д}
		\label{plck}
	\end{equation}
	
	
	Из Таблицы \eqref{momtel} и формул \eqref{plc}, \eqref{plk} мы можем найти момент инерции обода и кольца: $I_\text{об} = I_\text{пл+об} - I_\text{пл} = \left(3.9 \pm 0.3\right)\cdot 10^{-3} \text{,  кг $\cdot$ $\text{м}^2$}$, а $I_\text{д} = I_\text{пл+д} - I_\text{пл} = \left(3.1 \pm 0.3\right) \cdot 10^{-3} \text{,  кг $\cdot$ $\text{м}^2$ }$. 
	
	Тогда, для доказательства аддитивности, проверим уравнение \eqref{plck}: $I_\text{пл} + I_\text{ц} + I_\text{д}=14.9 \pm 0.6 \cdot 10^{-3} \text{,  кг $\cdot$ $\text{м}^2$ }\approx I_\text{пл+д+об}$.   Оно выполняется, следовательно моменты инерции аддитивны.
	
	Теперь сравним полученные нами моменты инерциии для тел, и их теоретические значения:
	
	Для диска: $I_\text{д} = \frac{1}{2}m_\text{д}R_\text{д}^2$. Радиус данного цилиндра $R_\text{д} = 85.05\text{ мм}$, тогда $I_\text{д} = 2.7 \cdot 10^{-3}\text{,  кг $\cdot$ $\text{м}^2$ }$, что, к сожалению, выходит за пределы погрешности полученного значения, но в нашем расчете так же присутствует погрешность, так что эти два диапазона пересекаются
	
	Для обода же сильно сложнее т.к. оно явно не является тонким: $I_\text{об}=\pi\rho(R_\text{внеш}^4-R_\text{внут}^4)/2$, где $\rho=m/(\pi(R_\text{внеш}^2-R_\text{внут}^2)) $. $m=584.4 \text{ г}$, $R_\text{внеш}=83.35$ мм, $R_\text{внут}=78.65$ мм. Получаем, что $I_\text{к} = 3.8\cdot 10^{-3}\text{  кг $\cdot$ $\text{м}^2$ }$, тут рассчитанное значение уже совпадает с измеренным.
	
	\subsection{Зависимость момента инерции системы тел от их расположения. График зависимости $I(h^2)$}
	
	\begin{wrapfigure}[10]{r}{0.4\textwidth}
		\vspace{-3em}
		\includegraphics[width=0.33\textwidth]{position}
		\caption{Схема расположения грузов.}
		\label{ris:position}
	\end{wrapfigure}
	
	Определим зависимость момента инерции системы двух тел от их взаимного расположения. Для этого располагая грузы, как показано на рис. \ref{ris:position}, получим зависимость от расстояния. Затем Используя формулу \ref{momin}, определим зависимость $I(h^2)$.
	
	Полученные результаты измерений занесем в таблицу \eqref{tab:period}. Основывыаясь на результатах таблицы, построим график зависимости $ I(h^{2}) $ (Рис. \ref{ris:grafik}).
	\bigskip\bigskip\bigskip\bigskip
	\bigskip\bigskip
	
	
	\begin{table}[h]
		\begin{center}
		\begin{tabular}{| l | r | r | r | r | r | r | r | r | r |}
			\hline
			Число делений & 0 & 2 & 4 & 6 & 8 & 10 & 12 & 14 & 16 \\
			\cline{1-10}
			 $h$, см & 0 & 1 & 2 & 3 & 4 & 5 & 6 & 7 & 8 \\
			\cline{1-10}
			$t$, c & 96.76 & 100.56 & 99.48 & 106.04 & 114.34 & 117.05 & 127.90 & 135.68 & 130.69 \\
			\cline{1-10}
			Число колебани й$n$ & 31 & 32 & 31 & 32 & 33 & 32 & 33 & 33 & 30 \\
			\cline{1-10}
			Период $T$,c & 3.1213 & 3.1425 & 3.2090 & 3.3138 & 3.4648 & 3.6578 & 3.8758 & 4.1115 & 4.3563 \\
			\cline{1-10}
			$I\cdot 10^{3}$ $\text{,  кг $\cdot$ $\text{м}^2$ }$ & 9.56 & 9.69 & 10.11 & 10.78 & 11.78 & 13.13 & 14.74 & 16.59 & 18.62 \\
			\cline{1-10}
		\end{tabular}
			\label{tab:period}
		\end{center}
	\end{table}
	
	\begin{figure}[H]
		\begin{center}
			\includegraphics[width=0.95\textwidth]{graph}
			\label{ris:grafik}
		\end{center}
	\end{figure}
	
	По графику понятно, что $I = kh^2 + b$.Тогда $b$ -- момент инерции платформы + диска. Для вычисления коэффициентов $ k $ и $ b $ воспользуемся методом наименьших квадратов:
	
	\begin{equation}
		k=\frac{\langle xy\rangle-\langle x\rangle \langle y\rangle}{\langle x^2\rangle - \langle x\rangle^2}\approx 1.429\text{ кг}
	\end{equation}
	
	\begin{equation}
		b=\langle y \rangle -k\langle x \rangle\approx 9.54\cdot 10^{-3}\text{  кг $\cdot$ $\text{м}^2$ },
	\end{equation}
	где $ x=h^2 $, $ y=I $.
	
	Случайные погрешности вычисления $ k $ и $ b $ можно найти по следующим формулам:
	
	\begin{equation}
		\sigma_k^\text{случ}=\frac{1}{\sqrt{N}}\sqrt{\frac{\langle y^2 \rangle - \langle y \rangle^2}{\langle x^2 \rangle - \langle x \rangle^2} - k^2  } \approx 0.007 \text{ кг}
	\end{equation}
	
	\begin{equation}
		\sigma_b^\text{случ}= \sigma_k^\text{случ} \sqrt{\langle x^2 \rangle - \langle x \rangle^2} \approx 0.02\cdot 10^{-3} \text{,  кг $\cdot$ $\text{м}^2$ }.
	\end{equation}
	
	Систематическая погрешность вычисления коэффициентов определяется следующим соотношением:
	
	\begin{equation}
		\sigma^\text{сист}_k = k\sqrt{\left( \varepsilon_{I} \right)^2 + \left( 2\varepsilon_{h} \right)^2 } \approx 0.02 \text{ кг}
	\end{equation}
	\begin{equation}
		\sigma^\text{сист}_b = b \cdot \varepsilon_I \approx 0.15\cdot 10^{-3} \text{  кг $\cdot$ $\text{м}^2$ }.
	\end{equation}
	
	Тогда полную погрешность вычисления коэффициентов подсчитываем по следующей формуле:
	
	\begin{equation}
		\sigma_k = \sqrt{\left( \sigma_k^\text{случ} \right)^2 + \left( \sigma_k^\text{сист} \right)^2 } \approx 0.02 \text{ кг}
	\end{equation}
\begin{equation}
	\sigma_b = \sqrt{\left( \sigma_b^\text{случ} \right)^2 + \left( \sigma_b^\text{сист} \right)^2 } \approx 0.15 \cdot 10^{-3}\text{  кг $\cdot$ $\text{м}^2$} .
\end{equation}
	
	Теперь можно рассчитать параметры диска из параметров графика. $m=k\approx1.429 \pm 0.02$, что совпадает с измеренным значением $m_\text{изм}=1.417г$. Теперь, зная, что момент инерции диска $I=b\approx (9.54\pm0.15)\cdot 10^{-3}\text{  кг $\cdot$ $\text{м}^2$ }$ определим радиус диска $r=\sqrt{\frac{2b}{k}}\approx0.116$ м. К сожалению у нас в данных не оказалось радиуса диска, но выглядит он достаточно реалистично.
	
	\section{Вывод}
	
	С помощью трифилярного подвеса можно определять момент инерции с достаточно большой точностью $\varepsilon \approx 1.6\%$. Такая точность обусловлена малой погрешностью измерения времени и условиями, при которых колебания подвеса можно считать слабозатухающими.
	
	Мы экспериментально доказали аддитивность моментов инерции с помощью различных тел.
	
	Так-же мы проверили, что выполняется теорема Гюйгенса-Штейнера, построив график $I(h^2)$, который оказался с линейным. С помощью этого графика были рассчитаны некоторые параметры данного нам диска, которые с допустимой точностью совпали с измеренными параметрами
	
	
\end{document}