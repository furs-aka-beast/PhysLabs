 \documentclass[a4paper,12pt]{article}
\usepackage[a4paper,top=1.3cm,bottom=2cm,left=1.5cm,right=1.5cm,marginparwidth=0.75cm]{geometry}
\usepackage{setspace}
\usepackage{cmap}					
\usepackage{mathtext} 				
\usepackage[T2A]{fontenc}			
\usepackage[utf8]{inputenc}			
\usepackage[english,russian]{babel}
\usepackage{multirow}
\usepackage{graphicx}
\usepackage{wrapfig}
\usepackage{tabularx}
\usepackage{float}
\usepackage{longtable}
\usepackage{hyperref}
\hypersetup{colorlinks=true,urlcolor=blue}
\usepackage[rgb]{xcolor}
\usepackage{amsmath,amsfonts,amssymb,amsthm,mathtools} 
\usepackage{icomma} 
\mathtoolsset{showonlyrefs=true}
\usepackage{euscript}
\usepackage{mathrsfs}

\DeclareMathOperator{\sgn}{\mathop{sgn}}
\newcommand*{\hm}[1]{#1\nobreak\discretionary{}
	{\hbox{$\mathsurround=0pt #1$}}{}}

\title{\textbf{Исследование прецессии уравновешенного гироскопа (1.2.5)}}
\author{Лавыгин Кирилл}
\date{20 октября 2022}


\begin{document}
	
	\maketitle
	
	\section{Аннотация}
	
	В ходе работы будет исследована вынужденная прецессия гироскопа, установлена зависимость скорости вынужденной прецессии от величины момента сил, действующий на ось гироскопа.Полученные значения будут сопоставлены с теми, что получены из теоретической модели. Ожидается подтверждение предложенной модели.\\
	\textbf{Оборудование:} гироскоп в кардановом подвесе, секундомер, набор грузов, отдельный ротор гироскопа, цилиндр известной массы, крутильный маятник, штангенциркуль, линейка.
	
	\section{Теоретические сведения}
	
	В этой работе исследуется зависимость скорости прецессии гироскопа от момента силы, приложенной к его оси. Для этого к оси гироскопа подвешиваются грузы. Скорость прецессии определяется по числу оборотов рычага вокруг вертикальной оси и времени, которое на это ушло, определяемому секундомером. В процессе измерений рычаг не только поворачивается в результате прецессии гироскопа, но и опускается. Поэтому его в начале опыта следует приподнять на 5-6 градусов.  Опять надо закончить, когда рычаг опустится на такой же угол.\\
	\begin{center}$
		\begin{array}{cc}
			\includegraphics[width=0.40\textwidth]{img1.png}&
			\includegraphics[width=0.40\textwidth]{img2.png}\\
		\end{array}$
	\end{center}
	В данной работе для описания движения гироскопа применяется уравнение моментов $\frac{d \vec{L}}{d t}=\vec{M}$, которое в случае нашей задачи поворота гироскопа принимает вид $\vec{M}=\vec{\Omega} \times \vec{L} .$
	
	Путем преобразований получим:
	
	\begin{equation}
		\Omega = \frac{mgl}{I_z\omega_0},
	\end{equation}
	
	где $m$ -- масса груза, $l$ -- расстояние от центра карданова подвеса до точки крепления груза на оси гироскопа, $I_z$ -- момент инерции гироскопа по его главной оси вращения. $\omega_0$ -- частота его вращения относительно главной оси, $\Omega$ -- частота прецессии.\\
	Таким образом измерение скорости прецессии гироскопа позволяет вычислить угловую скорость вращения его ротора.
	
	Момент инерции ротора относительно оси симметрии $I_0$ измеряется по крутильным колебаниям точной копии ротора, подвешиваемой вдоль оси симметрии на жесткой проволоке. Период крутильных колебаний $T_0$ зависит от момента инерции $I_0$ и модуля кручения проволоки $f$:
	
	\begin{equation}
		T_0 = 2\pi\sqrt{\frac{I_0}{f}}.
	\end{equation}
	
	Чтобы исключить модуль кручения проволоки, вместо ротора гироскопа к той же проволоке подвешивают цилиндр правильной формы с известными размерами и массой, для которого легко можно вычислить момент инерции $I_\text{ц}$. Для определения момента инерции ротора гироскопа имеем:
	
	\begin{equation}
		I_0 = I_\text{ц}\frac{T_0^2}{T_\text{ц}^2},
		\label{moment}
	\end{equation}
	Здесь $T_\text{ц}$ -- период крутильных колебаний цилиндра.\\
	\begin{center}
		\includegraphics[width=0.7\textwidth]{img3.png}
	\end{center}
	
	Скорость вращения ротора гироскопа можно определить и не прибегая к исследованию прецессии. У используемых в работе гироскопов статор имеет две обмотки, необходимые для быстрой раскрутки гироскопа. В данной работе одну обмотку используют для раскрутки гироскопа, а вторую -- для измерения числа оборотов ротора. Ротор электромотора всегда немного намагничен. Вращаясь, он наводит во второй обмотке переменную ЭДС индукции, частота которой равна частоте вращения ротора. Частоту этой ЭДС можно, в частности, измерить по фигурам Лиссажу, получаемым на экране осциллографа, если на один вход подать исследуемую ЭДС, а на другой -- переменное напряжение с хорошо прокалиброванного генератора. При совпадении частот на экране получаем эллипс.
	
	\section{Ход работы}
	Данные для частоты прецессии и опускания гироскопа: $\Omega=\frac{2\pi N}{t}, M=mgl$, где $l=120 \text{мм}$ -- длина рычага.
	
	\begin{center}
		\begin{tabular}{|r|r|r|r|r|}
			 \hline
			 $m, $г & $t, c$ & n & $\Omega\cdot 10^{-2}, \text{c}^{-1}$ & $M, H\cdot\text{м}$ \\
			 \hline
			342 & 148.97 & 5 & 21.088761 & 0.402192 \\
			\hline
			342 & 119.53 & 4 & 21.026304 & 0.402192\\
			\hline
			342 & 118.71 & 4 & 21.171545 & 0.402192\\
			\hline
			342 & 119.19 & 4 & 21.086283 & 0.402192\\
			\hline
			342 & 118.66 & 4 & 21.180466 & 0.402192\\
			\hline
			274 & 74.00 & 2 & 16.981582 & 0.322224\\
			\hline
			274 & 74.22& 2 & 16.931246 & 0.322224\\
			\hline
			274 & 74.44 & 2 & 16.881207 & 0.322224\\
			\hline
			220 & 92.44 & 2 & 13.594083 & 0.25872\\
			\hline
			220 & 92.22 & 2 & 13.626513 & 0.25872\\
			\hline

		\end{tabular}
		\begin{tabular}{|r|r|r|r|r|}
		\hline
		$m, $г & $t, c$ & $n$ & $\Omega\cdot 10^{-2}, \text{c}^{-1}$ & $M, H\cdot\text{м}$ \\
		\hline
		220 & 92.38 & 2 & 13.602913 & 0.25872\\
		\hline
		142 & 72.10 & 1 & 8.714543 & 0.166992\\
		\hline
		142 & 72.27 & 1 & 8.694044 & 0.166992\\
		\hline
		142 & 72.32 & 1 & 8.688033 & 0.166992\\
		\hline
		93 & 109.06 & 1 & 5.761219 & 0.109368\\
		\hline
		93 & 109.27 & 1 & 5.750147 & 0.109368\\
		\hline
		93 & 110.41 & 1 & 5.690776 & 0.109368\\
		\hline
		57 & 178.22 & 1 & 3.525522 & 0.067032\\
		\hline
		57 & 177.78 & 1 & 3.534248 &0.067032\\
		\hline
		\end{tabular}
	\end{center}
	
	Также мы измерили скорость опускания рычага ($h_0$ -- начальная высота точки подвеса, $h_\text{к}$ -- конечная, $\alpha$ -- угол, на который опустился рычаг). $\alpha \approx 
	\frac{h_0-h_\text{к}}{l}$:
	\begin{center}
		\begin{tabular}{|r|r|r|r|r|r|}
			\cline{1-6}
			$h_0, \text{м}$ & $h_\text{к}, \text{м}$ & $t, c$ & $m, \text{г}$ & $\alpha$ & $M, H\cdot\text{м}$ \\
			\cline{1-6}
			13.4 & 12.0 & 91.15 & 342 & 0.116667 & 0.402192 \\
			\cline{1-6}
			13.0 & 11.6 & 110.91 & 274 & 0.116667 & 0.322224 \\
			\cline{1-6}
			13.0 & 11.7 & 92.54 & 220 & 0.108333 & 0.258720 \\
			\cline{1-6}
			13.0 & 11.3 & 142.78 & 142 & 0.141667 & 0.166992 \\
			\cline{1-6}
			13.0 & 11.6 & 110.09 & 93 & 0.116667 & 0.109368 \\
			\cline{1-6}
			13.0 & 11.0 & 188.18 & 57 & 0.166667 & 0.067032 \\
			\cline{1-6}
		\end{tabular}
	\end{center}
	
	Построим график зависимости $\Omega(M)$ (аппроксимация производилась по МНК):
	\begin{figure}[H]
		\includegraphics[scale=1]{1.2.5 graph}
		\caption{Зависимость $ \Omega $ от $ M $}
		\label{graph}
	\end{figure}
	
	Как и ожидалось, график оказался линейным, что говорит о том, что наша теоритическая модель скорее всего верна. Далее найдем момент инерции ротора гироскопа по формуле \eqref{moment}, для этого посчитаем момент инерции цилиндра, с известной нам массой($1,6169$ кг) и диаметром($7,81$ см): $I_\text{ц} = \dfrac{1}{2}mr^2 \approx 1.2328\cdot 10^{-3}$ кг$\cdot \text{м}^2$. Теперь рассчитаем периоды. Для цилиндра проводилось измерение 8 колебаний, $t_\text{ц}=32.59, 32.46, 32.56$, а для ротора 10 колебаний $t_0=32.12, 32.10 ,32.16 $, тогда периоды: $T_\text{ц} =4.067 $ с и $T_0 =3.213 $ с. Тогда $I_0 \approx 7.694\cdot 10^{-4}$ кг$\cdot \text{м}^2$
	
	\section{Погрешности $\Omega$ и $I_0$}
	
	\begin{equation}
		\sigma_\Omega = \sqrt{ \sigma_\text{случ}^2 + \sigma_\text{сист}^2} \;\;\;\;\;\; \sigma_\Omega^\text{сист} = \Omega \varepsilon_T << \sigma_\Omega^\text{случ}=  \sqrt{\frac{1}{n(n-1)} \sum_{i=1}^{n}(\Omega_i - \overline{\Omega})^2}
	\end{equation}
	
	Каждая частота $\Omega$ с учетом погрешностей:
	\begin{itemize}
		\item $\Omega = (21.11 \pm 0,06)\cdot10^{-2}\text{ }\text{с}^{-1}$ для $m=342$г
		\item $\Omega = (16.93 \pm 0,05)\cdot10^{-2}\text{ }\text{с}^{-1}$ для $m=274$г
		\item $\Omega = (13.608 \pm 0.017)\cdot10^{-2}\text{ }\text{с}^{-1}$ для $m=220$г
		\item $\Omega = (8.698 \pm 0,014)\cdot10^{-2}\text{ }\text{с}^{-1}$ для $m=142$г
		\item $\Omega = (85.73 \pm 0,04)\cdot10^{-2}\text{ }\text{с}^{-1}$ для $m=93$г
		\item $\Omega = (3.529 \pm  0.006)\cdot10^{-2}\text{ }\text{с}^{-1}$ для $m=57$г
	\end{itemize}
	
	Погрешность $\sigma_{I_0} = I_0\cdot\sqrt{\varepsilon_{I_\text{ц}}^2+ 4\varepsilon_{T_0}^2+ 4\varepsilon_{T_\text{ц}}^2 } \approx 0.03\cdot10^{-4}$ кг$\cdot \text{м}^2$, значит $I_0 = (7.69\pm 0.03)\cdot10^{-4}$ кг$\cdot \text{м}^2$
	
	\section{Определение частоты вращения ротора гироскопа}
	
	Определить частоту вращения ротора можно по формуле $\omega_0 = \dfrac{1}{kI_0}$, где $k$ -- коэффицент наклона графика зависимости $\Omega(M)$.
	
	График построен по МНК, а значит:
	\begin{equation}
		k=\frac{\langle xy\rangle-\langle x\rangle \langle y\rangle}{\langle x^2\rangle - \langle x\rangle^2}\approx 0,5256\text{ }\frac{1}{\text{Дж}\cdot\text{с}}
	\end{equation}
	\begin{equation}
		\sigma_k^\text{сл}=\frac{1}{\sqrt{N}}\sqrt{\frac{\langle y^2 \rangle - \langle y \rangle^2}{\langle x^2 \rangle - \langle x \rangle^2} - k^2  } \approx 0,0009\text{ }\frac{1}{\text{Дж}\cdot\text{с}}
	\end{equation}
	
	Тогда $\omega_0 = 2474\text{ }\text{с}^{-1}$, а $\sigma_{\omega_0} = \omega_0\cdot\sqrt{\varepsilon_{I_0}^2+ \varepsilon_k^2} \approx 10\text{ }\text{с}^{-1}$
	
	Используя полученную угловую скорость можно определить частоту вращения ротора гироскопа: $\nu = \dfrac{\omega_0}{2\pi} \approx 393.7\text{ }\text{Гц}$, а $\sigma_{\nu} = \nu\varepsilon_{\omega_0} \approx 1.6\text{ }\text{Гц}$
	
	Таким образом получаем: \underline{$\nu = (393.7\pm1.6)\text{ Гц}$}, что с учетом погрешности, правда, не попадает в полученное значение при помощи осциллографа: $\nu_0 = 391\text{ Гц}$. Думаю, это связано с тем, что измерения с помощью осциллографа проведены с достаточно низкой точностью, так как для получения корректных показаний приходилось отключать питание мотора на время измерений, из-за чего было сложно провести много попуток измерений, а так-же гироскоп начинал замедляться.
	
	\section{Момент силы трения}
	
	Оценить момент силы трения мы можем по формуле: $M = \omega I_0 \omega_0$. Момент милы трения для каждой массы:
	
	\begin{itemize}
		\item $m = 342$ г, $\omega = 12.8\cdot 10^{-4}$ $\text{с}^{-1}$, $M = 2.5\cdot10^{-3}$ Н$\cdot$м
		\item $m = 274$ г, $\omega = 10.5\cdot 10^{-4}$ $\text{с}^{-1}$, $M = 2.0\cdot10^{-3}$ Н$\cdot$м
		\item $m = 220$ г, $\omega = 11.1\cdot 10^{-4}$ $\text{с}^{-1}$, $M = 2.2\cdot10^{-3}$ Н$\cdot$м
		\item $m = 142$ г, $\omega = 9.9\cdot 10^{-4}$ $\text{с}^{-1}$, $M = 1.9\cdot10^{-3}$ Н$\cdot$м
		\item $m = 93$ г, $\omega = 10.6\cdot 10^{-4}$ $\text{с}^{-1}$, $M = 2.0\cdot10^{-3}$ Н$\cdot$м
		\item $m = 57$ г, $\omega = 8.8\cdot 10^{-4}$ $\text{с}^{-1}$, $M = 1.7\cdot10^{-3}$ Н$\cdot$м
	\end{itemize}  
	Если рассчитать $\sigma_\omega$, то получим $\sigma_\omega \approx0.8$. По измеренным данным видно, что погрешность является существенно заниженной. Основная ошибка кроется в измерении угла, так как мы просто измеряли высоту точки крепления груза над столом, что достаточно неточно. С другой стороны, по предоставленным данным можно сделать предположение, хорошо согласующееся с теорией: момент силы трения пропорционален подвешенной массе.

\end{document}