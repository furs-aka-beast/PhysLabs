 \documentclass[a4paper,12pt]{article}
\usepackage[a4paper,top=1.3cm,bottom=2cm,left=1.5cm,right=1.5cm,marginparwidth=0.75cm]{geometry}
\usepackage{setspace}
\usepackage{cmap}					
\usepackage{mathtext} 				
\usepackage[T2A]{fontenc}			
\usepackage[utf8]{inputenc}			
\usepackage[english,russian]{babel}
\usepackage{multirow}
\usepackage{graphicx}
\usepackage{wrapfig}
\usepackage{tabularx}
\usepackage{float}
\usepackage{longtable}
\usepackage{hyperref}
\hypersetup{colorlinks=true,urlcolor=blue}
\usepackage[rgb]{xcolor}
\usepackage{amsmath,amsfonts,amssymb,amsthm,mathtools} 
\usepackage{icomma} 
\mathtoolsset{showonlyrefs=true}
\usepackage{euscript}
\usepackage{mathrsfs}

\DeclareMathOperator{\sgn}{\mathop{sgn}}
\newcommand*{\hm}[1]{#1\nobreak\discretionary{}
	{\hbox{$\mathsurround=0pt #1$}}{}}


\title{\textbf{Определение ускорения свободного падения при помощи оборотного маятника (1.4.2)}}
\author{Лавыгин Кирилл}
\date{18.11.22}


\begin{document}
	
	\maketitle
	
	\section{Аннотация}
	В ходе работы с помощью оборотного маятника будет измерена величина свободного падения. Помимо маятника в работе используются: две подвесные призмы; два груза ; электронный счётчик времени и числа колебаний; подставка с острием для определения положения центра масс маятника; закреплённая на стене консоль для подвешивания маятника; металлические линейки, штангенциркуль длиной 1 м.
	
	\section{Теоретические сведения}	
	Физическим маятником называют твёрдое тело, способное совершать колебания в вертикальной плоскости, будучи подвешено за одну из своих точек в поле тяжести. Ось, проходящая через точку подвес перпендикулярно плоскости качания, называется осью качания маятника.
	
	При малых колебаниях период колебаний физического маятника определяется формулой
	
	\[
	T=2 \pi \sqrt{\frac{J}{m g l}},
	\]
	
	где \(J\) - момент инерции маятника относительно оси качания, \(m\) - масса маятника, \(l\) - расстояние от оси качания до центра масс маятника.
	
	Если сравнить (1) с известной формулой колебаний математического маятника длиной \(l(T=2 \pi \sqrt{l / g})\), можно определить приведённую длину физического маятника как
	
	\[
	l_{\text {пр }}=\frac{J}{m l} \text {. }
	\]
	
	Смысл приведённой длины в том, что при длине математического маятника, равной \(l_{\text {пр }}\), его период колебаний совпадает с периодом колебаний физического маятника.
	
	\subsection{Теорема Гюйгенса об оборотном маятнике}
	Пусть \(O_{1}\) - точка подвеса физического маятника, а \(C\) - его центр масс. Отложим отрезок длиной \(l_{\text {пр вдоль }}\) линии \(O_{1} C\), и обозначим соответствующую точку как \(O_{2}\) - эту точку называют центром качания физического маятника. Заметим, что приведённая длина всегда больше расстояния до центра масс \(\left(l_{\text {пр }}>l\right)\), поэтому точка \(O_{2}\) лежит по другую сторону от центра масс.
	
	Точки \(O_{1}\) и \(O_{2}\) обладают свойством взаимности: если перевернуть маятник и подвесить его за точку \(O_{2}\), то его период малых колебаний останется таким же, как и при подвешивании за точку \(O_{1}\) (теорема Гюйгенса). На этом свойстве - «оборотности» - и основан довольно точный метод определения ускорения свободного падения, применяемый в данной работе.
	
	Докажем теорему Гюйгенса об оборотном маятнике.
	
	\begin{wrapfigure}[35]{r}{0.35\linewidth} 
		\includegraphics[width=\linewidth]{2022_11_17_bcb6dacc30afae2df83fg-02}
		\caption{К теореме Гюйгенса}
	\end{wrapfigure}
	
	Пусть \(O_{1}\) и \(O_{2}\) - две точки подвеса физического маятника, лежащие на одной прямой с точкой \(C\) по разные стороны от неё. Тогда периоды колебаний маятника равны соответственно
	
	\[
	T_{1}=2 \pi \sqrt{\frac{J_{1}}{m g l_{1}}}, \quad T_{2}=2 \pi \sqrt{\frac{J_{2}}{m g l_{2}}} .
	\]
	
	По теореме Гюйгенса-Штейнера имеем
	
	\[
	J_{1}=J_{C}+m l_{1}^{2}, \quad J_{2}=J_{C}+m l_{2}^{2},
	\]
	
	где \(J_{C}\) - момент инерции маятника относительно оси, проходящей через центр масс перпендикулярно плоскости качания.
	
	Пусть периоды колебаний одинаковы: \(T_{1}=T_{2}\). Тогда одинаковы должны быть и приведённые длины:
	
	\[
	l_{\text {пр }}=\frac{J_{1}}{m l_{1}}=\frac{J_{2}}{m l_{2}} .
	\]
	
	С учётом (4) имеем:
	
	\[
	l_{\text {пр }}=\frac{J_{C}}{m l_{1}}+l_{1}=\frac{J_{C}}{m l_{2}}+l_{2},
	\]
	
	откуда следует, что при \(l_{1} \neq l_{2}\) справедливо равенство
	
	\[
	J_{C}=m l_{1} l_{2} .
	\]
	
	Наконец, подставляя (6) обратно в (5), получим
	
	\[
	l_{\text {пр }}=l_{1}+l_{2} \text {. }
	\]
	
	Таким образом, если периоды колебаний при подвешивании маятника в точках \(O_{1}\) и \(O_{2}\) paвны, то расстояние между точками подвеса равно приведённой длине маятника. Нетрудно видеть, что и обратное утверждение также верно.
	
	Заметим также, что период колебаний маятника (3), рассматриваемый как функция от \(l_{1}\),
	
	\[
	T=2 \pi \sqrt{\frac{J_{C}+m l_{1}^{2}}{m g l_{1}}},
	\]
	
	имеет минимум при \(l_{1 \mathrm{~min}}=\sqrt{J_{C} / m}\). Из (6) видно, что в этой точке \(l_{2}=\) \(l_{1}=l_{\text {пр }} / 2\), то есть центр масс находится посередине между сопряжёнными точками \(O_{1}\) и \(O_{2}\). График зависимости \(T\left(l_{1}\right)\) схематично представлен на Рис. 2.
	
	\includegraphics[scale=0.5]{2022_11_17_bcb6dacc30afae2df83fg-03}
	
	Рис. 2. Зависимость периода колебаний от положения центра масс относительно оси качания
	
	\subsection{Измерение \(\operatorname{g}\)}
	Пусть \(L \equiv \overline{O_{1} O_{2}}=l_{1}+l_{2}-\) расстояние между двумя «сопряжёнными» точками подвеса физического маятника. Если соответствующие периоды малых колебаний равны, \(T_{1}=T_{2}=T\), то по теореме Гюйгенса \(L=l_{\text {пр }}\). Тогда из (1) и (2) находим ускорение свободного падения:
	
	\[
	\mathrm{g}_{0}=(2 \pi)^{2} \frac{L}{T^{2}} .
	\]
	
	Точного совпадения \(T_{1}=T_{2}\) на опыте добиться, конечно, невозможно. Поэтому получим формулу для определения ускорения свободного падения g, если измеренные периоды незначительно различаются: \(T_{1}=T, T_{2}=\) \(T+\Delta T\). Из системы (3) и (4) получаем:
	
	\[
	\mathrm{g}=(2 \pi)^{2} \frac{l_{1}^{2}-l_{2}^{2}}{T_{1}^{2} l_{1}-T_{2}^{2} l_{2}},
	\]
	
	что также можно переписать как
	
	\[
	g=g_{0} \cdot \frac{\lambda-1}{\lambda-\frac{T_{2}^{2}}{T_{1}^{2}}},
	\]
	
	где \(\mathrm{g}_{0}=(2 \pi)^{2} L / T^{2}\), и для краткости введено обозначение \(\lambda=l_{1} / l_{2}\).
	
	Проанализируем отличия формулы (9) от (8) и оценим величину поправки. Пусть \(\varepsilon=\frac{\Delta T}{T} \ll 1\) - относительное отклонение при измерении периодов. Тогда при \(\lambda \neq 1\), пользуясь малостью \(\varepsilon\), получим
	
	\[
	g=g_{0} \cdot \frac{\lambda-1}{\lambda-(1+\varepsilon)^{2}} \approx g_{0} \cdot \frac{1}{1-\frac{2 \varepsilon}{\lambda-1}} \approx g_{0} \cdot(1+2 \beta \varepsilon) \text {, }
	\]
	
	
	где обозначено:
	
	\[
	\beta \equiv \frac{1}{\lambda-1}=\frac{l_{2}}{l_{1}-l_{2}} .
	\]
	
	Видно, что поправка \(\Delta \mathrm{g} \approx 2 \beta \varepsilon \mathrm{g}\) к формуле (8) остаётся малой, если мало относительное различие измеренных периодов \(\varepsilon\), но при этом также мал и коэффициент \(\beta=\frac{l_{2}}{l_{1}-l_{2}}\). В частности, при \(l_{2} \rightarrow l_{1}\) эта поправка неограниченно возрастает при любом конечном \(\varepsilon\). Поэтому на опыте отношение \(l_{1} / l_{2}\) не должно быть слишком близко к единице. На практике желательно, чтобы выполнялось условие
	
	\[
	l_{1}>2,5 l_{2} .
	\]
	\subsection{Предварительный расчёт положения грузов}
	Если первоначально расположить грузы на стержне произвольным образом, то для достижения равенства периодов колебаний потребуется исследовать зависимости периодов колебаний \(T_{1}\) и \(T_{2}\) при перемещении поочерёдно обоих грузов по штанге. При этом всякий раз необходимо при перестановке призм переворачивать тяжёлый маятник. Такая методика требует много времени и не всегда приводит к нужному результату.
	
	Существенно облегчить и ускорить поиск нужного положения грузов можно, если провести предварительные расчётыл. При этом для грубой оценки достаточно использовать максимально упрощённую модель, например, считать маятник тонким стержнем с закреплёнными на нём точечными массами. После установки грузов согласно предварительным расчётам, их положение может быть уже уточнено экспериментально.
	
	Пусть призмы \(\Pi_{1} \Pi_{2}\) задают сопряжённые точки подвеса, то есть период колебаний при перевороте маятника не изменяется. Тогда по теореме Гюйгенса расстояние между призмами \(L-\) это приведённая длина маятника. Это условие может быть записано либо в форме (2):
	
	\[
	J_{\Pi}=M L l_{2},
	\]
	
	где \(J_{\Pi}-\) момент инерции маятника относительно призмы \(\Pi_{2}\), либо в форме (5):
	
	\[
	J_{C}=M l_{1} l_{2},
	\]
	
	где \(J_{C}\) - момент инерции маятника относительно его центра масс. Здесь \(M\) - полная масса маятника.
	
	Как \(J_{\Pi}\), так и \(J_{C}\) являются функциями положений грузов \(b_{1}\) и \(b_{2}\) относительно соответствующих призм \(\Pi_{1}\) и \(\Pi_{2}\) (см. Рис. 5). Задание длины \(l_{1}\) (или \(l_{2}\) ) определяет положение центра масс маятника. Это позволяет, вопервых, рассчитать правые части (13) или \(\left(13^{\prime}\right)\), и, во-вторых, накладывает дополнительную связь на расстояния \(b_{1}\) и \(b_{2}\) (при известных массах всех элементов маятника). Тогда соотношения (13) или \(\left(13^{\prime}\right)\) превращаются в уравнение с одной неизвестной, например, \(b_{2}\). Корень этого уравнения можно приближённо найти графически с помощью приложенной электронной таблицы.
	
	\section{Задание}
	\subsection{Подбор положений грузов}
	Так-как предложенная модель дает недостаточно тонный результат -- определим положения грузов экспериментально. Нам повезло и в начальный момент периоды отличались лишь на $3\%$, далее мы пробовали сдвигать призму в разные стороны и получили отличие в $0.4\%$ ($T_1=\frac{31.10}{20}c, T_2=\frac{30.92}{20}c$).
	
	Далее мы измерили различные параметры получившейся установки, необходимые для расчетов:
	\[
	l1=12.3\pm0.1 \text{ см}, \text{ } l2=46.8\pm0.1 \text{ см}
	\]
	Эти длины были измерены штангенциркулями с точность $0.01$ см, но основная погрешность содержится в определении центра тяжести, которую можно оценить как $0.1$ см.
	
	Основная серия измерений:
	\begin{center}
		\begin{tabular}{|r|r|r|r|r|r|}
			\hline
			$t_1$, c & $n$ & $t_2$, c & $T_1$, c & $T_2$, c & $g, \frac{\text{м}}{c^2}$ \\
			\hline
			   77.31 &  50 &    77.76 &   1.5462 &   1.5552 &                9.800030 \\
			   \hline
			   77.30 &  50 &    77.75 &   1.5460 &   1.5550 &                9.802571 \\
			   \hline
			   77.31 &  50 &    77.74 &   1.5462 &   1.5548 &                9.798205 \\
			   \hline
			   77.31 &  50 &    77.71 &   1.5462 &   1.5542 &                9.795469\\
			   \hline
		\end{tabular}
	\end{center}
	Итого полученное значение и его случайная погрешность:
	\[
	g\approx9.799 \text{ } \frac{\text{м}}{c^2},\text{ } \sigma_g^\text{случ}=0.003 \text{ } \frac{\text{м}}{c^2}
	\]
	Теперь рассчитаем систематическую погрешность g ($\sigma_l=0.1$ см, $\sigma_T=0.01/50=0.0002$ c ):
	\[
	\sigma_g^\text{сист}=\frac{2(l_1-l_2)*\sigma_{l}(T_{1}^{2} l_{1}-T_{2}^{2} l_{2})+(l_1^2-l_2^2)(2\sigma_T(T_1l_1-T_2l_2)+\sigma_l(T_1^2-T_2^2))}{(T_{1}^{2} l_{1}-T_{2}^{2} l_{2})^2}=0.0003 \text{ }\frac{\text{м}}{c^2}
	\]
	Как можно видеть основной вклад в погрешность вносит именно погрешность измерения времени, а именно регистрации прохождения стержня через датчик.
	
	Итоговое значение:
	\[
	g= 9.799 \pm 0.003\text{ } \frac{\text{м}}{c^2}, \text{ } \epsilon_g=0.03\%
	\]
	\subsection{Анализ результата}
	Полученное значение попадает в ожидаемый диапазон значений. Точность метода оказалась действительно очень высокой. Чтобы её увеличить можно или увеличить точность регистрации прохождения, или просто измерять большее число колебаний. 
\end{document}