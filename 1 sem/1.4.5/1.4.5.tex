 \documentclass[a4paper,12pt]{article}
\usepackage[a4paper,top=1.3cm,bottom=2cm,left=1.5cm,right=1.5cm,marginparwidth=0.75cm]{geometry}
\usepackage{setspace}
\usepackage{cmap}					
\usepackage{mathtext} 				
\usepackage[T2A]{fontenc}			
\usepackage[utf8]{inputenc}			
\usepackage[english,russian]{babel}
\usepackage{multirow}
\usepackage{graphicx}
\usepackage{wrapfig}
\usepackage{tabularx}
\usepackage{float}
\usepackage{longtable}
\usepackage{hyperref}
\hypersetup{colorlinks=true,urlcolor=blue}
\usepackage[rgb]{xcolor}
\usepackage{amsmath,amsfonts,amssymb,amsthm,mathtools} 
\usepackage{icomma} 
\mathtoolsset{showonlyrefs=true}
\usepackage{euscript}
\usepackage{mathrsfs}

\DeclareMathOperator{\sgn}{\mathop{sgn}}
\newcommand*{\hm}[1]{#1\nobreak\discretionary{}
	{\hbox{$\mathsurround=0pt #1$}}{}}


\title{\textbf{Изучение колебаний струны (1.4.5)}}
\author{Лавыгин Кирилл}
\date{14.10.22}


\begin{document}
	
	\maketitle
	
	\section{Аннотация}
	
	В ходе работы будут изучены поперечные стоячие волны на струне, определены её собственные частоты колебаний, исследована зависимость скорости распространения поперечных волн на струне в зависимости от её натяжения. Ожидается подтверждение предложенной теоретической зависимости.\\
	\textbf{В работе используются:} закрепленная на станине стальная струна, набор грузов,
	звуковой генератор, постоянный магнит.
	
	\section{Теоретические сведения}
	
	Основное свойство струны -- гибкость, является следствием ее большой длины по сравнению с поперечными размерами. Даже струны, изготовленные из жестких материалов, практически не сопротивляются изгибанию, если размер изгибаемого участка значительно больше поперечного размера струны. Данный факт позволяет не учитывать при дальнейшей работе изгибные напряжения.
	
	Горизонтально закрепленная струна провисает под действием поля тяжести, при отсутствии натяжения. Достаточно натянутую струну можно считать прямой, если ее концы закреплены на одном горизонтальном уровне. Учитывая этот факт, в дальнейшем действие силы тяжести учитываться не будет.
	
	Натянутая струна с жестко закрепленными концами удобна для изучения колебаний. Это связанно с тем, что в струне можно непосредственно наблюдать простейшие типы колебаний и волн, измерять их параметры и сравнивать результаты наблюдения с результатами теоретических расчетов.
	
	Движение элементов струны может быть вызвано изменением ее формы или передачей ей импульса. Натяжение струны стремиться вернуть ее в изначальное прямолинейное положение, и это приводит к тому, что возникает движение элементов струны. Возмущения бегут вдоль струны.
	
	Скорость распространения подобного возмущения можно вычислить по формуле \eqref{velocity_of_deformation}.
	
	\begin{equation}
		u = \sqrt{\frac{T}{\rho_l}}
		\label{velocity_of_deformation}
	\end{equation}
	где $F$ -- сила натяжения струны, $\rho_{l}$ -- масса струны на единицу длины.
	
	При заданной частоте $\nu$ длина волны определяется по формуле:
	
	\begin{equation}
		\lambda = \frac{u}{\nu}
	\end{equation}
	
	Частоты собственных колебаний струны определяются формулой:
	
	\begin{equation}
		\nu_{n} = \frac{nu}{2l} = \frac{n}{2l}\sqrt{\frac{T}{\rho_l}}
		\label{frequency_velocity_equation}
	\end{equation}
	где $n$ -- число полуволн, $l$ -- длина струны.
	
	\section{Экспериментальная установка}
	
	\begin{figure}[h!]
		\begin{center}
			\includegraphics[width = 0.9\textwidth]{1.4.5 ustan}
			\caption{Схема экспериментальной установки}
			\label{facility}
		\end{center}
	\end{figure}
	
	Схема установки приведена на рис. \ref{facility}. Стальная гитарная струна 1 закрепляется в горизонтальном положении между двумя стойками с зажимами 2 и 3, расположенными на массивной станине 4. Один конец струны
	закреплен в зажиме 2 неподвижно. К противоположному концу струны, перекинутому через блок, прикреплена платформа с грузами 5, создающими
	натяжение струны. Зажим 3 можно передвигать по станине, устанавливая
	требуемую длину струны. Возбуждение и регистрация колебаний струны
	осуществляются с помощью электромагнитных датчиков (вибраторов),
	расположенных на станине под струной. Электромагнитный датчик 6 подключен к звуковому генератору 7 и служит для возбуждения колебаний струны, частота которых измеряется с помощью частотомера 10 (в нашей установке частотомер встроен в генератор). Колебания струны регистрируются с помощью электромагнитного датчика 8, сигнал с которого
	передается на вход осциллографа 9. Разъёмы, через которые датчики с помощью кабелей соединяются с генератором и осциллографом, расположены на корпусе станины
	
	В натянутой струне возникнуть колебания, которые сложившись после отражения от опор 2 и 4 создадут стоячую волну, если на длине струны уложится целое число полуволн.
	
	\section{Задание}
	\subsection{Визуальное наблюдение стоячих волн}
	На первой установке, на который выполняли работу, сигнал на осциллографе был очень нестабильным, а также была замечена большая нелинейность колебаний даже на $U=1$В. Так-же была зафиксирована большая величина гистерезиса, порядка $1.5$ Гц при $\nu\approx400$ Гц. Поэтому установка была заменена на другую
	
	Для натяжения струны подвесим грузы массой $1023.3 \text{, г}$, тогда сила натяжения: $T_1 = (M+ M_\text{плат})g = (1.0233 + 0.1081)9.81  = 11.10 \text{, Н}$. По формуле \eqref{frequency_velocity_equation}, где масса струны на единицу длины $\rho_l = 568.4\text{, мг}/\text{м}$ и $l=0.5$ м, получаем, что: $\nu_1^{\text{выч}} = 139.7\text{, Гц}$.
	
	Посредством возбуждения стоячей волны на основной гармонике находим частоты визуально наблюдаемых гармоник:
	\begin{table}[!h]
		\begin{center}
			\begin{tabular}{|c|c|c|c|c|}
				\hline
				$\nu_1$, Гц &$\nu_2$, Гц& $\nu_3$, Гц& $\nu_4$, Гц& $\nu_5$, Гц\\
				\hline
				136.8 & 275.7 & 414.2 & 555.0& 694.8\\
				\hline 
			\end{tabular}
			\caption{Частоты наблюдаемых гармоник}
		\end{center}
	\end{table}
	
	\subsection{Регистрация стоящих волн с помощью осциллографа}
	
	С помощью осциллографа будем находить частоты гармоник с 1 до 10. Изменяя массы грузов получаем таблицу:
	\begin{table}[!h]
		\begin{center}
			\begin{tabular}{|r|r|r|r|r|r|r|r|r|r|r|r|r|}
			\hline
			$m, $ г& $T, H$& $\nu_1, c^{-1}$ & $\nu_2, c^{-1}$ & $\nu_3, c^{-1}$ & $\nu_4, c^{-1}$ & $\nu_5, c^{-1}$ & $\nu_6, c^{-1}$ & $\nu_7, c^{-1}$ & $\nu_8, c^{-1}$ & $\nu_9, c^{-1}$ & $\nu_10, c^{-1}$\\
			\hline
			1023.3 & 10.028 & 136.8 & 275.7 & 414.2 & 555.0 & 694.8 & 834.7 & 974.1 & 1117 & 1259 & 1402 \\
			
				\hline
			1460.7&14.315& 169.3 & 340.3 & 510.4 & 681.4 & 852.2 & 1025 & 1196.3 & 1369 & 1543 & 1718 \\
				
			\hline
			1962.1&19.229 & 187.1 & 378.0 & 564.4 & 755.3 & 941.2 & 1163 & 1321 & 1519 & 1705 & 1905\\
			\hline
		\end{tabular}
			\caption{Снятые результаты частот гармоник}
		\end{center}
	\end{table}
	
	Построим график зависимости $\nu_n$ от $n$ по МНК:
	
	\begin{figure}[H]
		\begin{center}
			\includegraphics[width = 0.8\textwidth]{1.4.5 graph}
		\end{center}
	\end{figure}
	
	Воспользуемся формулой \eqref{frequency_velocity_equation}, получаем, что угол наклона графика $k = \dfrac{u}{2l}$. По формулам МНК рассчитаем также случайные погрешности коэффициентов наклона:
	
	\begin{equation}
		k=\frac{\langle n\nu\rangle-\langle n\rangle \langle \nu\rangle}{\langle n^2\rangle - \langle n\rangle^2},
	\end{equation}
	
	\begin{equation}
		\sigma_k^\text{случ}=\frac{1}{\sqrt{N}}\sqrt{\frac{\langle \nu^2 \rangle - \langle \nu \rangle^2}{\langle n^2 \rangle - \langle n \rangle^2} - k^2  },
	\end{equation}
	
	\begin{equation}
		\sigma_k^{\text{сист}} = k\sqrt{ \varepsilon_\nu^2 + \varepsilon_l^2 }
	\end{equation}
	
	\begin{equation}
		\sigma_k = \sqrt{\sigma_\text{случ}^2 + \sigma_\text{сист}^2}
	\end{equation}
	
	Посчитав все погрешности, коэффиценты наклона и из них найдем $u$, получаем таблицу:
	
	\begin{table}[H]
		\begin{center}
		\begin{tabular}{|r|r|r|r|r|}
			\hline
 $T\text{, Н}$ & 	$u\text{, м/с}$ & $\sigma_{u}^{\text{случ}} \text{, м/с}$ & $\sigma_{u}^{\text{сист}} \text{, м/с}$ & $\sigma_{u} \text{, м/с}$ \\
 \hline
 10.028 & 140.491515 & 0.235777 & 0.314149 & 0.392785 \\
 \hline
 14.315 & 171.925455 & 0.242994 & 0.384437 & 0.454795 \\
 \hline
 19.229 & 190.557576 & 1.077702 & 0.426100 & 1.158880 \\
 \hline
		\end{tabular}
	\caption{Полученные скорости с погрешностями}
	\end{center}
	\end{table}
	
	
	Из этого полученные значения скорости:
	
	\begin{itemize}
		\item $T = 10.028$, Н -- $u = (140.5 \pm 0.4)$, м/с
		\item  $T = 14.315$, Н -- $u = (171.9 \pm 0.5)$, м/с
		\item  $T = 19.229$, Н -- $u = (190.6 \pm 1.2)$, м/с
	\end{itemize}
	
	С помощью полученных данных построим график зависимости $u^2(T)$, для того, чтобы найти погонную плотность струны $ \rho_l $. 
	
	\begin{figure}[H]
		\begin{center}
			\includegraphics[scale=0.9]{1.4.5 grapg2}
			\label{1.4.5 grapg2}
		\end{center}
	\end{figure}
	
	С помощью формулы \eqref{velocity_of_deformation}, можно понять, что коэффициент наклона $k$, для графика \ref{1.4.5 grapg2}, будет равен: $$k = \dfrac{1}{\rho_l}$$. График строим по МНК, а значит $k$ и его погрешность находится по формулам:
	
	\begin{equation}
		k=\frac{\langle Tu^2\rangle-\langle T\rangle \langle u^2\rangle}{\langle T^2\rangle - \langle T\rangle^2} \approx 1791 \text{ $\dfrac{\text{м}}{\text{кг}}$},
	\end{equation}
	
	\begin{equation}
		\sigma_k^\text{случ}=\frac{1}{\sqrt{N}}\sqrt{\frac{\langle u^4 \rangle - \langle u^2 \rangle^2}{\langle T^2 \rangle - \langle T \rangle^2} - k^2  } \approx 262\text{ $\dfrac{\text{м}}{\text{кг}}$},
	\end{equation}
	
	\begin{equation}
		\sigma_k^{\text{сист}} = k\sqrt{ \varepsilon_{u^2}^2 + \varepsilon_T^2 } \approx 12.8\text{ $\dfrac{\text{м}}{\text{кг}}$},
	\end{equation}
	
	\begin{equation}
		\sigma_k \approx \sigma_k^\text{случ} = 262\text{, $\dfrac{\text{м}}{\text{кг}}$},
	\end{equation}
	
	Таким образом $k = (1800 \pm 300) {, \dfrac{\text{м}}{\text{кг}}} $.
	Тогда \underline{ $\rho_l = (550 \pm 90) \text{, $\dfrac{\text{мг}}{\text{м}}$}$}.
	
	Понятно, что из-за малого числа различных сил натяжения мы не можем подтвердить предполагаемую теоретическую зависимость, а так же получаем достаточно большую погрешность определения линейной плотности струны, но при этом результат отличается от действительности($\rho^{\text{ист}}_l = 568.4\text{, $\dfrac{\text{мг}}{\text{м}}$} $.) не так сильно и попадает в рассчитанную погрешность. 
	\section{Вывод}
	
	\begin{enumerate}
		\item Во время выполнения работы было подтверждено несколько теоретических зависимостей между физическими величинами. С точностью $\varepsilon_{\nu_{1}} = 0.02$ подтверждена формула для определения частот гармоники струны. С точностью  $\varepsilon_{u} = 0,003$ подтверждена формула для определения скорости распространения волны в твердом теле под действием внешней силы.
		\item Первый график имеет вид, предсказанный теоретически., второй содержит слишком мало точек, что подтвердить с  помощью него предложенную зависимость  
		\item С точностью $ \varepsilon_{\rho_{l}} = 0,16 $ определена линейная плотность струны, значение которой в пределах погрешности совпало со значением, указанным на данной установке.
	\end{enumerate}
	
\end{document}