 \documentclass[a4paper,12pt]{article}
\usepackage[a4paper,top=1.3cm,bottom=2cm,left=1.5cm,right=1.5cm,marginparwidth=0.75cm]{geometry}
\usepackage{setspace}
\usepackage{cmap}					
\usepackage{mathtext} 				
\usepackage[T2A]{fontenc}			
\usepackage[utf8]{inputenc}			
\usepackage[english,russian]{babel}
\usepackage{multirow}
\usepackage{graphicx}
\usepackage{wrapfig}
\usepackage{tabularx}
\usepackage{float}
\usepackage{longtable}
\usepackage{hyperref}
\hypersetup{colorlinks=true,urlcolor=blue}
\usepackage[rgb]{xcolor}
\usepackage{amsmath,amsfonts,amssymb,amsthm,mathtools} 
\usepackage{icomma} 
\mathtoolsset{showonlyrefs=true}
\usepackage{euscript}
\usepackage{mathrsfs}

\DeclareMathOperator{\sgn}{\mathop{sgn}}
\newcommand*{\hm}[1]{#1\nobreak\discretionary{}
	{\hbox{$\mathsurround=0pt #1$}}{}}


\title{\textbf{Исследование нелинейных колебаний длиннопериодного маятника}}
\author{Лавыгин Кирилл Дмитриевич Б02-213}
\date{13 января 2023г.}


\begin{document}
	
	\maketitle
	
	\section{Аннотация}
	В своей работе я определю зависимость периода колебаний длиннопериодного маятника от его амплитуды, при этом будет учтено сопротивление воздуха
	\section{Экспериментальная установка}
	Используемый в работе маятник представляет собой два одинаковых груза, закрепленных на жестком стержне, способном вращаться вокруг горизонтальной оси, немного смещенной относительно центра масс системы. На \hyperref[pic1]{Рис. 1} показано расположение грузов и стержня, приведены используемые обозначения.
	
		\begin{figure}[H]
		\begin{center}
			\label{pic1}
			\includegraphics[scale=0.6]{2023_01_09_f475f2a1bde3aebb7460g-1}
			\caption{Длиннопериодный маятник}
		\end{center}
	\end{figure}
	
	Колебания маятника происходят под действием момента силы тяжести. Движение маятника будем описывать временной зависимостью угла отклонения $\varphi$ от положения равновесия. 
	Для измерений использовалась система elliptik, которая способно при помощи светодиода и фотодатчика фиксировать прохождение стержнем точки равновесия и его скорость (так как стержень достаточно толстый). Так-же данная программа обладает удобным интерфейсом, а так-же автоматическим расчетом всех необходимых данных по результатам экспериментов.
	\section{Теоретическая часть}
	\subsection{Безразмерное уравнение движения физического маятника с вязким трением.}
	Уравнение движения физического маятника с учётом вязкого трения:
	
	\begin{equation}
		\label{eq:1}
	I \ddot{\varphi}+b \dot{\varphi}+m g a \sin (\varphi)=0,
\end{equation}
	
	где $I$ - момент инерции, $b$ - коэффициент момента сил вязкого трения, $m$-масса маятника, $g$ - ускорение свободного падения, $a-$ расстоянии от точки подвеса до центра масс. Введя частоту $\omega_{0}$, перепишем уравнение \eqref{eq:1} в виде:
	
	\begin{equation}
	\begin{gathered}
		\ddot{\varphi}+\frac{b}{I} \dot{\varphi}+\frac{m g a}{I} \sin (\varphi)=0 \\
		\omega_{0}=\sqrt{\frac{m g a}{I}}
	\end{gathered}
\end{equation}
\begin{equation}
			\label{eq:2}
\ddot{\varphi}+\frac{b}{I} \dot{\varphi}+\omega_{0}^{2} \sin (\varphi)=0
\end{equation}
	
	И приведём полученное уравнение \eqref{eq:2} к безразмерному виду, удобному для анализа:
	
	\begin{equation}
	\begin{gathered}
		\frac{\ddot{\varphi}}{\omega_{0}^{2}}+\frac{b}{\omega_{0} I} \cdot \frac{\dot{\varphi}}{\omega_{0}}+\sin (\varphi)=0 \\
		\tau=\omega_{0} t, \quad \frac{d}{\omega_{0} d t}=\frac{d}{d \tau}, \quad \beta=\frac{b}{\omega_{0} I} 
	\end{gathered}
\end{equation}

\begin{equation}\label{eq:3}
		\varphi^{\prime \prime}+\beta \varphi^{\prime}+\sin (\varphi)=0
\end{equation}
	
	В уравнении \eqref{eq:3} производные берутся по безразмерному времени $\tau=\omega_{0} t$. Отличие решений уравнения \eqref{eq:3} от решения нелинейного маятника без затухания определяется только значением безразмерного параметра $\beta$, который в дальнейшем предполагается малым:
	
	\begin{equation}
	0<\beta<0.1
\end{equation}
	\section{Фазовая диаграмма нелинейного маятника с затуханием.}
	С помощью численного решения уравнения \eqref{eq:3} можно определить фазовый портрет нелинейного маятника для заданных начальных условий и параметра затухания $\beta$. На рисунке 1 приведён пример фазового портрет для $\beta=0.1$ и с начальной безразмерной скоростью $\varphi_{0}^{\prime}=2$. На рисунке также показаны красным цветом сепаратрисы нелинейного маятника без затухания ограничивающие область колебательных решений.
	
	\begin{center}\label{fazp}
		\includegraphics[width=\textwidth/2]{2023_01_11_30dc2b904fe48f3d7274g-02}
	\end{center}
	
	Рис. 1: Пример фазового портрета для нелинейного маятника с затуханием вычисленный с помощью численного интегрирования уравнения движения \eqref{eq:3}
	
	\section{Закон сохранения энергии для нелинейно- го маятника с затуханием}
	\begin{equation}
	\begin{aligned} \label{int1}
		& \text { Интегрирование уравнения \eqref{eq:3} по углу отклонения: } \\
		& \varphi^{\prime \prime} d \varphi+\beta \varphi^{\prime} d \varphi+\sin (\varphi) d \varphi=0 \rightarrow \int \varphi^{\prime} d \varphi^{\prime}+\beta \int \varphi^{\prime} d \varphi+\int \sin (\varphi) d \varphi=\text { Const } \rightarrow \\
		& \frac{\varphi^{\prime 2}}{2}-\cos (\varphi)=-\beta \int \varphi^{\prime} d \varphi+\text { Const }
	\end{aligned}
\end{equation}
	
	Постоянную интегрирования найдём из начальных условий. В начальный момент времени $(\tau=0)$ будем считать, что маятник проходит положение равновесия $(\varphi=0)$, и имеет скорость $\varphi_{0}^{\prime}$, тогда:
	
	\begin{equation}
		\label{const}
	\text { Const }=\frac{\varphi_{0}^{\prime 2}}{2}-1
\end{equation}
	
	В момент первого максимального отклонения маятника:
	
	\begin{equation}
	\begin{gathered}
		\label{fi0}
		-\cos \left(\varphi_{x}\right)=-\beta \int_{0}^{\varphi_{x}} \varphi^{\prime} d \varphi+\frac{\varphi_{0}^{\prime 2}}{2}-1 \rightarrow \beta \int_{0}^{\varphi_{x}} \varphi^{\prime} d \varphi=\frac{\varphi_{0}^{\prime 2}}{2}+\cos \left(\varphi_{x}\right)-1 \rightarrow \\
		\beta \int_{0}^{\varphi_{x}} \varphi^{\prime} d \varphi=\frac{\varphi_{0}^{\prime 2}}{2}-2 \sin ^{2}\left(\frac{\varphi_{x}}{2}\right)
	\end{gathered}
\end{equation}
	
	При возвращении к положению равновесия угловая скорость маятника $\varphi_{1}^{\prime}$ из-за трения станет меньше чем начальная $\varphi_{0}^{\prime}$ :
	
	\begin{equation}
		\label{fi1}
	\begin{aligned}
		& \varphi_{1}^{\prime 2}-1=\frac{\varphi_{0}^{\prime 2}}{2}-1-\beta \int_{0}^{\varphi_{x}} \varphi^{\prime} d \varphi-\beta \int_{\varphi_{x}}^{0} \varphi^{\prime} d \varphi \rightarrow \\
		& \frac{\varphi_{1}^{\prime 2}}{2}=\frac{\varphi_{0}^{\prime 2}}{2}-\beta \int_{0}^{\varphi_{x}} \varphi^{\prime} d \varphi-\beta \int_{\varphi_{x}}^{0} \varphi^{\prime} d \varphi
	\end{aligned}
\end{equation}
	
	В выражения \eqref{fi0} и \eqref{fi1} входят интегралы учитывающие потери энергии на трение. Потери механической энергии связаны с работой сил трения.
	
	\section{Связь амплитуды колебаний и угловых скоростей в положении равновесия.}
	Для оценки значения интегралов в уравнении \eqref{fi1} используем следующее приближение которое, обосновывается на рисунке 2.
	
	\begin{center}\label{fazd}
		\includegraphics[scale = 0.3]{2023_01_11_30dc2b904fe48f3d7274g-04}
	\end{center}
	
	Рис. 2: Участок траектории маятника на фазовой плоскости
	
	Интегралы, входящие в \eqref{fi1} равны площадям заштрихованных на рисунке 2 областей. Примем следующее приблизительное правило:
	
	\begin{equation} \label{snizu}
	\begin{aligned}
		& \int_{0}^{\varphi_{x}} \varphi^{\prime} d \varphi \approx A \cdot \varphi_{x} \cdot \varphi_{0}^{\prime} \\
		& \int_{\varphi_{x}}^{0} \varphi^{\prime} d \varphi \approx-A \cdot \varphi_{x} \cdot \varphi_{1}^{\prime}
	\end{aligned}
	\end{equation}
	
	которое означает, что отношение площади верхней заштрихованной области к площади прямоугольника со сторонами $\varphi_{0}^{\prime}$ и $\varphi_{x}$ такое же как и отношение нижней заштрихованной области к площади прямоугольника со сторонами $\varphi_{1}^{\prime}$ и $\varphi_{x}$. Знак '-' во втором уравнении введён для учёта отрицательного значения скорости $\varphi_{1}^{\prime}$ (заметим, что оба интеграла положительны). Это приблизительное равенство тем лучше будет выполняться, чем меньше параметр безразмерного затухания. Тогда уравнение \eqref{fi1} преобразуется:
	
	\begin{equation}\label{snizu2}
		\frac{\varphi_{1}^{\prime 2}}{2}=\frac{\varphi_{0}^{\prime 2}}{2}-\beta A \varphi_{x}\left(\varphi_{0}^{\prime}-\varphi_{1}^{\prime}\right) \rightarrow 
		\varphi_{0}^{\prime}+\varphi_{1}^{\prime}=2 \beta A \varphi_{x}
\end{equation}
	
	А уравнение \eqref{fi0} приводится к виду:
	
	\begin{equation}\label{snizuf}
	\begin{gathered}
		\beta A \varphi_{x} \varphi_{0}^{\prime}=\frac{\varphi_{0}^{\prime 2}}{2}-2 \sin ^{2}\left(\frac{\varphi_{x}}{2}\right) \rightarrow 2 \beta A \varphi_{x} \varphi_{0}^{\prime}=\varphi_{0}^{\prime 2}-4 \sin ^{2}\left(\frac{\varphi_{x}}{2}\right) \rightarrow \\
		\varphi_{0}^{\prime 2}+\varphi_{1}^{\prime} \varphi_{0}^{\prime}=\varphi_{0}^{\prime 2}-4 \sin ^{2}\left(\frac{\varphi_{x}}{2}\right) \rightarrow \varphi_{1}^{\prime} \varphi_{0}^{\prime}=-4 \sin ^{2}\left(\frac{\varphi_{x}}{2}\right) \\
		\sin ^{2}\left(\frac{\varphi_{x}}{2}\right)=-\frac{\varphi_{1}^{\prime} \varphi_{0}^{\prime}}{4}
	\end{gathered}
\end{equation}
	
	Здесь знак '-' в правой части учитывает то, что скорости $\varphi_{0}^{\prime}$ и $\varphi_{1}^{\prime}$ всегда должные иметь разные знаки.
	
	Другой способ оценки интегралов \eqref{snizu} состоит в том, что угловая скорость $\varphi^{\prime}$ в подынтегральном выражении выводится через формулы \eqref{int1} и \eqref{const} но с параметром затухания $\beta=0$. (Метод последовательных приближений.) Это приводит к следующему соотношению между амплитудой и скоростями:
	
	\begin{equation}
		\label{sverhu}
	\sin \left(\frac{\varphi_{x}}{2}\right)^{2}=\frac{\varphi_{0}^{\prime 2}+\varphi_{1}^{\prime 2}}{8}
\end{equation}
	
	Среднее между \eqref{snizuf} и \eqref{sverhu}:
	
	\begin{equation} \label{mid}
	\sin \left(\frac{\varphi_{x}}{2}\right)^{2}=\frac{\left(\varphi_{0}^{\prime}-\varphi_{1}^{\prime}\right)^{2}}{16}
\end{equation}
	
	Заметим, что фактически \eqref{snizuf} фактически является оценкой снизу, а \eqref{sverhu} -- оценкой сверху, поэтому наиболее точным является \eqref{mid}.  Отметим, что эти выражения позволяют вычислять амплитуду колебания без явного знания значения параметра затухания. Но информация об этом параметре связана с различием угловых скоростей $\varphi_{0}^{\prime}$ и $\varphi_{1}^{\prime}$.
	
	\section{Период колебаний}
	Под половиной периода колебаний маятника с затуханием понимается время между двумя последовательными моментами времени прохождения маятником положения своего равновесия. Полупериод:
	
\begin{equation}
	\tau_{\frac{1}{2}}=\int_{0}^{\varphi_{x}} \frac{d \varphi}{\varphi^{\prime}}+\int_{\varphi_{x}}^{0} \frac{d \varphi}{\varphi^{\prime}}
\end{equation}
	
	Если полностью пренебречь затуханием, тогда из \eqref{int1}, \eqref{const} и \eqref{fi0} :
	
\begin{equation}
	\begin{gathered}
		\tau_{\frac{1}{4}}=\int_{0}^{\varphi_{x}} \frac{d \varphi}{\varphi^{\prime}}=\int_{0}^{\varphi_{x}} \frac{d \varphi}{2 \sqrt{\sin \left(\frac{\varphi_{x}}{2}\right)^{2}-\sin \left(\frac{\varphi}{2}\right)^{2}}} \approx 
		\frac{\pi}{2}\left(1+\left(\frac{\sin \left(\frac{\varphi_{x}}{2}\right)}{2}\right)^{2}+\frac{9}{4}\left(\frac{\sin \left(\frac{\varphi_{x}}{2}\right)}{2}\right)^{4}+\ldots\right)\approx\\
		 \approx
		\frac{\pi}{2}\left(1+\left(\frac{\varphi_{x}}{4}\right)^{2}+\frac{11}{12}\left(\frac{\varphi_{x}}{4}\right)^{4}+\ldots\right)
	\end{gathered}
\end{equation}
	
Данный интеграл называется эллиптическим и может быть вычислен приближенно. В моем случае вычисления производились с помощью встроенных методов в библиотеку SymPy для Python. При переходе от безразмерных величин обратно к размерным получим следующее выражение для периода нелинейных колебаний:
	
\begin{equation}
	\begin{gathered}
		T=2 \cdot \frac{\tau_{\frac{1}{2}}}{\omega_{0}}=2 \cdot \frac{\tau_{\frac{1}{2}}}{2 \pi / T_{0}}=T_{0} \frac{\tau_{\frac{1}{2}}}{\pi}=T_{0} \frac{2 	\tau_{\frac{1}{4}}}{\pi} \approx T_{0}\left(1+\left(\frac{\varphi_{x}}{4}\right)^{2}+\frac{11}{12}\left(\frac{\varphi_{x}}{4}\right)^{4}+\ldots\right)
	\end{gathered}
\end{equation}
	
	Здесь $T_{0}$ - период линейных колебаний маятника. 
	
	\section{Учёт вязкого сопротивления при движении маятника}
	Исходные выражения для связи скорости и угла:
	
	$$
	\begin{aligned}
		& \frac{\varphi^{\prime 2}}{2}-\cos (\varphi)=-\beta \int \varphi^{\prime} d \varphi+\text { Const } \\
		& 0-\cos \left(\varphi_{x}\right)=\text { Const } \\
		& \frac{\varphi^{\prime 2}}{2}=\cos (\varphi)-\cos \left(\varphi_{x}\right)-\beta \int \varphi^{\prime} d \varphi \\
		& \varphi^{\prime 2}=4\left(\sin \left(\frac{\varphi_{x}}{2}\right)^{2}-\sin \left(\frac{\varphi}{2}\right)^{2}\right)-2 \beta \int \varphi^{\prime} d \varphi
	\end{aligned}
	$$
	
	Обозначим через $\varphi_{0}^{\prime}$ скорость маятника без трения (нулевое приближение), и через $\varphi_{1}^{\prime}$скорость маятника с учётом трения в первом приближении:
	
	\begin{enumerate}
		\setcounter{enumi}{-1}
		\item $\varphi_{0}^{\prime}=2 \sqrt{\sin \left(\frac{\varphi_{x}}{2}\right)^{2}-\sin \left(\frac{\varphi}{2}\right)^{2}}$
		
		\item ${\varphi_{1}^{\prime}}^2=4\left(\sin \left(\frac{\varphi_{x}}{2}\right)^{2}-\sin \left(\frac{\varphi}{2}\right)^{2}\right)-2 \beta \int \varphi_{0}^{\prime} d \varphi$
		
	\end{enumerate}
	
	Далее, для учёта трения, преобразуем интеграл скорости по углу, с учётом нулевого приближения скорости:
	
	\begin{equation}\label{mainint}
			I=\int_{0}^{\varphi_{x}}\frac{\varphi_0^{\prime}}{2} d \varphi=\int_{0}^{\varphi_{x}} \sqrt{\sin \left(\frac{\varphi_{x}}{2}\right)^{2}-\sin \left(\frac{\varphi}{2}\right)^{2}} d \varphi=\sin \left(\frac{\varphi_{x}}{2}\right) \int_{0}^{\varphi_{x}} \sqrt{1-\frac{\sin \left(\frac{\varphi}{2}\right)^{2}}{\sin \left(\frac{\varphi_{x}}{2}\right)^{2}}} d \varphi
	\end{equation}
	
	Максимальный угол отклонения обозначен через $\varphi_{x}$. Применим замену переменной:
	
	$$
	\frac{\sin \left(\frac{\varphi}{2}\right)^{2}}{\sin \left(\frac{\varphi x}{2}\right)^{2}}=\sin (\theta)^{2}, \quad d \varphi=2 \sin \left(\frac{\varphi_{x}}{2}\right) \frac{\sqrt{1-\sin (\theta)^{2}}}{\sqrt{1-\sin \left(\frac{\varphi x}{2}\right)^{2} \cdot \sin (\theta)^{2}}} d \theta
	$$
	
	И дальнейшие преобразования интеграла \eqref{mainint} принимают вид:
	
	$$
	\begin{gathered}
		I=\sin \left(\frac{\varphi_{x}}{2}\right) \int_{0}^{\pi / 2} \sqrt{1-\sin (\theta)^{2}} \cdot 2 \sin \left(\frac{\varphi_{x}}{2}\right) \frac{\sqrt{1-\sin (\theta)^{2}}}{\sqrt{1-\sin \left(\frac{\varphi_{x}}{2}\right)^{2} \cdot \sin (\theta)^{2}}} d \theta= \\
		=2 \sin \left(\frac{\varphi_{x}}{2}\right)^{2} \int_{0}^{\pi / 2} \frac{1-\sin (\theta)^{2}}{\sqrt{1-\sin \left(\frac{\varphi_{x}}{2}\right)^{2} \cdot \sin (\theta)^{2}}} d \theta
	\end{gathered}
	$$
	
Данный интеграл так-же может быть удобно вычислен численно как функция $\varphi_{x}$. Используя полученное выражение для интеграла \eqref{mainint} можно записать в первом приближении скорость маятника $\varphi_{0}^{\prime}$ - до и $\varphi_{1}^{\prime}$ - после текущего максимального отклонения:
	
	$$
	\begin{aligned}
		\varphi_{0}^{\prime 2} & =4 \sin \left(\frac{\varphi_{x}}{2}\right)^{2}+4 I \\
		\varphi_{1}^{\prime 2} & =4 \sin \left(\frac{\varphi_{x}}{2}\right)^{2}-4 I
	\end{aligned}
	$$
	
	Сумма квадратов скоростей в этом приближении не зависит от параметра затухания:
	
	$$
	\sin \left(\frac{\varphi_{x}}{2}\right)^{2}=\frac{\varphi_{0}^{\prime 2}+\varphi_{1}^{\prime 2}}{8}
	$$
	
	А из разности квадратов скоростей получается оценочное выражение для параметра вязкого трения:
	
	$$
		\varphi_{0}^{\prime 2}-\varphi_{1}^{\prime 2}  =8 \beta I \Rightarrow \beta  =\frac{\varphi_{0}^{\prime 2}-\varphi_{1}^{\prime 2}}{8 I}
	$$
	Итоговое выражение для периода колебаний:
	\begin{equation}
	T=\frac{2T_0}{\pi}\int_{0}^{\pi/2} \frac{d \varphi}{\sqrt{4\left(\sin \left(\frac{\varphi_{x}}{2}\right)^{2}-\sin \left(\frac{\varphi}{2}\right)^{2}\right)- 2\beta I}}
	\end{equation}
	
	\section{Полный оборот маятника.}
	Отдельно стоит рассмотреть случай, когда максимальная амплитуда отклонения маятника составляет 180 градусов. Полученные выражения дадут нам возможность достаточно просто получить значение $T_0$ периода линейных колебаний.  Тогда интеграл \eqref{mainint} момента трения упрощается и в первом приближении скорость маятника:
	
	$$
	\begin{gathered}
		\varphi_{-}^{2}{ }_{1}^{2}=4\left(1-\sin \left(\frac{\varphi}{2}\right)^{2}\right)-4 \beta \int_{-\pi}^{\varphi} \sqrt{1-\sin \left(\frac{\varphi}{2}\right)^{2}} d \varphi=4\left(1-\sin \left(\frac{\varphi}{2}\right)^{2}\right)-8 \beta \sin \left(\frac{\varphi}{2}\right)-8 \beta= \\
		4\left(1-\sin \left(\frac{\varphi}{2}\right)\right)\left(1+\sin \left(\frac{\varphi}{2}\right)\right)-8 \beta\left(1+\sin \left(\frac{\varphi}{2}\right)\right)=4\left(1+\sin \left(\frac{\varphi}{2}\right)\right)\left(1-2 \beta-\sin \left(\frac{\varphi}{2}\right)\right)
	\end{gathered}
	$$
	
	A скорость при прохождении положения равновесия:
	
	$$
	\varphi_{1}^{\prime 2}=4-8 \beta \int_{-\pi}^{0} \cos \left(\frac{\varphi}{2}\right) d\left(\frac{\varphi}{2}\right)=4(1-2 \beta)
	$$
	
	-- при проходе от амплитуды в 180 градусов до первого пересечения положения равновесия. Это выражение можно использовать для определения периода линейной системы:
	
	$\varphi^{\prime}=\frac{d \varphi}{d \tau} \approx \frac{\Delta \varphi}{\omega_{0} \Delta t}=2 \sqrt{(1-2 \beta)} \Rightarrow \omega_{0}=\frac{\Delta \varphi}{\Delta t 2 \sqrt{(1-2 \beta)}} \Rightarrow T_{0}=4 \pi \frac{\Delta t}{\Delta \varphi} \sqrt{(1-2 \beta)}$
	
	A также определения эффективного углового размера спицы:
	
	$$
	\frac{\Delta \varphi}{\omega_{0} \Delta t}=\frac{\Delta \varphi_{n}}{\Delta t}=2 \sqrt{(1-2 \beta)} \Rightarrow \Delta \varphi_{n}=\frac{\Delta \varphi}{\omega_{0}}=2 \cdot \Delta t \cdot \sqrt{(1-2 \beta)}
	$$
	
	Эффективный угловой размер спицы $\Delta \varphi_{n}$ позволяет определять измеряемую угловую скорость сразу в безразмерном виде, а это очень удобно для использования уравнения \eqref{mid} для определения амплитуды качаний маятника.
	\section{Измерения скорости и полупериода колебаний с помощњю оптоэлектронного датчикa.}
	Оптоэлектронный датчик располагается так, что спица маятника пересекает луч оптопары тогда, когда маятник проходит через положение равновесия маятника. Спица маятника имеет в месте прохождения луча угловую ширину $\Delta \varphi$; моменты пересечения спицей луча происходят так, как показано на рисунке. Здесь отмечены моменты нарушения $T 1_{i}$ и восстановления $T 2_{i}$ оптического тракта оптопары $(i=1,2,3 .$.$) .$
	
	Зная угловой размер спицы маятника и учитывая, что этот размер достаточно мал $(\Delta \varphi \approx 0.0247)$, можно определять угловую скорость маятника в моменты прохождения положения равновесия:
	
	\begin{equation}
	\begin{gathered}\label{prohod}
		\dot{\varphi}_{i}=\frac{\Delta \varphi}{\Delta t_{i}}, \\
		\Delta t_{i}=T 2_{i}-T 1_{i}
	\end{gathered}
	\end{equation}
	
	\begin{center}
		\includegraphics[width=\textwidth/2]{2023_01_11_30dc2b904fe48f3d7274g-09}
	\end{center}
	
	Рис. 4: Зависимость угла поворота маятника от времени - синяя кривая. Красные кривые - положение граней спицы маятника. Также показаны моменты времени, регистрируемые с помощью оптопары
	
	Полупериод колебания:
	$$
	T_{\frac{1}{2}, i}=\frac{T 2_{i}+T 1_{i}}{2}-\frac{T 2_{i-1}+T 1_{i-1}}{2}
	$$
	
	Эти выражения используются для определения измеряемых в опытах величин: угловой скорости и полупериода колебаний. Период колебаний определяется как сумма двух последовательных полупериодов. Соответствующая этому периоду амплитуда - как среднее арифметическое от амплитуд \eqref{mid} двух последовательных качаний маятника. Если вместо углового размера $\Delta \varphi$ в формуле \eqref{prohod} использовать эффективный угловой размер $\Delta \varphi_{n}$ :
	
	$$
	\Delta \varphi_{n}=\frac{\Delta \varphi}{\omega_{0}}
	$$
	
	то угловая скорость будет определяться в безразмерных единицах.
	
	Далее достаточно лишь применить формулы из секции 9
	\section{Эксперимент}
	Проведем эксперимент, как было описано ранее
	\subsection{1 опыт}
	Результаты предварительного эксперимента:
	\begin{center}
	\begin{tabular}{|r|r|r|}
		\hline
		       $T$,c &      $\beta$ & $\Delta \varphi$ \\
		       \hline
		6.269000 & 0.002727 & 0.024569 \\ \hline
		6.273000 & 0.002708 & 0.024585 \\ \hline
		6.269300 & 0.002680 & 0.024571 \\ \hline
		6.270400 & 0.002642 & 0.024575 \\ \hline
		6.272500 & 0.002708 & 0.024583 \\ \hline
	\end{tabular} 
\end{center}
	Итого получены следующие значения:
	\begin{center}
		$T=6.2708 \pm 0.0018$ c , $\beta=(2.69  \pm 0.03) \cdot 10^{-3}$ , $\Delta \varphi=(2.4577  \pm 0.0007 )\cdot 10^{-2}$
	\end{center}
В ходе основного эксперимента было зарегистрировано 300 полуколебаний, но я использую лишь 105, поскольку при малых амплитудах из-за малого затухания получилось очень много точек. Результаты измерений приведены в \ref{table1}. Значения нанесены на график
\subsection{2 опыт}
Получены следующие значения:
\begin{center}
\begin{tabular}{|r|r|r|}
	 \hline
 $T$,c &      $\beta$ & $\Delta \varphi$ \\
 \hline
  2.784070 & 0.002053 & 0.010911 \\
\hline
 2.784340 & 0.002111 & 0.010912 \\
\hline
  2.783760 & 0.002016 & 0.010910 \\
\hline
  2.783970 & 0.002092 & 0.010911 \\
  \hline
\end{tabular}
\end{center}
Получены следующие значения:
\begin{center}
	$T=2.7840 \pm 0.0002$ c , $\beta=(2.07  \pm 0.04) \cdot 10^{-3}$ , $\Delta \varphi=(1.09110  \pm 0.00008 )\cdot 10^{-2}$
\end{center}
Здесь ещё меньшее количество данных подверглось обработке, так как опыт 1 показал, что даже такое количество избыточно. Результаты измерений приведены в \ref{table2}
\section{Оценка результатов}
Как видно, точки второго эксперимента легли на график очень хорошо, в то время как первые явно смещены. Так же я приложил графики коэффициентов трения, построенные измерительной системой. Из них хорошо видно, что для первого эксперимента существует какая-то неучтенная ошибка. В описании к системе не рекомендовалось использовать конфигурации с периодом больше 4х секунд, поэтому ожидаемо, что полученные данные имеют ошибку. Я связываю её с маленькой величиной момента силы тяжести, из-за которого увеличиваются требования к установке (в частности к величине сухого трения). График, учитывающий вязкое трение я не строил, так как ввиду малости $\beta$ он визуально не отличим от графика без учета трения, а значит вязкое трение можно вовсе не учитывать при расчете периода для нашего осциллятора. Для наглядности так-же построены графики моделей основанных на разложении функции периода до 2го и 4го порядков.Модель второго порядка хорошо работает для углов меньше 1, 4 го -- для меньших 1.5
В ходе работы я ознакомился с теорией, необходимой для описания нелинейных маятников, библиотекой SymPy и подтвердил полученную модель.



\begin{center}
	\includegraphics[width=\textheight, angle=270] {1} 
	\includegraphics[width=\textwidth]{tren1}
	\includegraphics[width=\textwidth]{tren2}
\end{center}


\subsection{Приложение}
Первый эксперимент:
\begin{center}
	\begin{tabular}{lrrrrrrrr}\label{table1}
		A & 5.267750 & 5.408500 & 5.450300 & 5.491500 & 5.533500 & 5.576800 & 5.620000 & 5.663300 \\
		\cline{1-1}
		T & 6.770800 & 6.775700 & 6.780900 & 6.786600 & 6.792100 & 6.797300 & 6.802900 & 6.809300 \\
		\cline{1-1}
		otn & 0.079735 & 0.080516 & 0.081345 & 0.082254 & 0.083131 & 0.083961 & 0.084854 & 0.085874 \\
		\cline{1-1}
	\end{tabular}
	
	\begin{tabular}{lrrrrrrrr}
		A & 5.708600 & 5.752000 & 5.797900 & 5.843000 & 5.889100 & 5.912900 & 5.960300 & 6.008200 \\
		\cline{1-1}
		T & 6.815800 & 6.225000 & 6.829200 & 6.832900 & 6.839500 & 6.846500 & 6.853600 & 6.857300 \\
		\cline{1-1}
		otn & 0.086911 & -0.007304 & 0.089048 & 0.089638 & 0.090690 & 0.091806 & 0.092939 & 0.093529 \\
		\cline{1-1}
	\end{tabular}
	
	\begin{tabular}{lrrrrrrrr}
		A & 6.055100 & 6.103700 & 6.152000 & 6.225900 & 6.301500 & 6.376100 & 6.451800 & 6.529000 \\
		\cline{1-1}
		T & 6.864200 & 6.871300 & 6.882300 & 6.937000 & 6.905800 & 6.917700 & 6.930500 & 6.944000 \\
		\cline{1-1}
		otn & 0.094629 & 0.095761 & 0.097515 & 0.106238 & 0.101263 & 0.103161 & 0.105202 & 0.107355 \\
		\cline{1-1}
	\end{tabular}
	
	\begin{tabular}{lrrrrrrrr}
		A & 6.606700 & 6.688300 & 6.769500 & 6.854100 & 6.939000 & 7.023100 & 7.113300 & 7.208400 \\
		\cline{1-1}
		T & 6.957500 & 6.971700 & 6.985300 & 6.999400 & 7.014600 & 7.030700 & 7.047500 & 7.065700 \\
		\cline{1-1}
		otn & 0.109508 & 0.111772 & 0.113941 & 0.116189 & 0.118613 & 0.121181 & 0.123860 & 0.126762 \\
		\cline{1-1}
	\end{tabular}
	
	\begin{tabular}{lrrrrrrrr}
		A & 7.299400 & 7.396600 & 7.495300 & 7.596100 & 7.698400 & 7.800900 & 7.906500 & 8.015200 \\
		\cline{1-1}
		T & 7.084500 & 7.103800 & 7.124000 & 7.145500 & 7.168100 & 7.190700 & 7.214700 & 7.239000 \\
		\cline{1-1}
		otn & 0.129760 & 0.132838 & 0.136059 & 0.139488 & 0.143092 & 0.146696 & 0.150523 & 0.154398 \\
		\cline{1-1}
	\end{tabular}
	
	\begin{tabular}{lrrrrrrrr}
		A & 8.128200 & 8.204800 & 8.284000 & 8.400200 & 8.528000 & 8.653500 & 8.738100 & 8.871300 \\
		\cline{1-1}
		T & 7.265000 & 7.284100 & 7.301300 & 7.331500 & 7.362000 & 7.393600 & 7.415700 & 7.452400 \\
		\cline{1-1}
		otn & 0.158544 & 0.161590 & 0.164333 & 0.169149 & 0.174013 & 0.179052 & 0.182576 & 0.188429 \\
		\cline{1-1}
	\end{tabular}
	
	\begin{tabular}{lrrrrrrrr}
		A & 9.005000 & 9.142800 & 9.283400 & 9.430000 & 9.583300 & 9.740300 & 9.905700 & 10.077000 \\
		\cline{1-1}
		T & 7.490400 & 7.528600 & 7.568600 & 7.612900 & 7.659600 & 9.740200 & 7.763800 & 7.821900 \\
		\cline{1-1}
		otn & 0.194489 & 0.200580 & 0.206959 & 0.214024 & 0.221471 & 0.553263 & 0.238088 & 0.247353 \\
		\cline{1-1}
	\end{tabular}
	
	\begin{tabular}{lrrrrrrrr}
		A & 10.194000 & 10.314000 & 10.438000 & 10.569000 & 10.701000 & 10.839000 & 10.907000 & 10.976000 \\
		\cline{1-1}
		T & 7.862700 & 7.906300 & 7.950100 & 7.997300 & 8.049700 & 8.104300 & 8.132300 & 8.160900 \\
		\cline{1-1}
		otn & 0.253859 & 0.260812 & 0.267797 & 0.275324 & 0.283680 & 0.292387 & 0.296852 & 0.301413 \\
		\cline{1-1}
	\end{tabular}
	
	\begin{tabular}{lrrrrrrrr}
		A & 11.048000 & 11.120000 & 11.194000 & 11.269000 & 11.345000 & 11.423000 & 11.502000 & 11.579000 \\
		\cline{1-1}
		T & 8.190000 & 8.219400 & 8.251000 & 8.282900 & 8.316800 & 8.353000 & 8.390200 & 8.427700 \\
		\cline{1-1}
		otn & 0.306053 & 0.310742 & 0.315781 & 0.320868 & 0.326274 & 0.332047 & 0.337979 & 0.343959 \\
		\cline{1-1}
	\end{tabular}
	
	\begin{tabular}{lrrrrrrrr}
		A & 11.657000 & 11.739000 & 11.825000 & 11.909000 & 11.997000 & 12.086000 & 12.177000 & 12.270000 \\
		\cline{1-1}
		T & 8.464600 & 8.504000 & 8.546100 & 8.589300 & 8.633300 & 8.679400 & 8.728400 & 8.779700 \\
		\cline{1-1}
		otn & 0.349844 & 0.356127 & 0.362840 & 0.369730 & 0.376746 & 0.384098 & 0.391912 & 0.400092 \\
		\cline{1-1}
	\end{tabular}
	
	\begin{tabular}{lrrrrrrrr}
		A & 12.365000 & 12.463000 & 12.566000 & 12.672000 & 12.781000 & 12.893000 & 13.010000 & 13.129000 \\
		\cline{1-1}
		T & 8.833400 & 9.891500 & 8.950500 & 9.012900 & 9.080000 & 9.150300 & 9.225200 & 9.306300 \\
		\cline{1-1}
		otn & 0.408656 & 0.577390 & 0.427330 & 0.437281 & 0.447981 & 0.459192 & 0.471136 & 0.484069 \\
		\cline{1-1}
	\end{tabular}
	
	\begin{tabular}{lrrrrrrrr}
		A & 13.252000 & 13.384000 & 13.519000 & 13.654000 & 13.796000 & 13.945000 & 14.097000 & 14.259000 \\
		\cline{1-1}
		T & 9.391900 & 9.485500 & 9.584600 & 9.689600 & 9.803000 & 9.927200 & 10.060900 & 10.208900 \\
		\cline{1-1}
		otn & 0.497720 & 0.512646 & 0.528449 & 0.545194 & 0.563277 & 0.583083 & 0.604405 & 0.628006 \\
		\cline{1-1}
	\end{tabular}
	
	\begin{tabular}{lrrrrrrrr}
		A & 14.433000 & 14.624000 & 14.827000 & 15.036000 & 15.264000 & 15.514000 & 15.797000 & 16.135000 \\
		\cline{1-1}
		T & 10.377700 & 10.569100 & 10.786900 & 11.026800 & 11.319500 & 11.672300 & 12.110900 & 12.754700 \\
		\cline{1-1}
		otn & 0.654924 & 0.685447 & 0.720179 & 0.758436 & 0.805113 & 0.861373 & 0.931317 & 1.033983 \\
		\cline{1-1}
	\end{tabular}
\end{center}
Второй эксперимент:
\begin{center}\label{table2}
	\begin{tabular}{lrrrrrrrr}
		A & 0.919396 & 0.943961 & 0.951257 & 0.958448 & 0.965778 & 0.973335 & 0.980875 & 0.988432 \\
		\cline{1-1}
		T & 6.770800 & 6.775700 & 6.780900 & 6.786600 & 6.792100 & 6.797300 & 6.802900 & 6.809300 \\
		\cline{1-1}
		otn & 0.079735 & 0.080516 & 0.081345 & 0.082254 & 0.083131 & 0.083961 & 0.084854 & 0.085874 \\
		\cline{1-1}
	\end{tabular}
	
	\begin{tabular}{lrrrrrrrr}
		A & 0.996339 & 1.003913 & 1.011924 & 1.019796 & 1.027842 & 1.031996 & 1.040269 & 1.048629 \\
		\cline{1-1}
		T & 6.815800 & 6.822500 & 6.829200 & 6.832900 & 6.839500 & 6.846500 & 6.853600 & 6.857300 \\
		\cline{1-1}
		otn & 0.086911 & 0.087979 & 0.089048 & 0.089638 & 0.090690 & 0.091806 & 0.092939 & 0.093529 \\
		\cline{1-1}
	\end{tabular}
	
	\begin{tabular}{lrrrrrrrr}
		A & 1.056814 & 1.065297 & 1.073727 & 1.086625 & 1.099819 & 1.112839 & 1.126052 & 1.139525 \\
		\cline{1-1}
		T & 6.864200 & 6.871300 & 6.882300 & 6.937000 & 6.905800 & 6.917700 & 6.930500 & 6.944000 \\
		\cline{1-1}
		otn & 0.094629 & 0.095761 & 0.097515 & 0.106238 & 0.101263 & 0.103161 & 0.105202 & 0.107355 \\
		\cline{1-1}
	\end{tabular}
	
	\begin{tabular}{lrrrrrrrr}
		A & 1.153087 & 1.167329 & 1.181501 & 1.196266 & 1.211084 & 1.225762 & 1.241505 & 1.258103 \\
		\cline{1-1}
		T & 6.957500 & 6.971700 & 6.985300 & 6.999400 & 7.014600 & 7.030700 & 7.047500 & 7.065700 \\
		\cline{1-1}
		otn & 0.109508 & 0.111772 & 0.113941 & 0.116189 & 0.118613 & 0.121181 & 0.123860 & 0.126762 \\
		\cline{1-1}
	\end{tabular}
	
	\begin{tabular}{lrrrrrrrr}
		A & 1.273986 & 1.290950 & 1.308177 & 1.325770 & 1.343624 & 1.361514 & 1.379945 & 1.398916 \\
		\cline{1-1}
		T & 7.084500 & 7.103800 & 7.124000 & 7.145500 & 7.168100 & 7.190700 & 7.214700 & 7.239000 \\
		\cline{1-1}
		otn & 0.129760 & 0.132838 & 0.136059 & 0.139488 & 0.143092 & 0.146696 & 0.150523 & 0.154398 \\
		\cline{1-1}
	\end{tabular}
	
	\begin{tabular}{lrrrrrrrr}
		A & 1.418639 & 1.432008 & 1.445831 & 1.466111 & 1.488417 & 1.510321 & 1.525086 & 1.548334 \\
		\cline{1-1}
		T & 7.265000 & 7.284100 & 7.301300 & 7.331500 & 7.362000 & 7.393600 & 7.415700 & 7.452400 \\
		\cline{1-1}
		otn & 0.158544 & 0.161590 & 0.164333 & 0.169149 & 0.174013 & 0.179052 & 0.182576 & 0.188429 \\
		\cline{1-1}
	\end{tabular}
	
	\begin{tabular}{lrrrrrrrr}
		A & 1.571669 & 1.595720 & 1.620259 & 1.645845 & 1.672601 & 1.700003 & 1.728871 & 1.758768 \\
		\cline{1-1}
		T & 7.490400 & 7.528600 & 7.568600 & 7.612900 & 7.659600 & 7.740200 & 7.763800 & 7.821900 \\
		\cline{1-1}
		otn & 0.194489 & 0.200580 & 0.206959 & 0.214024 & 0.221471 & 0.234324 & 0.238088 & 0.247353 \\
		\cline{1-1}
	\end{tabular}
	
	\begin{tabular}{lrrrrrrrr}
		A & 1.779189 & 1.800133 & 1.821775 & 1.844638 & 1.867677 & 1.891762 & 1.903631 & 1.915673 \\
		\cline{1-1}
		T & 7.862700 & 7.906300 & 7.950100 & 7.997300 & 8.049700 & 8.104300 & 8.132300 & 8.160900 \\
		\cline{1-1}
		otn & 0.253859 & 0.260812 & 0.267797 & 0.275324 & 0.283680 & 0.292387 & 0.296852 & 0.301413 \\
		\cline{1-1}
	\end{tabular}
	
	\begin{tabular}{lrrrrrrrr}
		A & 1.928240 & 1.940806 & 1.953722 & 1.966812 & 1.980076 & 1.993690 & 2.007478 & 2.020917 \\
		\cline{1-1}
		T & 8.190000 & 8.219400 & 8.251000 & 8.282900 & 8.316800 & 8.353000 & 8.390200 & 8.427700 \\
		\cline{1-1}
		otn & 0.306053 & 0.310742 & 0.315781 & 0.320868 & 0.326274 & 0.332047 & 0.337979 & 0.343959 \\
		\cline{1-1}
	\end{tabular}
	
	\begin{tabular}{lrrrrrrrr}
		A & 2.034530 & 2.048842 & 2.063852 & 2.078513 & 2.093872 & 2.109405 & 2.125287 & 2.141519 \\
		\cline{1-1}
		T & 8.464600 & 8.504000 & 8.546100 & 8.589300 & 8.633300 & 8.679400 & 8.728400 & 8.779700 \\
		\cline{1-1}
		otn & 0.349844 & 0.356127 & 0.362840 & 0.369730 & 0.376746 & 0.384098 & 0.391912 & 0.400092 \\
		\cline{1-1}
	\end{tabular}
	
	\begin{tabular}{lrrrrrrrr}
		A & 2.158100 & 2.175204 & 2.193181 & 2.211681 & 2.230705 & 2.250253 & 2.270673 & 2.291443 \\
		\cline{1-1}
		T & 8.833400 & 8.891500 & 8.950500 & 9.012900 & 9.080000 & 9.150300 & 9.225200 & 9.306300 \\
		\cline{1-1}
		otn & 0.408656 & 0.417921 & 0.427330 & 0.437281 & 0.447981 & 0.459192 & 0.471136 & 0.484069 \\
		\cline{1-1}
	\end{tabular}
\end{center}
\end{document}
