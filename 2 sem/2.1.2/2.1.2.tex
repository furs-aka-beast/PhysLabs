 \documentclass[a4paper,12pt]{article}
\usepackage[a4paper,top=1.3cm,bottom=2cm,left=1.5cm,right=1.5cm,marginparwidth=0.75cm]{geometry}
\usepackage{setspace}
\usepackage{cmap}		
\usepackage{amsmath,amsfonts,amssymb,amsthm,mathtools} 			

\usepackage[T2A]{fontenc}			
\usepackage[utf8]{inputenc}			
\usepackage[english,russian]{babel}
\usepackage{multirow}
\usepackage{graphicx}
\usepackage{wrapfig}
\usepackage{tabularx}
\usepackage{float}
\usepackage{longtable}
\usepackage{hyperref}
\hypersetup{colorlinks=true,urlcolor=blue}
\usepackage[rgb]{xcolor}
\usepackage{icomma} 
\usepackage{euscript}


\DeclareMathOperator{\sgn}{\mathop{sgn}}
\newcommand*{\hm}[1]{#1\nobreak\discretionary{}
	{\hbox{$\mathsurround=0pt #1$}}{}}


\title{\textbf{Определение $C_p / C_v$ методом адиабатического расширения (2.1.2)}}
\author{Лавыгин Кирилл}
\date{\today}


\begin{document}
	
	\maketitle
	
	\section{Аннотация}
	В ходе работы при помощи сосуда, жидкостного манометра, груши и газгольдера с углекислым газом будет определена $\gamma =C_p / C_v$ для углекислого газа, при помощи адиабатического расширения и изохорного нагревания.
	

	
	
	
	\section{Экспериментальная установка.} Используемая для опытов экспериментальная установка состоит из стеклянного сосуда А (объёмом около 20 л), снабженного краном К, и U-образного жидкостного манометра, измеряющего избыточное давление газа в сосуде. Схема установки показана на Рис. 1. 
	
		\begin{figure}[H]	\label{plan2}
		
		\center{\includegraphics[width=1 \linewidth]{1.jpg}}
		\caption{Установка для определения $C_p / C_v$ методом адиабатического расширения газа}
		
	\end{figure}
	Избыточное давление создаётся с помощью резиновой груши, сосединённой с сосудом трубкой с краном $K_1$.
	
	В начале опыта  в стеклянном сосуде А находится исследуемый газ при комнатной температуре $T_1$ и давлении $P_1$, несколько превышающем атмосферное давление  $P_0$. После открытия крана К, соединяющего сосуд А с атмосферой, давление и температура газа будут понижаться. Это уменьшение температуры приближённо можно считать адиабатическим. 
	
	Для адиабатического процесса можно записать следующее уравнение: 
	
	\begin{equation}\label{mk}
		\left(\dfrac{P_1}{P_2}\right)^{\gamma - 1} = \left(\dfrac{T_1}{T_2}\right)^\gamma , 
	\end{equation} 
	
	где индексом "1" обозначено состояние после повышения давления в сосуде и выравнивания температуры с комнатной, а индексом "2"  $-$ сразу после открытия крана и выравнивания давления с атмосферным. 
	
	После того, как кран К вновь отсоединит сосуд от атмосферы , происходит медленное изохорное нагревание газа со скоростью, определяемой теплопроводностью стеклянных стенок сосуда. Вместе с ростом температуры растёт и давление газа. За время порядка $\Delta t_T$  (время установления температуры) система достигает равновесия, и установившаяся температура газа $T_3$ становится равной комнатной температуре $T_1$. 
	
	Тогда используя закон Гей-Люссака для изохорный процесса и уравнение \eqref{mk} найдём $\gamma$:
	
	\begin{equation}\label{acc}
		\gamma = \dfrac{\ln(P_1 / P_0)}{\ln (P_1 / P_3)}.
	\end{equation}
	
	С учётом того, что $P_i = P_0 + \rho g h_i$ и пренебрегая членами второго порядка малости получим из \eqref{acc}:
	
	\begin{equation}\label{r}
		\gamma \approx \dfrac{h_1}{h_1 - h_2}.
	\end{equation}
	
	\section*{Ход работы}
	
	\subparagraph*{1.} Перед началом работы убедимся в том, что краны и места сочленений трубок достаточно герметичны. Для этого нужно наполнить баллон углекислым газом до давления, превышающего атмосферное  и перекроем кран $К_1$. По  U-образному манометру снимем зависимость давления $h$ в баллоне от времени $t$ и построим график зависимости $h = f(t)$. Из графика определим время установления термодинамического равновесия $\Delta t_T$. Стабильное избыточное давление воздуха $h_1$ в баллоне каждый раз измерялось точно. 
	\begin{figure}[H]
		\centering
		\begin{tabular}{|c|c|c|c|}
			\hline
			\multicolumn{2}{|c}{Воздух} & \multicolumn{2}{|c|}{Углекислый газ}\\
			\hline
			$t$, с & $h$, мм &$t$, с & $h$, мм\\
			\hline
			8.990000 & 21.600000 & 10.200000 & 10.200000 \\
			\hline
			14.280000 & 21.200000 & 13.800000 & 10.000000 \\
			\hline
			16.950000 & 21.000000 & 20.200000 & 9.800000 \\
			\hline
			21.220000 & 20.800000 & 27.700000 & 9.600000 \\
			\hline
			31.250000 & 20.400000 & 35.500000 & 9.400000 \\
			\hline
			42.700000 & 20.000000 & 52.700000 & 9.200000 \\
			\hline
			 & & 74.000000 & 9.000000\\
			 \hline
		\end{tabular}
	\end{figure}




	
\begin{figure}[H]
	\centering
	\includegraphics{test}
	\caption[Зависимость h(t) после накачки]{}
	\label{fig:test}
\end{figure}
	
	
	
	\subparagraph*{2.} Откроем кран К на короткое время и закроем его снова. Подождём, пока уровень жидкости в манометре перестанет изменяться. Это произойдёт, когда температура газа в сосуде сравняется с комнатной, примерно через время $\Delta t_T$. Запишем разность уровней жидкости в манометре $h_2$. Проведём серию из 5- измерений сначала для времени открытия крана $\Delta t = 0.5 ~c$, а затем для $\Delta \in [0.5,1.5]$ с и $ \Delta t \in [1.5,5]$ с. По полученным данным вычислим используя формулу \eqref{r} вычислим $\gamma$ и построим график зависимости $\gamma(\Delta t)$.
	\begin{figure}[H]
		\centering
		\begin{tabular}{|r|r|r|r|r|} \hline
			\multicolumn{5}{|c|}{Воздух} \\ \hline
			$\Delta t$, с & $h_1$, см & $h_2$, см & $\gamma$ & $\Delta \gamma$ \\ \hline
			     0.900000 & 24.600000 &  6.400000 & 1.351648 &        0.009002 \\ \hline
			     0.800000 & 20.800000 &  5.400000 & 1.350649 &        0.010639 \\ \hline
			     0.500000 & 25.000000 &  6.600000 & 1.358696 &        0.008900 \\ \hline
			     1.000000 & 24.400000 &  7.200000 & 1.418605 &        0.009533 \\ \hline
			     0.800000 & 25.000000 &  6.200000 & 1.329787 &        0.008735 \\ \hline
			     0.600000 & 23.600000 &  6.600000 & 1.388235 &        0.009627 \\ \hline
			     1.200000 & 25.200000 &  6.400000 & 1.340426 &        0.008724 \\ \hline
			     1.500000 & 23.800000 &  5.800000 & 1.322222 &        0.009134 \\ \hline
			     1.100000 & 27.600000 &  7.000000 & 1.339806 &        0.007962 \\ \hline
			     1.800000 & 26.200000 &  6.200000 & 1.310000 &        0.008239 \\ \hline
			     2.100000 & 22.800000 &  5.400000 & 1.310345 &        0.009470 \\ \hline
			     4.000000 & 27.200000 &  6.000000 & 1.283019 &        0.007823 \\ \hline
			     4.500000 & 28.000000 &  5.800000 & 1.261261 &        0.007521 \\ \hline
			     4.200000 & 23.000000 &  5.000000 & 1.277778 &        0.009227 \\ \hline
		\end{tabular}
	\end{figure}
	
		\begin{figure}[H]
		\centering
		\begin{tabular}{|r|r|r|r|r|} \hline
			\multicolumn{5}{|c|}{Углекислый газ} \\ \hline
 $\Delta t$, с & $h_1$, см & $h_2$, см & $\gamma$ & $\Delta \gamma$ \\ \hline
 0.600000 & 9.000000 & 1.600000 & 1.216216 & 0.022978 \\ \hline
 0.700000 & 9.000000 & 1.600000 & 1.216216 & 0.022978 \\ \hline
 0.700000 & 9.000000 & 1.600000 & 1.216216 & 0.022978 \\ \hline
 0.700000 & 9.000000 & 1.600000 & 1.216216 & 0.022978 \\ \hline
 1.200000 & 9.000000 & 1.600000 & 1.216216 & 0.022978 \\ \hline
 1.600000 & 9.000000 & 1.400000 & 1.184211 & 0.022745 \\ \hline
 1.300000 & 9.000000 & 1.600000 & 1.216216 & 0.022978 \\ \hline
 0.700000 & 9.000000 & 1.400000 & 1.184211 & 0.022745 \\ \hline
 1.900000 & 9.000000 & 1.400000 & 1.184211 & 0.022745 \\ \hline
 2.300000 & 9.000000 & 1.400000 & 1.184211 & 0.022745 \\ \hline
 2.500000 & 9.000000 & 1.400000 & 1.184211 & 0.022745 \\ \hline
 4.400000 & 9.000000 & 1.200000 & 1.153846 & 0.022570 \\ \hline
 5.400000 & 9.800000 & 1.600000 & 1.195122 & 0.020956 \\ \hline
 3.900000 & 9.000000 & 1.200000 & 1.153846 & 0.022570 \\ \hline
		\end{tabular}
	\end{figure}
	
	
	Теперь оценим вклад приборной погрешности при вычислении величины $\gamma$. Измерения $h_1$  и $h_2$ проводились с точностью 2 мм, $\Delta t$ с точностью порядка 0.1 с. Такая точность достигается за счет снятия процесса открытия крана на камеру и последующей обработке этого видео. Пользуясь формулой \eqref{r} можно получить, что погрешность искомой величины 
	
	$$\sigma_\gamma = \sqrt{\left(\dfrac{h_2}{(h_1-h_2)^2}\sigma_{h_1}\right)^2 +
		\left(\dfrac{h_1}{(h_1-h_2)^2}\sigma_{h_2}\right)^2}$$
	Теперь построим искомый график:
	
\begin{figure}[H]
	\centering
	\includegraphics{main}
	\caption[График зависимости $\gamma(\Delta t)]{}
	\label{fig:main}
\end{figure}

Полученные по МНК погрешности свободного члена сильно превышают случайные погрешности отдельных $\gamma$, поэтому систематическими погрешностями можно пренебречь
\begin{align*}
	\sigma_{\gamma(\text{возд})}\approx\sigma_{\gamma(\text{возд})}^\text{случ}=0.011, && \sigma_{\gamma(CO_2)}\approx\sigma_{\gamma(CO_2)}^\text{случ}=0.004
\end{align*}
Для построения аппроксимирующих прямых использовались только точки с $\Delta t<2$ с для воздуха и $\Delta t$<3 с для углекислого газа, поскольку на иных значениях теплопроводность начинает давать ощутимый вклад.

	\subparagraph*{3.} Окончательный результат следует получить экстраполяцией зависимости $\gamma$ от $t$ примерно к значению $\Delta t \in [0.1;0.2]$ с, когда давление уже почти сравнялось с атмосферным, но теплопроводность ещё не так сильно повлияла на уменьшение $\gamma$. 
	
	\begin{align*}
		\gamma_{CO_2}=1.221\pm0.004,  &&\gamma_{\text{возд}}=1.389\pm0.011
	\end{align*}
	
	В то время как табличные значения $\gamma_{CO_2} = 1.300$ и $\gamma_{\text{возд}}=1.400$ , т. е. совпадают с полученным значением в пределах погрешности. 
	

	
	\section*{Вывод}
	В ходе работы были получены значения $\gamma$ для воздуха и углекислого газа. В пределах погрешности они совпали с табличными. Было замечено, что при долгом открытии крана действительно становится ощутимым влияние теплообмена, что не позволяет получить по таким измерениям верный результат. Основной вклад в погрешность вносит именно случайна погрешность, поэтому для увеличения точности измерения необходимо увеличить число измерений, а также проводить его более аккуратно (дольше ждать установления равновесия).
	
	
	
\end{document}
