 \documentclass[a4paper,12pt]{article}
\usepackage[a4paper,top=1.3cm,bottom=2cm,left=1.5cm,right=1.5cm,marginparwidth=0.75cm]{geometry}
\usepackage{setspace}
\usepackage{cmap}		
\usepackage{amsmath,amsfonts,amssymb,amsthm,mathtools} 			
 				
\usepackage[T2A]{fontenc}			
\usepackage[utf8]{inputenc}			
\usepackage[english,russian]{babel}
\usepackage{multirow}
\usepackage{graphicx}
\usepackage{wrapfig}
\usepackage{tabularx}
\usepackage{float}
\usepackage{longtable}
\usepackage{hyperref}
\hypersetup{colorlinks=true,urlcolor=blue}
\usepackage[rgb]{xcolor}
\usepackage{icomma} 
\usepackage{euscript}


\DeclareMathOperator{\sgn}{\mathop{sgn}}
\newcommand*{\hm}[1]{#1\nobreak\discretionary{}
	{\hbox{$\mathsurround=0pt #1$}}{}}


\title{\textbf{Определение теплоёмкости твёрдых тел (2.1.4)}}
\author{Лавыгин Кирилл}
\date{\today}


\begin{document}
	
	\maketitle
	
	\section{Аннотация}
	В ходе работы при помощи калориметра с нагревателем и термометром сопротивления будет измерена теплоемкость калориметра и нескольких образцов.
	
	\section{Теоретическая часть}
	В данной работе происходит измерение теплоемкости твердого тела с использованием следующей принципиальной связи:
	
	\begin{equation}\label{eq:first_eq_of_thermal_capacity}
		C = \frac{\Delta Q}{\Delta T}
	\end{equation}
	
	Определение количества теплоты, переданного телу вызывает некоторые затруднения, так как часть теплоты будет передана окружающей среде через стенки калориметра. В итоге, количество теплоты, переданное телу с учетом теплопотерь через стенки можно определить как:

	
	\begin{equation}\label{eq:termal_with_heat_lossing}
		\Delta Q = P\Delta t - \lambda \left( T - T_{\text{к}} \right) \Delta t,
	\end{equation}
	где $P$ -- мощность нагревателя, $\lambda$ -- коэффициент теплоотдачи стенок калориметра, $T$ -- температура тела, $T_{\text{к}}$ -- температура окружающего калориметр воздуха, $\Delta t$ -- время, в течении которого происходит нагрев.
	
	Из уравнений (\ref{eq:first_eq_of_thermal_capacity}) и (\ref{eq:termal_with_heat_lossing}) Для случаев охлаждения и нагревания получаем:
	
	\begin{align}\label{eq:int}
	\frac{-\mathrm{C}}{\lambda} \ln \frac{T_{\text {cool }}-T_{\mathrm{K}}}{T-T_{\mathrm{K}}}=t &&
	 T_{\text {heat }}(t)=\frac{P}{\lambda}\left(1-e^{\frac{-\lambda}{C} t}\right)+T_{\mathrm{K}}
	\end{align}
	
	Формулы \eqref{eq:int} являются основными расчетными формулами данной работы.
	Так-же будет использован дифференциальный метод по следующим формулам (тоже полученным из (\ref{eq:first_eq_of_thermal_capacity}) и (\ref{eq:termal_with_heat_lossing}))
	\begin{align}
			\lambda  =\frac{P}{\left(T-T_{\mathrm{K}}\right)\left(1-\frac{A}{B}\right)} &&
			C  =\frac{P}{A-B} && A=\left(\frac{d T}{d t}\right)_{\text {heat }} &&B=\left(\frac{d T}{d t}\right)_{\text {cool }}
	\end{align}
	\begin{align}\label{eq:diff}
					C=\frac{P}{\left(d T_{\text {heat }} / d t\right)_{T=T_{\mathrm{K}}}} ~ (\text{упрощенный вид для } T=T_{\mathrm{K}})
	\end{align}
	
	Измерение температуры строится на принципе линейной зависимости сопротивления материала от изменения температуры по закону:
	
	\begin{align}
		R_{T} = R_{0} \left( 1 + \alpha \Delta T \right), && 
		\Delta T(R_T)=\beta R_T + B
	\end{align}
	
	Где $R_{0}$ -- сопротивление термометра при температуре $0^\circ$C, $R_{T}$ -- сопротивление термометра при данной температуре, $\beta$, B -- константы
	\section{Экспериментальная установка}
	\begin{wrapfigure}[9]{r}{0.3\textwidth}
		\vspace{-2.5ex}
		\begin{center}
			\includegraphics[scale=0.07]{schem.jpg}
			\caption{Схема включения нагревателя}
			\label{fig:schem_of_facility}
		\end{center}
	\end{wrapfigure}
	
	Установка состоит из калориметра с пенопластовой изоляцией, помещенного в ящик из многослойной клееной фанеры. Внутренние стенки калориметра выполнены из материала с высокой теплопроводностью. Надежность теплового контакта между телом и стенками обеспечивается их формой: они имеют форму усеченных конусов и плотно прилегают друг к другу. Для выталкивания образца служит винт в донышке внутренней стенки калориметра.
	
	В стенку калориметра вмонтированы электронагреватель и термометр сопротивления. Схема включения нагревателя изображена на рисунке (\ref{fig:schem_of_facility}). Система реостатов позволяет установить нужную силу тока в цепи нагревателя. По амперметру и вольтметру определяется мощность, выделяемая током в нагревателе. Величина сопротивления термометра нагревателя  измеряется мостом постоянного тока.
	
	\begin{figure}[h!]
		\begin{center}
			\includegraphics[scale=0.3]{ustfig.jpg}
			\caption{Устройство калориметра}
			\label{fig:Ris_of_facility}
		\end{center}
	\end{figure}	
	
	На рисунке (\ref{fig:Ris_of_facility}) изображено устройство калориметра.
	\section{Ход работы}
	Запишем параметры экспериментальной установки:
	\begin{table}[h!]
		\centering
		\begin{tabular}{|c|c|c|}
			\hline
			Материал образца: & Алюминий  & Железо    \\ \hline
			Масса образца, г  &$294.2 \pm 0.1 $ &$815.1 \pm 0.1$ \\ \hline
		\end{tabular}
		\caption{Параметры исследуемых образцов}
		\label{tab:param_of_facility}
	\end{table}
	\begin{align*}
		U_0=27.247\pm0.001\text{ В (напряжение на нагревателе)} && I_0=0.2262\pm 0.0001 \text{ A (ток через нагреватель)}
	\end{align*}
	\begin{align*}
		P=U_0I_0=6.163\pm0.003 \text{ Вт(мощность нагревателя)}\\
		\end{align*}
\begin{align*}
	\beta=14.377798 ~ \text{К/Ом} && B=-233.79486 ~ K
\end{align*}
\subsection{Пустой калориметр}
Сначала охлаждаем калориметр при помощи холодного латунного цилиндра (для того, чтобы калориметр достиг комнатной температуры в установившемся режиме). Вынимаем цилиндр и включаем нагреватель. Затем выключаем нагреватель и некоторое время наблюдаем остывание, чтобы по формулам \eqref{eq:int} определить теплоемкость калориметра $C_\text{к}$

Сразу заметим, что так как написанные уравнения верны только для установившегося режима, то аппроксимацию будем проводить по участку, на котором режим уже точно установился

Построим график в координатах $\left(\ln \frac{T_{\text {cool }}-T_{\mathrm{K}}}{T-T_{\mathrm{K}}}, \mathrm{t}\right)$, где $T_{cool}$ -- температура калориметра при охлаждении, $T$ -- температура в начале процесса, $T_K=24.770 \pm 0.006 ~ ^\circ C$ -- средняя комнатная температура.

При таком построении вклад в систематическую погрешность вносит только погрешность $t_{cool}$, поэтому она очень мала
\begin{figure}[h!]
	\centering
	\includegraphics{cool_none.png}
	\caption{График охлаждения пустого калориметра}
\end{figure}

Из графика $\frac{\lambda}{C_K}=(3.70 \pm 0.01) \cdot 10^{-4} ~ 1/c$

Теперь построим график для нагревания в координатах $(T_{heat}, ~ exp(\frac{-\lambda t}{CK}))$

Важно не забыть разместить 0 времени в точке, когда температура калориметра равна комнатной($T_K=24.250 \pm 0.006 ~ ^\circ C$).
\begin{equation*}
	\sigma_{P/\lambda}^{\text{сист}}\approx t T_{heat} \sigma_\frac{\lambda}{C_K} / \exp{\frac{-\lambda t}{C_K}} \approx 0.013 ~ K
\end{equation*}
\begin{figure}[H]
	\centering
	\includegraphics{heat_none.png}
	\caption{График нагревания пустого калориметра}
\end{figure}
\begin{align*}
	P/\lambda \approx 26.06 ~ K  && \sigma_{P/\lambda}^\text{случ}=0.013 ~ K && \sigma_{P/\lambda}=0.02 ~ K
c\end{align*}
Такая большая погрещность обусловлена тем, что через полученную кривую можно провести достаточно широкий пучок прямых, что говорит о плохой точности предложенной млдели
\begin{align*}
	\frac{C_K}{P}=\left(\frac{P}{\lambda}\right)^{-1} \left(\frac{\lambda}{C_K}\right)^{-1}=114.0 ~ ^\circ C/K &&
	\sigma_{\frac{C_K}{P}}=\frac{C_K}{P} \left(\frac{\sigma_{\frac{P}{\lambda}}}{P/\lambda} + \frac{\sigma_{\frac{\lambda}{C_K}}}{\frac{\lambda}{C_K}} \right)=0.2 ~ ^\circ C/K
\end{align*}
\begin{align*}
	C_K=\frac{C_K}{P}P=702.5 ~ \frac{\text{Дж}}{^\circ C}, && \sigma_{C_K}\approx \sigma_\frac{C_K}{P}P=1.4 ~ \frac{\text{Дж}}{^\circ C}, && C_K=702.5 \pm1.4 ~ \frac{\text{Дж}}{^\circ C}
\end{align*}
Теплоемкость калориметра таким образом получена. Стоит отметить, что графики в действительности получились отличными от прямых, но в интересующих нас областях они хорошо приближаются прямыми. Основная погрешность вносится как-раз несовпадением характера зависимости с предсказанным.
\subsection{Алюминиевый конус}
Проведём опыт,аналогичный первому, но после охлаждения калориметра поместим в него алюминиевый конус теплоемкости $C_A$.
Построим график в координатах $\left(\ln \frac{T_{\text {cool }}-T_{\mathrm{K}}}{T-T_{\mathrm{K}}}, \mathrm{t}\right)$, где $T_{cool}$ -- температура калориметра при охлаждении, $T$ -- температура в начале процесса, $T_K=25.268 \pm 0.005 ~ ^\circ C$ -- средняя комнатная температура.

При таком построении вклад в систематическую погрешность вносит только погрешность $T_{cool}$, поэтому она очень мала
\begin{figure}[H]
	\centering
	\includegraphics{cool_al.png}
	\caption{График охлаждения калориметра с алюминием}
\end{figure}

Из графика $\frac{\lambda}{C_K+C_A}=(2.038 \pm 0.003) \cdot 10^{-4} ~ 1/c$

Теперь построим график для нагревания в координатах $(T_{heat}, ~ exp(\frac{-\lambda t}{C_K+C_A}))$

Важно не забыть разместить 0 времени в точке, когда температура калориметра равна комнатной($T_K=25.030 \pm 0.004 ~ ^\circ C$).
\begin{equation*}
	\sigma_{P/\lambda}^{\text{сист}}\approx tT_{heat} \sigma_\frac{\lambda}{C_K+C_A} / \exp{\frac{-\lambda t}{C_K+C_A}} \approx 0.013 ~ K
\end{equation*}
\begin{figure}[H]
	\centering
	\includegraphics{heat_al.png}
	\caption{График нагревания калориметра с алюминием}
\end{figure}
\begin{align*}
	P/\lambda \approx 30.64 ~ K  && \sigma_{P/\lambda}^\text{случ}=0.013 ~ K && \sigma_{P/\lambda}=0.02 ~ K
\end{align*}

\begin{align*}
	\frac{C_K+C_A}{P}=\left(\frac{P}{\lambda}\right)^{-1} \left(\frac{\lambda}{C_K+C_A}\right)^{-1}=160.1 ~ ^\circ C/K &&
	\sigma_{\frac{C_K+C_A}{P}}=\frac{C_K+C_A}{P} \left(\frac{\sigma_{\frac{P}{\lambda}}}{P/\lambda} + \frac{\sigma_{\frac{\lambda}{C_K+C_A}}}{\frac{\lambda}{C_K+C_A}} \right)=1.3 ~ ^\circ C/K
\end{align*}
\begin{align*}
	C_K+C_A=\frac{C_K}{P}P=987 ~ \frac{\text{Дж}}{^\circ C}, && \sigma_{C_K+C_A}\approx \sigma_\frac{C_K+C_A}{P}P=8 ~ \frac{\text{Дж}}{^\circ C}, && C_A=285 \pm 9 ~ \frac{\text{Дж}}{^\circ C}
\end{align*}
\begin{align*}
	C_A^\text{уд}=\frac{C_A}{m_A}=989\pm 30 ~ \frac{\text{Дж}}{\text{кг}^\circ C}
\end{align*}
Полученное значение весьма реалистично, так как табличное: $C_A^\text{уд табл}=920 ~ \frac{\text{Дж}}{\text{кг}^\circ C}$, но оно может варьироваться для разных марок алюминия
\subsection{Железный конус}
Проведём опыт,аналогичный первому, но после охлаждения калориметра поместим в него железный конус теплоемкости $C_F$.
Построим график в координатах $\left(\ln \frac{T_{\text {cool }}-T_{\mathrm{K}}}{T-T_{\mathrm{K}}}, \mathrm{t}\right)$, где $T_{cool}$ -- температура калориметра при охлаждении, $T$ -- температура в начале процесса, $T_K=25.300 \pm 0.005 ~ ^\circ C$ -- средняя комнатная температура.

При таком построении вклад в систематическую погрешность вносит только погрешность $t_{cool}$, поэтому она очень мала
\begin{figure}[H]
	\centering
	\includegraphics{cool_fe.png}
	\caption{График охлаждения калориметра с железом}
\end{figure}

Из графика $\frac{\lambda}{C_K+C_F}=(1.731 \pm 0.002) \cdot 10^{-4} ~ 1/c$


Теперь построим график для нагревания в координатах $(T_{heat}, ~ exp(\frac{-\lambda t}{C_K+C_F}))$

Важно не забыть разместить 0 времени в точке, когда температура калориметра равна комнатной($T_K=25.030 \pm 0.004 ~ ^\circ C$).
\begin{equation*}
	\sigma_{P/\lambda}^{\text{сист}}\approx tT_{heat} \sigma_\frac{\lambda}{C_K+C_F} / \exp{\frac{-\lambda t}{C_K+C_F}} \approx 0.01 ~ K
\end{equation*}
\begin{figure}[H]
	\centering
	\includegraphics{heat_fe.png}
	\caption{График нагревания калориметра с железом}
\end{figure}
\begin{align*}
	P/\lambda \approx 33.06 ~ K  && \sigma_{P/\lambda}^\text{случ}=0.01 ~ K  && \sigma_{P/\lambda}=0.2 ~ K
\end{align*}

\begin{align*}
	\frac{C_K+C_F}{P}=\left(\frac{P}{\lambda}\right)^{-1} \left(\frac{\lambda}{C_K+C_F}\right)^{-1}= 174.7~ ^\circ C/K &&
	\sigma_{\frac{C_K+C_A}{P}}=\frac{C_K+C_A}{P} \left(\frac{\sigma_{\frac{P}{\lambda}}}{P/\lambda} + \frac{\sigma_{\frac{\lambda}{C_K+C_A}}}{\frac{\lambda}{C_K+C_A}} \right)=1.5 ~ ^\circ C/K
\end{align*}
\begin{align*}
	C_K+C_F=\frac{C_K}{P}P=1077 ~ \frac{\text{Дж}}{^\circ C}, && \sigma_{C_K+C_F}\approx \sigma_\frac{C_K+C_F}{P}P=8 ~ \frac{\text{Дж}}{^\circ C}, && C_F=375 \pm 9 ~ \frac{\text{Дж}}{^\circ C}
\end{align*}
\begin{align*}
	C_F^\text{уд}=\frac{C_F}{m_F}=460 \pm 11 ~ \frac{\text{Дж}}{\text{кг}^\circ C}
\end{align*}
Результат опять хорошо совпал с табличным $C_F^\text{уд табл}=450 \frac{\text{Дж}}{\text{кг}^\circ C}$. Но опять-же, её значение варьируется для разных сплавов.
\subsection{Дифференциальный метод}
Используя уравнение \eqref{eq:diff} определим теплоемкости, построением прямой по 10 точкам в окрестности комнатной температуры

Для пустого:
\begin{align*}
	\frac{d T}{d t}=0.0094 \pm 0.0005 ~ \frac{K}{c}, &&
	C_K=660 ~ \frac{\text{Дж}}{^\circ C}, &&
	\sigma_{C_K}=C_K\frac{\sigma_{dT/dt}}{\frac{d T}{d t}}=40 ~ \frac{\text{Дж}}{^\circ C}
\end{align*}
Для алюминия:
\begin{align*}
	\frac{d T}{d t}=0.0086 \pm 0.0005 ~ \frac{K}{c}, &&
	C_K+C_A=820 ~ \frac{\text{Дж}}{^\circ C}, &&
	\sigma_{C_K+C_A}=(C_K+C_A)\frac{\sigma_{dT/dt}}{\frac{d T}{d t}}=50 ~ \frac{\text{Дж}}{^\circ C}
\end{align*}
Для стали:
\begin{align*}
	\frac{d T}{d t}=0.0071 \pm 0.0002 ~ \frac{K}{c}, &&
	C_K+C_F=870 ~ \frac{\text{Дж}}{^\circ C}, &&
	\sigma_{C_K+C_F}=(C_K+C_F)\frac{\sigma_{dT/dt}}{\frac{d T}{d t}}=30 ~ \frac{\text{Дж}}{^\circ C}
\end{align*}
Итого:
\begin{align*}
	C_K=660 \pm 40 ~ \frac{\text{Дж}}{^\circ C}, &&
	C_A=160 \pm 90 ~ \frac{\text{Дж}}{^\circ C}, &&
	C_F=210 \pm 70 ~ \frac{\text{Дж}}{^\circ C}
\end{align*} 
\subsection{Анализ результатов}
Интегральный метод дал достаточно точные результаты, которые сошлись с табличными. Предложенные модели действительно являются верными в случае установившегося режима

Дифференциальный дал удовлетворительный, но низкоточный результат для теплоемкости калориметра и неудовлетворительные для теплоемкостей веществ (не смотря на низкую точность). В первую очередь такое расхождение происходит ввиду того, что режим при комнатной температуре ещё не является установившимся, что видно из графиков.

Также интегральный метод сильно уменьшает случайную погрешность измерений, поскольку опирается на все точки, в то время как дифференциальный использует только 10. 
\end{document}
