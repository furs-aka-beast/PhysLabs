 \documentclass[a4paper,12pt]{article}
\usepackage[a4paper,top=1.3cm,bottom=2cm,left=1.5cm,right=1.5cm,marginparwidth=0.75cm]{geometry}
\usepackage{setspace}
\usepackage{cmap}					
\usepackage{mathtext} 				
\usepackage[T2A]{fontenc}			
\usepackage[utf8]{inputenc}			
\usepackage[russian]{babel}
\usepackage{multirow}
\usepackage{graphicx}
\usepackage{wrapfig}
\usepackage{tabularx}
\usepackage{float}
\usepackage{longtable}
\usepackage{hyperref}
\hypersetup{colorlinks=true,urlcolor=blue}
\usepackage[rgb]{xcolor}
\usepackage{amsmath,amsfonts,amssymb,amsthm,mathtools} 
\usepackage{icomma} 
\mathtoolsset{showonlyrefs=true}
\usepackage{euscript}
\usepackage{mathrsfs}
\usepackage{rotating}

\DeclareMathOperator{\sgn}{\mathop{sgn}}
\newcommand*{\hm}[1]{#1\nobreak\discretionary{}
	{\hbox{$\mathsurround=0pt #1$}}{}}


\title{\textbf{Исследование взаимной диффузии газов (2.2.1)}}
\author{Лавыгин Кирилл}
\date{14.10.22}


\begin{document}
	
	\maketitle
	
	\textbf{Аннотация:} в работе при помощи установки из системы трубок и кранов, форвакуумного насоса, датчика теплопроводности и манометра будет проверен закон Фика, измерен коэффициент взаимной диффузии на различных давлениях.   
	
	\section*{Теоретическая часть}
	Рассмотрим процесс выравнивания концентрации. Пусть концентрации одного из компонентов смеси в сосудах $V_1$ и $V_2$ равны $n_1$ и
	$n_2$. Плотность диффузионного потока любого компонента (т. е. количество вещества, проходящее в единицу времени через единичную поверхность) определяется законом Фика:
	$$j=-D\frac{\partial n}{\partial x},$$ где $D$ — коэффициент взаимной диффузии газов, а $j$ - плотность потока частиц.
	
	В нашем случае ввиду того что, а) объем соединительной трубки мал по сравнению с объемами сосудов, б) концентрацию газов внутри каждого сосуда можно считать постоянной по всему объему. Диффузионный поток в любом сечении трубки(площади $S$ и длины $l$) одинаков. Поэтому, $$J=-DS\frac{n_1-n_2}{l}.$$
	
	Обозначим через $\Delta n_1$ и $\Delta n_2$ изменения концентрации в объемах
	$V_1$ и $V_2$ за время $\Delta t$. Тогда $V_1 \Delta n_1$ равно изменению количества компонента в объеме $V_1$, а $V_2 \Delta n_2$ — изменению количества этого компонента в $V_2$. Из закона сохранения вещества следует, что $V_1n_1+V_2n_2 = const$, откуда $V_1 \Delta n_1 = -V_2\Delta n_2.$ Эти изменения происходят вследствие диффузии, поэтому: $$V_1\Delta n_1=-V_2\Delta n_2.$$
	
	С другой стороны $V_1\Delta n_1=J\Delta t$ и $V_1\frac{dn_1}{dt}=-DS\frac{n_1-n_2}{l}.$ Аналогично $V_2\frac{dn_2}{dt}=DS\frac{n_1-n_2}{l}$
	
	Тогда $$\frac{d(n_1-n_2)}{dt}=-\frac{n_1-n_2}{l} \frac{V_1+V_2}{V_1V_2}.$$
	
	Проинтегрируем и получим, что $$n_1-n_2=(n_1-n_2)_0 e^{-t/\tau},$$ где $(n_1-
	n_2)_0$ — разность концентраций в начальный момент времени, $$\tau=\frac{V_1V_2}{V_1+V_2}\frac{l}{SD}.$$
	
	Для измерения концентраций в данной установке применяются датчики теплопроводности $Д_1$, $Д_2$ (см. рис. 1) используется зависимость теплопроводности газовой смеси от ее состава.
	Для измерения разности концентраций газов используется мостовая схема (рис. 1). Здесь $\text{Д}_1$ и $\text{Д}_2$ — датчики теплопроводности, расположенные в сосудах $V_1$ и $V_2$. Сопротивления $R_1, R_2$ и $R$ служат для установки прибора на нуль (балансировка моста). В одну из диагоналей моста включен гальванометр, к другой подключается небольшое постоянное напряжение. Мост балансируется при заполнении сосудов (и датчиков) одной и той же смесью.
	
	При заполнении сосудов смесями различного состава возникает «разбаланc» моста. При незначительном различии в составах смесей показания гальванометра, подсоединённого к диагонали моста, будут пропорциональны разности концентраций примеси. В процессе диффузии
	разность концентраций убывает по экспоненте, и значит по тому же закону изменяются во времени показания гальванометра ($V_0$ -- начальное напряжение) $$V=V_0 \exp(-t/\tau).$$
	\section*{Экспериментальная установка}
	Схема установки изображена на рис. 1. Там же показана схема электрических соединений и конструкция многоходового крана $K_6$
	
	\begin{figure}[h]
		\center{\includegraphics[scale=0.6]{lab_2_2_1_ust}}
		\caption{схема установки}
	\end{figure}
	Установка состоит из двух сосудов $V_1$ и $V_2$ соединенных краном $K_3$, форвакуумного насоса Ф.Н. с выключателем $Т$, манометра $M$ и системы напуска гелия, включающей в себя краны $K_6$ и $K_7$. Кран $K_5$ позволяет соединять форвакуумный насос либо с установкой, либо с атмосферой. Между форвакуумным насосом и краном $K_5$ вставлен предохранительный баллон П.Б., защищающий кран $K_5$ и установку при неправильной эксплуатации ее от попадания форвакуумного масла из насоса Ф.Н. Сосуды $V_1$ и $V_2$ и порознь и вместе можно соединять как с системой напуска гелия, так и с форвакуумным насосом. Для этого служат краны $K_1$, $K_2$, $K_4$ и $K_5$. Манометр  $M$
	регистрирует давление газа, до которого заполняют тот или другой
	сосуды.
	В нашей установке связывать $\text{Д}_1$ и $\text{Д}_2$ с атмосферой нужно было с помощью $K_4$. 
	
	Для сохранения гелия, а также для уменьшения неконтролируемого попадания гелия в установку (по протечкам в кране $K_6$) между
	трубопроводом подачи гелия и краном $K_6$ поставлен металлический
	кран $K_7$. Его открывают только на время непосредственного заполнения установки гелием. Все остальное время он закрыт.
	
	В силу того, что в сосуд требуется подавать малое давление гелия,
	между кранами $K_7$ и $K_4$ стоит кран $K_6$, снабженный дозатором. Дозатор - это маленький объем, который заполняют до давления гелия в трубопроводе, а затем уже эту порцию гелия с помощью крана $K_6$ впускают в установку.
	
	Описание схемы электрического соединения. $\text{Д}_1$ и $\text{Д}_2$ — сопротивления проволок датчиков парциального давления, которые составляют одно плечо моста. Второе плечо моста составляют сопротивления $r_1$, $R_1$ и $r_2$, $R_2$. $r_1 \ll R_1$, $r_2 \ll R_2$, $R_1$ и $R_2$ спаренные, их подвижные контакты находятся на общей оси. Оба они используются для грубой регулировки моста. Точная балансировка моста выполняется потенциометром R.
	
	\section{Ход работы}
	\begin{enumerate}
		\item Включим питание электрической схемы установки рубильником $B$. Откроем краны $K_1$, $K_2$, $K_3$. Перепишем параметры установки: $$V_1 = V_2 = V = 800 \pm 5 \; \text{см}^{3}, \; \frac{L}{S} = 15.0 \pm 0.1 \; \text{см}^{-1}$$
		Поскольку манометр измеряет разность давления внутри резервуаров с атмосферным в $\frac{\text{кгс}}{\text{cм}^2}$, необходимо записать показание манометра при полностью откачанном сосуде $P_0 = 98.0 \;\frac{\text{кгс}}{\text{cм}^2}$ и в дальнейшем постоянно вычитать из него показания прибора, тем самым будет найдено давление внутри установки (учитываем, что 1 торр = 1 мм. рт. ст и цена деления прибора составляла $3.72$  торр).
		\item Очистим установку от всех газов, которые в ней есть. Для этого откроем кран $K_4$. Включим форвакуумный насос (Ф.Н.) выключателем $T$, находящемся на насосе, и соединим насос с установкой, повернув ручку крана $K_5$ длинным концом рукоятки влево (на установку). Откачиваем установку до тех пор, пока давление не перестанет меняться. Для прекращения откачки ручку крана $K_5$ поставим длинным концом вверх, отключим насос и повернём $K_5$ от установки, чтобы избежать попадания масла из насоса в установку.
		\item  Напустим в установку воздух до рабочего давления (вначале $P_\text{рабочее} = 40 \;$ торр), чтобы сбалансировать мост на рабочем давлении. Для этого используем кран $K_4$. Сбалансируем мост.
		\item Заполним установку рабочей смесью согласно порядку предложенному в указании к работе: в сосуде $V_2$ должен быть воздух, а в сосуде $V_1$ — смесь воздуха, с гелием.
		
		\item Проведём измерения. Для этого откроем кран $К_3$ и затем начнём запись данных на компьютере. Процесс измерений продолжим до тех пор, пока разность концентраций (показания гальванометра) не упадет на $30-50\%.$ Будем продолжать аналогичные измерения при различных значениях $P_{\text{рабочее}}$ в интервале $40-246$ торр. Данный в таблице приведены не все, а по 25 для каждого давления с примерно равным шагом, так-как данных очень много. При этом для всех вычислений и построений использовались все данные.
		
		
		\item Для каждого из давлений построим графики $ln(V/V_0)(t)$ для всех исследуемых давлений. 
		Точность используемых измерительный приборов превышает $0.01 \% $, Поэтому на данном этапе основная погрешность вносится именно случайной погрешностью и погрешностью используемой модели.
		
		Аппроксимацию производим по МНК:
		\begin{equation}
			k_{40}=(-1.8994\pm0.0008) \cdot 10^{-3} \frac{1}{c} \;k_{80}= (-1.1107\pm0.0011) \cdot 10^{-3} \frac{1}{c} \
		\end{equation}
	\begin{equation}
		k_{160}=(-0.6166\pm0.0008) \cdot 10^{-3} \frac{1}{c} \; k_{246}=(-0.3906\pm0.0007) \cdot 10^{-3} \frac{1}{c}
	\end{equation}
		
	
				\begin{sidewaystable}
			\caption{Данные полученные с помощью компьютера}
			%\begin{turn}{90}
			\begin{tabular}{|l|l|l|l|l|l|l|l|l|l|l|l|}
 \hline
\multicolumn{12}{|c|}{$P,$ торр} \\ 
\hline
 \multicolumn{3}{|c|}{40.0}      &
 \multicolumn{3}{c|}{80.0}                 & \multicolumn{3}{c|}{160.0}                 & \multicolumn{3}{c|}{246.0}                 \\
 \hline
$t,$с    & $U,$мв & $\ln{\frac{U}{U_0}}$ &  $t,$с    & $U,$мв & $\ln{\frac{U}{U_0}}$ & $t,$с    & $U,$мв & $\ln{\frac{U}{U_0}}$ & $t,$с    & $U,$мв & $\ln{\frac{U}{U_0}}$ \\
 \hline
 0.000000 & 12.332600 & 0.000000 & 0.000000 & 13.338300 & 0.000000 & 0.000000 & 14.924400 & 0.000000 & 0.000000 & 13.660900 & 0.000000 \\
 \hline
 9.952000 & 12.099400 & -0.019090 & 20.846000 & 13.020300 & -0.024130 & 27.000000 & 14.652900 & -0.018359 & 37.910000 & 13.411400 & -0.018433 \\
 \hline
 20.952000 & 11.845100 & -0.040332 & 42.846000 & 12.678400 & -0.050740 & 54.000000 & 14.374400 & -0.037549 & 75.910000 & 13.155900 & -0.037667 \\
 \hline
 31.952000 & 11.596200 & -0.061569 & 64.846000 & 12.348100 & -0.077137 & 81.000000 & 14.103300 & -0.056589 & 113.910000 & 12.913000 & -0.056303 \\
 \hline
 41.952000 & 11.374500 & -0.080872 & 86.845000 & 12.029600 & -0.103269 & 108.000000 & 13.839700 & -0.075456 & 152.910000 & 12.679200 & -0.074575 \\
 \hline
 52.952000 & 11.135500 & -0.102108 & 108.846000 & 11.722900 & -0.129095 & 135.000000 & 13.587700 & -0.093832 & 190.910000 & 12.461300 & -0.091910 \\
 \hline
 63.952000 & 10.901600 & -0.123337 & 130.846000 & 11.424400 & -0.154888 & 162.000000 & 13.348900 & -0.111563 & 228.910000 & 12.252300 & -0.108824 \\
 \hline
 73.952000 & 10.693900 & -0.142573 &152.846000 & 11.140100 & -0.180088 & 188.999000 & 13.115400 & -0.129210 & 266.910000 & 12.057700 & -0.124834 \\
 \hline
 84.951000 & 10.469900 & -0.163742 & 174.846000 & 10.860500 & -0.205507 & 216.000000 & 12.889800 & -0.146561 & 305.910000 & 11.863700 & -0.141054 \\
 \hline
 95.951000 & 10.251600 & -0.184812 & 196.845000 & 10.589400 & -0.230786 & 243.000000 & 12.669700 & -0.163784 & 343.910000 & 11.681400 & -0.156540 \\
 \hline
 105.951000 & 10.058600 & -0.203818 & 218.845000 & 10.331900 & -0.255403  & 270.000000 & 12.461200 & -0.180378 & 381.910000 & 11.504900 & -0.171765 \\
 \hline
 116.952000 & 9.848600 & -0.224917 & 240.845000 & 10.077900 & -0.280295 & 297.000000 & 12.251800 & -0.197325 & 419.910000 & 11.332100 & -0.186898 \\
 \hline
 127.952000 & 9.645400 & -0.245765 & 262.846000 & 9.835000 & -0.304692 & 324.000000 & 12.054200 & -0.213584 & 458.909000 & 11.164100 & -0.201834 \\
 \hline
 137.952000 & 9.463200 & -0.264836 & 283.846000 & 9.610800 & -0.327752 & 351.000000 & 11.860000 & -0.229826  & 496.910000 & 11.000500 & -0.216597 \\
 \hline
 148.952000 & 9.268600 & -0.285614 & 305.846000 & 9.382000 & -0.351847 & 378.001000 & 11.668300 & -0.246122  & 534.910000 & 10.847600 & -0.230594 \\
 \hline
 159.952000 & 9.077200 & -0.306480 & 327.846000 & 9.160100 & -0.375783 & 405.000000 & 11.481800 & -0.262234 & 572.910000 & 10.694400 & -0.244817 \\
 \hline
 169.952000 & 8.907900 & -0.325308 & 349.846000 & 8.946500 & -0.399377 & 432.000000 & 11.301500 & -0.278062 & 649.409000 & 10.396900 & -0.273030 \\
 \hline
 180.951000 & 8.726600 & -0.345870 & 371.846000 & 8.738000 & -0.422958 & 458.999000 & 11.122100 & -0.294063 & 687.409000 & 10.255900 & -0.286685 \\
 \hline
 191.952000 & 8.549900 & -0.366327 & 393.846000 & 8.537700 & -0.446148 & 486.000000 & 10.947000 & -0.309932 & 725.409000 & 10.120700 & -0.299955 \\
 \hline
 201.952000 & 8.389500 & -0.385265 & 415.846000 & 8.340600 & -0.469504 & 512.999000 & 10.775000 & -0.325769 & 763.410000 & 9.990700 & -0.312883 \\
 \hline
 212.952000 & 8.218500 & -0.405858 & 437.846000 & 8.152400 & -0.492327 & 540.000000 & 10.611100 & -0.341097 & 802.410000 & 9.849700 & -0.327097 \\
 \hline
 223.951000 & 8.050400 & -0.426524 & 459.846000 & 7.964000 & -0.515708 & 566.999000 & 10.446400 & -0.356740 & 840.410000 & 9.724500 & -0.339889 \\
 \hline
 233.952000 & 7.901800 & -0.445156 & 481.846000 & 7.782900 & -0.538711 & 593.999000 & 10.286900 & -0.372126 & 878.410000 & 9.596600 & -0.353129 \\
 \hline
 244.952000 & 7.744900 & -0.465212 & 503.846000 & 7.608200 & -0.561413 & 650.000000 & 9.962400 & -0.404179 & 916.410000 & 9.472700 & -0.366124 \\
 \hline
 255.952000 & 7.589400 & -0.485494 & 525.846000 & 7.430400 & -0.585060 & 677.000000 & 9.805700 & -0.420034 & 955.410000 & 9.351600 & -0.378990 \\
 \hline
			\end{tabular}
			%\end{turn}
		\end{sidewaystable}
		
		
		\begin{figure}[]
			\centering
			\includegraphics[width=1\linewidth]{40,80}
		\end{figure}
		\begin{figure}[]
			\centering
			\includegraphics[width=1\linewidth]{160,246}
		\end{figure}
	
		Видим, что теоретическая зависимость $U = U_0 \cdot e^{\frac{-t}{\tau}}$ не совсем описывает полученные данные при больших значениях давлений. При этом для $p=40$ торр график имеет вид, похожий на прямую. Вероятно, имеет место какая-то зависимость от давления, потому что вид графиков не похож на ошибку измерительных приборов.
		
		
		По угловым коэффициентам экспериментальных прямых и известным параметрам установки рассчитаем коэффициенты взаимной диффузии и их погрешности при выбранных давлениях по формулам : $$D =-  \frac{1}{2} kV \frac{L}{S}, \; \sigma_D = D \sqrt{\big(\frac{\sigma_k}{k}\big)^2 + \big(\frac{\sigma_V }{V}\big)^2 + \big(\frac{\sigma_{L/S}}{L/S}\big)^2},$$ где $k$ - коэффициенты наклонов прямых. Данные представлены в таблице 3.
		
		\begin{table}[!htb]
			\centering
			\begin{minipage}{0.45\linewidth}
				\centering
				\caption{}
				\begin{tabular}{|l|l|l|}
					\hline
					\label{tb1}	
					$P,$ торр & $D, \frac{\text{см}^2}{c}$ & $\sigma_D, \frac{\text{см}^2}{c}$ \\ 
					\hline
					40 &  11.40& 0.10 \\ 
					\hline
					80 & 6.66 & 0.06 \\ 
					\hline
					160 & 3.70 & 0.03 \\ 
					\hline
					246& 2.34 & 0.02 \\ 
					\hline
				\end{tabular}
			\end{minipage}
			\begin{minipage}{0.45\linewidth}
				\centering
				\caption{}
				\begin{tabular}{|c|c|c|}
					\hline
					\label{tb2}
					
					$\frac{1}{P} \cdot 10^{3} , \text{торр}^{-1} $ & $\sigma_{\frac{1}{P}} \cdot 10^{3} ,\;\text{торр}^{-1}$ & $D, \frac{\text{см}^2}{c} $\\ \hline
					25 & 2 & 11.40\\ 
					\hline
					12.5 & 0.6 &6.66 \\ 
					\hline
					6.25 & 0.14 &3.70 \\ 
					\hline
					4.07& 0.06 &2.34 \\ 
					\hline
					
				\end{tabular}
			\end{minipage}
		\end{table}
		
		\item Построим график зависимости коэффициента диффузии от давления $D\left(\frac{1}{P}\right).$ Погрешность $\frac{1}{P}$ рассчитывается по формуле $\sigma{_\frac{1}{P}} = \frac{\sigma_P}{P^2}$, где $\sigma_P = 3.7 \; $торр.
		 
		 Рассчитаем величину коэффициента диффузии при атмосферном давлении. Для этого рассчитаем погрешности получившейся аппроксимации (коэф-т наклона $kd=420\; \text{торр}\cdot \text{см}^2/c$ и свободный член $ad=0.9 \; \text{см}^2/c$).
		 \begin{equation}
		 	\sigma_kd=kd\sqrt{{\varepsilon_kd^\text{сист}}^2+{\varepsilon_kd^\text{случ}}^2}=kd\sqrt{{0.06}^2+{0.05}^2}\approx30\; \text{торр}\cdot \text{см}^2/c
		 \end{equation}
	 	\begin{equation}
	 		\sigma_ad=ad\sqrt{{\varepsilon_ad^\text{сист}}^2+{\varepsilon_ad^\text{случ}}^2}=ad\sqrt{{0.06}^2+{0.2}^2}\approx0.2 \; \text{см}^2/c
	 	\end{equation}
		 Тогда
		 \begin{equation}
		 	D_{\text{атм}} =1.5 \pm 0.2  \; \frac{\text{cм}^2}{c}, \;		 	
		 	D_{\text{табл}} = 0.57\;\frac{\text{cм}^2}{c}
		 \end{equation}
		
		
		\begin{figure}
			\centering
			\includegraphics[width=1\linewidth]{D}
		\end{figure}
		
		
		
		\item Оценим по полученным результатам длину свободного пробега и размер молекулы в атмосфере.  Для этого воспользуемся следующими формулами. $$\lambda = D_{\text{атм}} \frac{3}{\langle v \rangle},\; где \langle v \rangle = \sqrt{\frac{8RT}{\pi \mu_{He}}}, \; \Pi \approx \frac{kT}{\sqrt{2}\lambda P_{\text{атм}}},$$ 
		
		\begin{equation}
			\varepsilon_\lambda=\varepsilon_\Pi=0.13
		\end{equation}
		
		где $T$ -- температура газа, $\mu_{He}$ -- молярная масса гелия, $P$ -- давление в сосуде,  $\Pi$ - площадь эффективного сечения частиц, $r = \frac{1}{2}\sqrt{\frac{\Pi}{\pi}}, \; \sigma_r=\frac{\sigma_\Pi}{4}\sqrt{\frac{1}{\pi\Pi}} $.
		
		Итак, $$\lambda \approx 360\pm 50 \; \text{нм}, \; \Pi \approx (8.2\pm1.1)\cdot 10^{-20} \; \text{м}^2, \; r \approx (0.80\pm 0.05)  \cdot 10^{-10}\; \text{м}.$$ Табличное значение для размера молекулы $r = 1.0 \cdot 10^{-10}\; \text{м}.$
		\section{Вывод}
		В данной работе было проверено, что закон $U = U_0 \cdot e^{\frac{-t}{\tau}}$ выполняется с хорошей точностью для малых давлений, для больших давлений -- ощутимо хуже. Вероятно, это одна из причин возникшей в дальнейшем ошибки.
		
		Так же в работе было найдено значение коэффициента диффузии гелия в воздухе, а так же оценены длина свободного пробега и размер молекулы гелия. К сожалению, результат не сошёлся с табличным. Характер ошибки не похож на погрешность какого-либо из производимых измерений, поэтому ошибка вероятней всего кроется в модели или способе измерения теплопроводности. В любом случае методика данной работы позволяет определить порядок вычисленных величин и подтвердить исследуемую зависимость.
		\end{enumerate}
	
	
\end{document}
