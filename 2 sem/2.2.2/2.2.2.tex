 \documentclass[a4paper,12pt]{article}
\usepackage[a4paper,top=1.3cm,bottom=2cm,left=1.5cm,right=1.5cm,marginparwidth=0.75cm]{geometry}
\usepackage{setspace}
\usepackage{cmap}		
\usepackage{amsmath,amsfonts,amssymb,amsthm,mathtools} 			

\usepackage[T2A]{fontenc}			
\usepackage[utf8]{inputenc}			
\usepackage[english,russian]{babel}
\usepackage{multirow}
\usepackage{graphicx}
\usepackage{wrapfig}
\usepackage{tabularx}
\usepackage{float}
\usepackage{longtable}
\usepackage{hyperref}
\hypersetup{colorlinks=true,urlcolor=blue}
\usepackage[rgb]{xcolor}
\usepackage{icomma} 
\usepackage{euscript}

\DeclareMathOperator{\sgn}{\mathop{sgn}}
\newcommand*{\hm}[1]{#1\nobreak\discretionary{}
	{\hbox{$\mathsurround=0pt #1$}}{}}


\title{\textbf{Измерение теплопроводности воздуха при разных давлениях (2.2.2)}}
\author{Лавыгин Кирилл}
\date{\today}


\begin{document}
	\maketitle
	
	\section{Аннотация}
	В данной работе мы наблюдаем за изменением теплопроводности воздуха с помощью платиновой нити, амперметра, вольтметра, реостата и источника тока. Определяем коэффициент теплопередачи при высоких и низких давлениях
	
	\section{Теоретические сведения}
	
	Теплопроводность — это процесс передачи энергии от нагретых частей системы к холодным за счет хаотического движения частиц среды (молекул, атомов и т.п.). В газах теплопроводность осуществляется за счет непосредственной передачи кинетической энергии от быстрых молекул к медленным при их столкновениях. Перенос тепла описывается законом Фурье:
	\begin{equation}
		\vec{q} = -\varkappa\cdot\nabla T,
	\end{equation}
	где $\vec{q}$ --- плотность потока энергии, $\varkappa$ --- коэффициент теплопроводности. Система, используемая в данной установке, имеет цилиндрическую симметрию (пренебрегая краевым эффектами), поэтому имеем 
	\begin{equation}
		q = -\varkappa\frac{dT}{dr},
	\end{equation}
	где $r$ --- расстояние от оси симметрии системы.
	
	Закон Фурье применим при условиях $$\lambda\ll r\quad \text{и}\quad \lambda |\nabla T|\ll T, $$
	где $\lambda$--- длина свободного пробега молекул газа, а $r$--- характерный размер системы.
	
	Для количественного описания способности некоторой системы к теплопередаче в целом используют коэффициент $K$, называемый \emph{тепловым сопротивлением}, равный отношению перепада температур $\Delta T$ в системе к полному потоку энергии $Q$ через нее:
	\begin{equation}
		K = \frac{\Delta T}{Q}
	\end{equation}
	
	
	\section{Экспериментальная установка}
	\begin{wrapfigure}{L}{8cm}
		\centering
		\includegraphics[scale = 0.6]{ustan}
		\caption{Вакуумная часть установки}
		\label{facility}
	\end{wrapfigure}
	Схема установки приведена на рис. \eqref{facility} Внутренняя полость тонкостенной цилиндрической стеклянной колбы, на оси которой натянута металлическая (платиновая) нить, подсоединена к вакуумной установке. Колба заполнена воздухом и расположена вертикально. Контактные провода от нити выведены наружу через стеклянную вакуумную «слезку».
	
	Вакуумная установка состоит из форвакуумного насоса, стрелочного вакуумметра $M$ и U-образного масляного манометра. Вакуумметр служит для измерения высоких давлений вплоть до ~10 торр (он показывает разность давлений между установкой и атмосферой, так что нуль на его шкале соответствует атмосферному давлению в установке). U-образный манометр заполнен маслом с плотностью $\rho$ и предназначен для измерения низких давлений (вплоть до ~0,1 торр). Кран К$_1$ служит для соединения установки и насоса с атмосферой, кран К$_2$ — для отсоединения откачиваемого объема от насоса, кран К$_3$ — для соединения колен U-образного манометра.
	
	Металлическая нить служит как источником тепла, так и датчиком температуры (термометром сопротивления). В рабочем диапазоне температур (20–40 $^\circ C$) сопротивление платины зависит от температуры практически линейно:
	\begin{equation}\label{R}
		R(t)=R_{0}\left(1+\alpha_{0} t\right)
	\end{equation}
	где $t$ --- температура в $^\circ C$, $R_0$--- сопротивление про 0$^\circ C$, и
	\begin{equation}
		\alpha_{0}=\frac{1}{R_{0}} \frac{d R}{d t}=3,92 \cdot 10^{-3} ~^\circ C^{-1}.
	\end{equation}
	
	\begin{figure}[!h]
		\centering
		\includegraphics[width=50mm]{elustan}
		\caption{Электрическая схема измерений}
		\label{elfacility}
	\end{figure}
	
	Электрическая схема установки приведена на рис. \eqref{elfacility} Ток $I$ через сопротивление $R_\text{н}$ и напряжение $U$ на нем измеряются цифровыми мультиметрами, один из которых работает в режиме амперметра, а другой — вольтметра. Сопротивление $R_\text{н}$ находится по закону Ома. Те же измерения позволяют определить мощность нагрева проволоки как джоулево тепло. Ток в цепи регулируется с помощью магазина сопротивлений, включенного последовательно с источником тока.
	\section{Ход работы}
	\begin{enumerate}
		\item Оценим, когда длина свободного пробега сравняется с радиусом нити:
		\begin{equation*}
			P_1 \approx \frac{kT}{r_{н} \pi d^2}\approx 430\text{ Па}= 48 \text{ мм.масл.ст}.
		\end{equation*}
	
	\item И запишем все известные данные установки:
	\begin{align*}
		P_\text{атм}=739 \pm 1 \text{ торр} && r_\text{н}=0.025 \pm 0.001 \text{ мм (радиус нити)} && \rho=0.885 ~ \frac{\text{г}}{\text{см}^3} \\
		L= 202\pm2 \text{ мм} \text{(длина нити)} && d= 3.8 ~ \frac{\text{торр}}{\text{дел}} ~ \text{цена деления} && T_0=25.2 \pm 0.1 ~ ^\circ C  \\ \alpha = 3.92 \cdot 10^{-3}~ ^\circ C^{-1} && \sigma_U^\text{сист}=0.1 ~ \text{ мВ} && \sigma_I^\text{сист}=0.01 ~ \text{ мА}
	\end{align*}
	
	\item Проведя подготовку экспериментальной установки измерим измерим зависимость сопротивления нити от подаваемой на нее мощности:
	\begin{table}[H]
		\begin{center}
			\begin{tabular}{|l|r|r|r|r|r|r|r|}
				\hline
				$U,$ мВ & 111.7 & 227.1 & 329.0 & 450.5 & 559.3 & 674.2 & 791.8 \\
				\cline{1-8}
				$I$, мА & 10.14 & 20.55 & 29.69 & 40.46 & 50.03 & 59.98 & 69.81 \\
				\cline{1-8}
				$R$, Ом & 11.02 & 11.05 & 11.08 & 11.13 & 11.18 & 11.24 & 11.34 \\
				\cline{1-8}
				$Q$, мВт & 1.133 & 4.667 & 9.768 & 18.23 & 27.98 & 40.44 & 55.28 \\
				\cline{1-8}
			\end{tabular}
			
		\end{center}
	\end{table}
	\begin{figure}[!h]
		\centering
		\includegraphics{rq}
		\caption{График зависимости $R(Q)$}
		\label{graph1}
	\end{figure}
	\item Экстраполируя график \eqref{graph1} к нулевому значению, получаем (используя формулу \eqref{R}):
	\begin{align*}
		R(Q=0) = (11.020 \pm 0.008)\text{ Ом,} && R_0 = (10.029\pm0.008)\text{ Ом,}
	\end{align*}
		\begin{align*}
		 R_{max} = (12.2\pm0,010)\text{ Ом (максимальная температура при } \Delta T=20 ~ ^\circ C) 
	\end{align*}
	\item Проводим измерения, аналогичные пункту 2 при различных давлениях, используя полученные данные для измерения температуры.
	\begin{table}[H]
		\centering
		\begin{minipage}{.49\linewidth}
			\centering
			\begin{tabular}{|c|c|c|c|c|}
				\hline
				\multicolumn{5}{|c|}{$P_2 = 70.8$ Па}\\
				\hline
				    №  & $U,$м &  $I,$ мА &        $T,$ $10 \cdot ^\circ$С &       $Q$ мВт \\
				    \hline
				0 & 112.2 & 10.140000 & 266.353306 & 1.137708 \\
				\cline{1-5}
				1 & 228.1 & 20.550000 & 275.180395 & 4.687455 \\
				\cline{1-5}
				2 & 331.1 & 29.660000 & 291.329825 & 9.820426 \\
				\cline{1-5}
				3 & 454.6 & 40.360000 & 316.904859 & 18.347656 \\
				\cline{1-5}
				4 & 574.1 & 50.390000 & 349.879504 & 28.928899 \\
				\cline{1-5}
				5 & 694.3 & 60.160000 & 387.501237 & 41.769088 \\
				\cline{1-5}
				6 & 821.5 & 70.100000 & 432.844151 & 57.587150 \\
				\cline{1-5}
			\end{tabular}
		\end{minipage}
		\begin{minipage}{.49\linewidth}
			\centering
			\begin{tabular}{|c|c|c|c|c|}
				\hline
				\multicolumn{5}{|c|}{$P_2 = 106.2$ Па}\\
				\hline
				№  & $U,$м &  $I,$ мА &        $T,$ $10 \cdot^\circ$С &       $Q$ мВт \\
				\hline
			0 & 111.8 & 10.120000 & 261.857275 & 1.131416 \\
			\cline{1-5}
			1 & 227.6 & 20.550000 & 268.985303 & 4.677180 \\
			\cline{1-5}
			2 & 335.0 & 30.130000 & 279.949361 & 10.093550 \\
			\cline{1-5}
			3 & 452.0 & 40.430000 & 295.565204 & 18.274360 \\
			\cline{1-5}
			4 & 567.3 & 50.280000 & 321.790697 & 28.523844 \\
			\cline{1-5}
			5 & 685.1 & 60.040000 & 354.358908 & 41.133404 \\
			\cline{1-5}
			6 & 813.1 & 70.200000 & 398.126489 & 57.079620 \\
			\cline{1-5}
			\end{tabular}
		\end{minipage}
	\begin{minipage}{.49\linewidth}
		\centering
		\begin{tabular}{|c|c|c|c|c|}
			\hline
			\multicolumn{5}{|c|}{$P_2 = 194.7$ Па}\\
			\hline
			№  & $U,$м &  $I,$ мА &        $T,$ $10 \cdot^\circ$С &       $Q$ мВт \\
			\hline
			0 & 110.9 & 10.060000 & 255.854911 & 1.115654 \\
			\cline{1-5}
			1 & 277.1 & 25.020000 & 268.912507 & 6.933042 \\
			\cline{1-5}
			2 & 447.9 & 40.120000 & 291.539946 & 17.969748 \\
			\cline{1-5}
			3 & 622.8 & 55.100000 & 326.951572 & 34.316280 \\
			\cline{1-5}
			4 & 806.9 & 70.160000 & 377.307399 & 56.612104 \\
			\cline{1-5}
		\end{tabular}
	\end{minipage}
	\begin{minipage}{.49\linewidth}
		\centering
		\begin{tabular}{|c|c|c|c|c|}
			\hline
			\multicolumn{5}{|c|}{$P_2 = 265.5$ Па}\\
			\hline
			№  & $U,$м &  $I,$ мА &        $T,$ $10 \cdot^\circ$С &       $Q$ мВт \\
			\hline
			0 & 111.2 & 10.110000 & 249.528671 & 1.124232 \\
			\cline{1-5}
			1 & 332.1 & 30.020000 & 265.725967 & 9.969642 \\
			\cline{1-5}
			2 & 560.2 & 50.010000 & 301.152211 & 28.015602 \\
			\cline{1-5}
			3 & 801.1 & 70.080000 & 359.577380 & 56.141088 \\
			\cline{1-5}
		\end{tabular}
	\end{minipage}
\begin{minipage}{.49\linewidth}
	\centering
	\begin{tabular}{|c|c|c|c|c|}
		\hline
		\multicolumn{5}{|c|}{$P_2 = 354.0$ Па}\\
		\hline
		№  & $U,$м &  $I,$ мА &        $T,$ $10 \cdot^\circ$С &       $Q$ мВт \\
		\hline
	0 & 109.7 & 10.000000 & 242.141971 & 1.097000 \\
	\cline{1-5}
	1 & 338.9 & 30.720000 & 257.903107 & 10.411008 \\
	\cline{1-5}
	2 & 566.0 & 50.690000 & 292.024347 & 28.690540 \\
	\cline{1-5}
	3 & 798.9 & 70.210000 & 346.209798 & 56.090769 \\
	\cline{1-5}
	\end{tabular}
\end{minipage}
\begin{minipage}{.49\linewidth}
	\centering
	\begin{tabular}{|c|c|c|c|c|}
		\hline
		\multicolumn{5}{|c|}{$P_2 = 433.7$ Па}\\
		\hline
		№  & $U,$м &  $I,$ мА &        $T,$ $10 \cdot^\circ$С &       $Q$ мВт \\
		\hline
		0 & 110.7 & 10.110000 & 236.936274 & 1.119177 \\
		\cline{1-5}
		1 & 338.1 & 30.680000 & 254.925999 & 10.372908 \\
		\cline{1-5}
		2 & 558.1 & 50.080000 & 286.488664 & 27.949648 \\
		\cline{1-5}
		3 & 795.4 & 70.010000 & 341.757322 & 55.685954 \\
		\cline{1-5}
	\end{tabular}
\end{minipage}
\begin{minipage}{.49\linewidth}
	\centering
	\begin{tabular}{|c|c|c|c|c|}
		\hline
		\multicolumn{5}{|c|}{$P_2 = 1519$ Па}\\
		\hline
		№  & $U,$м &  $I,$ мА &        $T,$ $10 \cdot^\circ$С &       $Q$ мВт \\
		\hline
		0 & 110.3 & 10.050000 & 243.446731 & 1.108515 \\
		\cline{1-5}
		1 & 333.3 & 30.210000 & 258.124514 & 10.068993 \\
		\cline{1-5}
		2 & 558.3 & 50.070000 & 288.072421 & 27.954081 \\
		\cline{1-5}
		3 & 793.7 & 70.100000 & 331.868576 & 55.638370 \\
		\cline{1-5}
	\end{tabular}
\end{minipage}
\begin{minipage}{.49\linewidth}
	\centering
	\begin{tabular}{|c|c|c|c|c|}
		\hline
		\multicolumn{5}{|c|}{$P_2 = 10664$ Па}\\
		\hline
		№  & $U,$м &  $I,$ мА &        $T,$ $10 \cdot^\circ$С &       $Q$ мВт \\
		\hline
		0 & 110.3 & 10.050000 & 243.446731 & 1.108515 \\
		\cline{1-5}
		1 & 331.4 & 30.030000 & 258.852832 & 9.951942 \\
		\cline{1-5}
		2 & 558.2 & 50.080000 & 286.997087 & 27.954656 \\
		\cline{1-5}
		3 & 793.6 & 70.120000 & 330.683185 & 55.647232 \\
		\cline{1-5}
	\end{tabular}
\end{minipage}
\centering
\begin{tabular}{|c|c|c|c|c|}
	\hline
	\multicolumn{5}{|c|}{$P_2 = 10664$ Па}\\
	\hline
	№  & $U,$м &  $I,$ мА &        $T,$ $10 \cdot^\circ$С &       $Q$ мВт \\
	\hline
0 & 111.100000 & 10.100000 & 249.780519 & 1.122110 \\
\cline{1-5}
1 & 333.800000 & 30.210000 & 262.338653 & 10.084098 \\
\cline{1-5}
2 & 558.600000 & 50.060000 & 290.165438 & 27.963516 \\
\cline{1-5}
3 & 793.800000 & 70.100000 & 332.231798 & 55.645380 \\
\cline{1-5}
\end{tabular}
		
	\end{table}
Построим график T(Q) для всех давлений. при которых производились измерения:

Сразу заметим, что:
\begin{align*}
	\varepsilon_K^\text{сист}=0.002 ~ << ~ \varepsilon_K^\text{случ} && \sigma_k\approx \sigma_k^\text{случ} && \varepsilon_T(Q=0)=\sqrt{0.0002^2+{\varepsilon_T^{\text{случ}}(Q=0)}^2}
\end{align*}

	\begin{figure}
	\centering
	\includegraphics{raw}
	\caption{График зависимости $T(Q)$}
	\label{graph2}
	\end{figure}
	
	Полученные для этих давлений коэффициенты теплового сопротивления:
	\bgroup
	\def\arraystretch{1.2}%
	\begin{table}[H]
		\centering
		\begin{tabular}{|c|c|c|c|c|c|c|c|c|c|}
			\hline
$P,$ Па & 70.8 & 106.2 & 194.7 & 265.5 & 354.0 & 433.7 & 1519 & 10664 & 33458 \\
\cline{1-10}
$K,$ $\frac{^\circ\text{С}}{\text{мВт}}$ & 2.98 & 2.43 & 2.18 & 2.01 & 1.9 & 1.91 & 1.623 & 1.591 & 1.520 \\
\cline{1-10}
$\sigma_K,$ $\frac{^\circ\text{С}}{\text{мВт}}$ & 0.02 & 0.07 & 0.02 & 0.03 & 0.04 & 0.03 & 0.015 & 0.017 & 0.013 \\
\cline{1-10}
$T(Q=0)$ & 26.24 & 25.57 & 25.31 & 24.62 & 23.88 & 23.45 & 24.19 & 24.23 & 24.76 \\
\cline{1-10}
$\sigma_{T(Q=0)}$ & 0.07 & 0.14 & 0.06 & 0.09 & 0.08 & 0.08 & 0.06 & 0.06 & 0.06 \\
\cline{1-10}
		\end{tabular}
		\caption{Коэффициенты теплового сопротивления }
		\label{tab}
	\end{table}
	\egroup
	
	Как видно зависимости действительно получились прямыми с хорошей точностью, что подтверждает предложенную модель. Температуры при нулевой теплопередачи действительно получились отличными от комнатной. При низких давлениях, это, вероятно, обусловлено тем, что при низких давления имеется разница температур между стенкой и слоем газа у стенки (а мы получили именно температуру слоя газа), а при высоких - причина, вероятно, в увеличении комнатной температуры или общем нагревом установки.

	По полученным данным построим график зависимости теплового сопротивления системы от логарифма давления и от $1/P$ (для низких давлений):
	
		\begin{figure}
		\centering
		\includegraphics{ln}
		\caption{График зависимости $K(ln(P/P_0))$, где $P_0= 1$ Па}
		\label{graph3}
	\end{figure}
		\begin{figure}
	\centering
	\includegraphics{1fP}
	\caption{График зависимости $K(1/P)$}
	\label{graph3}
\end{figure}
	
	
	И, действительно, заметна область, где теплопередача перестает зависеть от давления $(K \approx const)$.
	 
	Из последнего графика получаем:
	\begin{align*}
		K_\infty = (168 \pm 5) \text{К}/\text{Вт} && A=8932 \pm 600~ \frac{K}{\text{Вт} \cdot \text{Па}}
	\end{align*}
	
	
	Теперь, с помощью полученного $K_\infty$, найдем коэффициент теплопроводности воздуха:
	\begin{align*}
		\varkappa \approx \frac{1}{2\pi LK_\infty} \ln\frac{R}{r_{\text{н}}} =  24.8\cdot 10^{-3}\: \text{Вт}/\text{м}\cdot\text{К}, && \sigma_\varkappa\approx\varkappa\frac{\sigma_{K_\infty}}{K_\infty}=0.7 ~\text{Вт}/\text{м}\cdot\text{К}
	\end{align*}
	
	\item Так же, с помощью $A$, можно получить коэффициент аккомодации:
	\begin{align*}
		s =\frac{1}{Lr_\text{н} C_V \cdot A} \sqrt {\frac{\mu RT_\text{к}}{2\pi}} \approx 0.60  &&
		 \sigma_s=s\frac{\sigma_A}{A}=0.04
	\end{align*}
\end{enumerate}
\section{Вывод}

В работе был проверен метод по определению коэффициента теплопроводности воздуха при комнатной температуре в зависимости от давления.

Был получен коэффициент теплопроводности:
\begin{equation*}
\varkappa  =  (24.8\pm0.7)\cdot 10^{-3}\: \text{Вт}/\text{м}\cdot\text{К}.
\end{equation*}
Который неплохо совпал с табличным значением $\varkappa=0.022 \text{Вт}/\text{м}\cdot\text{К}$
Тепловое сопротивление:
\begin{equation*}
K_\infty = (168 \pm 5) \text{К}/\text{Вт}
\end{equation*}

И коэффициент аккомодации:
\begin{equation*}
s = 0.60 \pm 0.04
\end{equation*}

Также проверена теория о том, что при высоком давлении теплопередача перестает от него зависеть.


\end{document}

