 \documentclass[a4paper,12pt]{article}
\usepackage[a4paper,top=1.3cm,bottom=2cm,left=1.5cm,right=1.5cm,marginparwidth=0.75cm]{geometry}
\usepackage{setspace}
\usepackage{cmap}					
\usepackage{mathtext} 				
\usepackage[T2A]{fontenc}			
\usepackage[utf8]{inputenc}			
\usepackage[english,russian]{babel}
\usepackage{multirow}
\usepackage{graphicx}
\usepackage{wrapfig}
\usepackage{tabularx}
\usepackage{float}
\usepackage{longtable}
\usepackage{hyperref}
\hypersetup{colorlinks=true,urlcolor=blue}
\usepackage[rgb]{xcolor}
\usepackage{amsmath,amsfonts,amssymb,amsthm,mathtools} 
\usepackage{icomma} 
\mathtoolsset{showonlyrefs=true}
\usepackage{euscript}
\usepackage{mathrsfs}

\graphicspath{{pictures/}}

\DeclareMathOperator{\sgn}{\mathop{sgn}}
\newcommand*{\hm}[1]{#1\nobreak\discretionary{}
	{\hbox{$\mathsurround=0pt #1$}}{}}


\title{\begin{center}Лабораторная работа №2.2.6\end{center}
	Определение энергии активации по температурной зависимости вязкости жидкости}
\author{Лавыгин Кирилл}
\date{\today}

\begin{document}
	
	\maketitle
	
	\textbf{Аннотация:} В работе при помощи цилиндра с исследуемой жидкостью, термостата, мелких шариков и измерительного оборудования определяется зависимость скорости падения шариков от температуры жидкости и рассчитывается вязкость и энергия активации жидкости..
	
	\section{Теоретическая часть}
	\subsection{Энергия активации}
	Для того чтобы перейти в новое состояние, молекула жидкости должна преодолеть участки с большой потенциальной энергией, превышающей среднюю тепловую энергию молекул. Для этого тепловая энергия молекул должна — вследствие флуктуации — увеличиться на некоторую величину $W$ , называемую энергией активации. Температурная зависимость вязкости жидкости при достаточно грубых предположениях можно описать формулой: ($k$ -- постоянная Больцмана, $T$ --  температура системы, $A$ -- некоторая константа)
	\begin{equation} 
		\label{activation_energy:1}
		\eta \sim A e^{W/kT}
	\end{equation}
	
	Из формулы (\ref{activation_energy:1}) следует, что существует линейная зависимость между величинами $ln(\eta)$ и $1/T$, и энергию активации можно найти по формуле:
	
	\begin{equation} 
		\label{activation_energy:2}
		W = k \frac{d(ln(\eta))}{d(1/T)}
	\end{equation}
	
	\subsection{Измерение вязкости}
	По формуле Стокса, если шарик радиусом $r$ и со скоростью $v$ движется в среде с вязкостью $\eta$, и при этом не наблюдается турбулентных явлении, тормозящую силу можно найти по формуле:
	
	\begin{equation}\label{stokes}
		F = 6\pi\eta rv
	\end{equation}
	
	
	Для измерения вязкости жидкости рассмотрим свободное падение шарика в жидкости. При медленных скоростях на шарик действуют силы Архимеда и Стокса, выражения для которых мы знаем. Отсюда находим выражения для установившейся скорости шарика и вязкости жидкости: ($\rho$ -- плотность тела, $\rho_\text{ж})$ -- плотность жидкости)
	
	\begin{equation}
		v_\text{уст} =\frac{2}{9}gr^2\frac{\rho - \rho_\text{ж}}{\eta} \text{ } \text{ } \text{ }
		\eta =\frac{2}{9}gr^2\frac{\rho - \rho_\text{ж}}{v_{уст}} \label{eta}
	\end{equation}
	
	Как видим, измерив установившуюся скорость шарика и параметры системы можно получить вязкость по формуле (\ref{eta}).
	
	\section{Экспериментальная установка}
	Для измерений используется стеклянный цилиндрический сосуд B, наполненный исследуемой жидкостью (глицерин). Диаметр сосуда $\approx 3$ см, длина $\approx 25$ см. На стенках сосуда нанесены две метки на некотором расстоянии друг от друга. Верхняя метка должна располагаться ниже уровня жидкости с таким расчетом, чтобы скорость шарика к моменту прохождения этой метки успевала установиться. Измеряя расстояние между метками и время падения определяем установившуюся скорость шарика $v_\text{уст}$. Сам сосуд B помещен в рубашку D, омываемую водой из термостата. При работающем термостате температура воды в рубашке D, а потому и температура жидкости 12 равна температуре воды в термостате.
	Схема прибора (в разрезе) показана на рис.~\ref{ustanovka}.
	\begin{figure}[H]
		\center{\includegraphics[scale=0.5]{ustanovka}}
		\caption{Установка для определения коэффициента вязкости жидкости.}
		\label{ustanovka}
	\end{figure}

	\section{Ход работы}
	\subsection{Подготовительные работы}
	Для каждой серии измерений выбираем по 4 стальных и стеклянных шарика и измеряем их диаметр микроскопом в двух различных направлениях ($d_1$ и  $d_2$) (так как некоторые сильно отличаются от сферических) Вычисляем среднее значение ($d_m$), которое и будем дальше использовать. В таблице обозначения материалов такие: $s=$стальной, $g$=стеклянный. Результаты измерений - в общей таблице. Погрешности измерении диаметров $\sigma_d = 0.02\text{мм}$. Плотности шариков в эксперименте
	
	\begin{equation}
		\rho_\text{стекло}=2.5\text{г/см}^3, \text{ }\text{ }
		\rho_\text{сталь}=7.8\text{г/см}^3
	\end{equation}
	
	Измеряем длины частей цилиндра установки (см. рис. \ref{ustanovka})
	\begin{equation*}
		l_1=l_2=(10.0\pm0.1)\text{см}
	\end{equation*}
	Для расчета числа Рейнольдса ($Re$) нам также пригодится диаметр нашего сосуда $\approx3$ см
	При больших температурах фиксировалось время пролета двух промежутков для уменьшения погрешности (расстояние указано в таблице как $l$)
	\subsection{Измерение установившихся скоростей}
	Мы знаем путь, который проходит шарик от одной отметки цилиндра к другой. Осталось измерить время прохождения между этими отметками для получения скорости. 
	
	
	\begin{center}
\begin{tabular}{|r|r|r|r|r|r|r|r|r|r|r|r|}
	\hline
	$d_1$, мм & $d_2$, мм & $d_m$, мм & $t$, c &$l$, см & $T,^\circ C$ & М-л & $\eta,\text{ м}^2/c$ & $\sigma_\eta^\text{сист}$ & $\tau\cdot10^@$, с & $S\cdot 10^5$, м &     $Re\cdot10^3$ \\
	\hline
	     2.10 &      2.06 &      2.08 &  34.47 & 10 &        20.19 & g     &                 1006 &                       208 &  0.059 &             0.17 & 0.10 \\
	     \hline
	     2.08 &      2.06 &      2.07 &  35.34 &10 &        20.19 & g     &                 1022 &                       216 &  0.058 &             0.16 & 0.10 \\
	     \hline
	     2.10 &      2.08 &      2.09 &  34.71 & 10&        20.19 & g     &                 1023 &                       213 &  0.059 &             0.17 & 0.10 \\
	     \hline
	     2.06 &      2.08 &      2.07 &  34.34 &10 &        20.19 & g     &                  993 &                       204 &  0.059&             0.17 & 0.11 \\
	     \hline
	     0.64 &      0.54 &      0.59 &  88.91 &10 &        20.19 & s     &                 1102 &                       589 &  0.013 &             0.01 & 0.04 \\
	     \hline
	     0.76 &      0.78 &      0.77 &  55.94 & 10&        20.19 & s     &                 1180 &                       397 &  0.021 &             0.03 & 0.06 \\
	     \hline
	     0.76 &      0.76 &      0.76 &  43.44 &10 &        20.19 & s     &                  893 &                       234 &  0.028 &             0.06 & 0.10 \\
	     \hline
	     0.82 &      0.84 &      0.83 &  39.35 &10 &        20.19 & s     &                  965 &                       229 &  0.030 &             0.07 & 0.10 \\
	     \hline
	     2.08 &      2.10 &      2.09 &  23.06 &10 &           30 & g     &                  682 &                        94 &  0.088 &             0.38 & 0.23 \\
	     \hline
	     2.12 &      2.10 &      2.11 &  22.03 &10 &           30 & g     &                  664 &                        88 &  0.093 &             0.42 & 0.25 \\
	     \hline
	     2.08 &      2.08 &      2.08 &  19.37 &10 &           30 & g     &                  568 &                        66 &  0.105 &             0.54 & 0.34 \\
	     \hline
	     2.02 &      2.12 &      2.07 &  19.13 &10 &           30 & g     &                  555 &                       63. &  0.107 &             0.55 & 0.35 \\
	     \hline
	     0.78 &      0.72 &      0.75 &  27.84 &10 &           30 & s     &                  558 &                        94 &  0.043 &             0.15 & 0.24 \\
	     \hline
	     0.88 &      0.88 &      0.88 &  20.94 &10 &           30 & s     &                  577 &                        73 &  0.058 &             0.27 & 0.31 \\
	     \hline
	     0.78 &      0.78 &      0.78 &  23.16 &10 &           30 & s     &                  502 &                        70 &  0.052 &             0.22 & 0.32 \\
	     \hline
	     0.74 &      0.82 &      0.78 &  25.34 &10 &           30 & s     &                  549 &                        84 &  0.048 &             0.18 & 0.27 \\
	     \hline
	     2.08 &      2.08 &      2.08 &  18.47 & 20 &         41.7 & g     &                  271 &                        15 &  0.221 &             2.39 & 1.49 \\
	     \hline
	     2.06 &      2.12 &      2.09 &  17.91 & 20 &         41.7 & g     &                  266 &                        14 &  0.227 &             2.54 & 1.57 \\
	     \hline
	     2.06 &      2.10 &      2.08 &  17.62 & 20 &         41.7 & g     &                  259 &                        13 &  0.231 &             2.62 & 1.64 \\
	     \hline
	     2.04 &      2.04 &      2.04 &  16.56 & 20 &         41.7 & g     &                  234 &                        11 &  0.246 &             2.97 & 1.93 \\
	     \hline
	     0.80 &      0.80 &      0.80 &  21.78 & 20 &         41.7 & s     &                  248 &                        17 &  0.111 &             1.02 & 1.38 \\
	     \hline
	     0.80 &      0.84 &      0.82 &  20.10 & 20 &         41.7 & s     &                  240 &                        15 &  0.120 &             1.20 & 1.54 \\
	     \hline
	     0.86 &      0.88 &      0.87 &  17.97 & 20 &         41.7 & s     &                  242 &                        14 &  0.135 &             1.50 & 1.72 \\
	     \hline
	     0.84 &      0.90 &      0.87 &  18.41 & 20 &         41.7 & s     &                  248 &                        14 &  0.132 &             1.43 & 1.63 \\
	     \hline
	     2.06 &      2.06 &      2.06 &   9.40 & 20 &         53.5 & g     &                  136 &                         4 &  0.432 &             9.20 & 5.95 \\
	     \hline
	     2.10 &      2.10 &      2.10 &   9.19 & 20 &         53.5 & g     &                  138 &                         4 &  0.442 &             9.62 & 5.83 \\
	     \hline
	     2.10 &      2.10 &      2.10 &   9.13 & 20 &         53.5 & g     &                  137 &                         4 &  0.445 &             9.75 & 5.87 \\
	     \hline
	     2.12 &      2.08 &      2.10 &   8.97 & 20 &         53.5 & g     &                  135 &                         3 &  0.453 &             10.1 & 6.16 \\
	     \hline
	     0.88 &      0.88 &      0.88 &   9.56 & 20 &         53.5 & s     &                  132 &                         4 &  0.254 &             5.31 & 5.91 \\
	     \hline
	     0.72 &      0.78 &      0.75 &  12.12 & 20 &         53.5 & s     &                  121 &                         5 &  0.200 &             3.30 & 5.06 \\
	     \hline
	     0.70 &      0.68 &      0.69 &  12.97 & 20 &         53.5 & s     &                  110 &                         5 &  0.187 &             2.88 & 5.22 \\
	     \hline
	     0.88 &      0.88 &      0.88 &   9.03 & 20 &         53.5 & s     &                  124 &                         4 &  0.268 &             5.95 & 6.62\\
	     \hline
\end{tabular}


	\end{center}
	Для каждого измерения считаем $\eta$, $Re$, $\tau$, $S$ где $v$ это установившаяся, $\tau$ это время релаксации (см. формулу \ref{relaxation_time}), а $S=v\tau$ это путь релаксации.
	\begin{equation}\label{relaxation_time}
		\tau = \frac{2r^2\rho}{9\eta}
	\end{equation}
	
	Плотность жидкости из графика:
	\begin{equation}
		\rho(T=20^\circ C)=1.26\text{ г/см}^3, \text{ }\text{ }
		\rho(T=30^\circ C)=1.255\text{ г/см}^3, \text{ }\text{ }
	\end{equation}
	\begin{equation}
				\rho(T=42^\circ C)=1.25\text{ г/см}^3 \text{ }\text{ }
		\rho(T=54^\circ C)=1.245\text{ г/см}^3 \text{ }\text{ }
	\end{equation}
	
	
	Как видим, времена и пути релаксации очень малые величины, поэтому предположение что установившейся скорость достигается на участках 1 и 2 оправдано. Как видим, числа Рейнольдса в основном меньше 1. Можно предположить что формула Стокса работает, но окончательный вердикт вынесет график зависимости $ln(\eta)(1/T)$.
	
	Проведем анализ погрешностей наших величин (рассчитываем для каждого значения):
	\begin{equation}
		 \sigma_d^\text{сист}=0.02
	\end{equation}
	\begin{equation}
		\sigma_v^\text{сист}=\frac{\sigma_l}{t}+\frac{l\sigma_t}{t^2}\approx6\cdot10^{-4} \text{ м/с} 
	\end{equation}
	\begin{equation}
		\sigma_{\eta}^\text{сист}=\eta\sqrt{4\left(\frac{\sigma_d^\text{сист}}{d} \right)^2 + \left(\frac{\sigma_v^\text{сист}}{v} \right)^2}
	\end{equation}
	Построим искомый график $ln(\eta)(1/T)$, проведя аппроксимацию, используя МНК для всех точек. 
	\begin{figure}[H]
		\center{\includegraphics[scale=1]{output}}

	\end{figure}
	Если считать нашу модель верной, то:
		\begin{equation}
		\frac{W}{k}\approx6040 \text{K}
	\end{equation}
	Используя МНК, оценим случайную погрешность полученной величины:
	\begin{equation}
		\sigma_{W/k}^\text{сл} \approx 120 \text{ К}
	\end{equation}
	Теперь оценим систематическую погрешность (погрешностью измерения температуры мы пренебрегаем, так как она составляет 0.1 К и по относительной величине существенно меньше погрешности $\eta$):
	\begin{equation}
		\sigma_{W/k}^\text{сист}=\frac{\sigma_{\eta}}{\eta}T\approx15 \text{ K}
\end{equation}
\begin{equation}
	\sigma_{W/k}=\sqrt{{\sigma_{W/k}^\text{сист}}^2+{\sigma_{W/k}^\text{сл}}^2}=120 K
\end{equation}
	Как можно видеть, вклад систематической погрешности оказался крайне мал. Итого:
	\begin{equation}
		W\approx (8.24\pm0.16)\cdot 10^{-20} \text{ Дж}
	\end{equation}
	
	
	\section{Обсуждение результатов}
	Наши предположения подтвердились и предложенная модель действительно работает. Это видно и по виду графика, и по тому, что не наблюдается зависимости $\eta$ от параметров шариков. Хоть и точность каждого отдельного измерения была невысока, благодаря большому числу измерений удалось получить значение энергии активации с неплохой точностью (реальное значение составляет примерно $8.4\cdot 10^{-20} \text{ Дж}$ в рассматриваемом диапазоне температур, что попадает в полученный диапазон). Притом погрешность была преимущественно случайной, поэтому точность опыта может быть увеличена, путем увеличения числа измерений.
\end{document}
