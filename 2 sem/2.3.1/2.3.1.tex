 \documentclass[a4paper,12pt]{article}
\usepackage[a4paper,top=1.3cm,bottom=2cm,left=1.5cm,right=1.5cm,marginparwidth=0.75cm]{geometry}
\usepackage{setspace}
\usepackage{cmap}					
\usepackage{mathtext} 				
\usepackage[T2A]{fontenc}			
\usepackage[utf8]{inputenc}			
\usepackage[english,russian]{babel}
\usepackage{multirow}
\usepackage{graphicx}
\usepackage{wrapfig}
\usepackage{tabularx}
\usepackage{float}
\usepackage{longtable}
\usepackage{hyperref}
\hypersetup{colorlinks=true,urlcolor=blue}
\usepackage[rgb]{xcolor}
\usepackage{amsmath,amsfonts,amssymb,amsthm,mathtools} 
\usepackage{icomma} 
\mathtoolsset{showonlyrefs=true}
\usepackage{euscript}
\usepackage{mathrsfs}

\DeclareMathOperator{\sgn}{\mathop{sgn}}
\newcommand*{\hm}[1]{#1\nobreak\discretionary{}
	{\hbox{$\mathsurround=0pt #1$}}{}}


\title{\textbf{Получение и измерение вакуума (2.3.5)}}
\author{Лавыгин Кирилл}
\date{07.03.23}


\begin{document}
	\maketitle
	
	\section{Введение}
	\noindent\textbf{Аннотация:} В ходе работы на предоставленной установке при помощи манометров различного вида будут определены объемы различных частей установки, а также скорость откачки системы (двумя различными способами).
	
	\section{Экспериментальна установка}
	
	\begin{figure}[h!]
		\centering
		\includegraphics[scale=0.6]{facility.jpg}
		\caption{Схема экспериментальной установки}
		\label{facility}
	\end{figure}
	
	Установка изготовлена из стекла,
	и состоит из форвакуумного баллона (ФБ), высоковакуумного диффузионного насоса (ВН), высоковакуумного баллона (ВБ), масляного (М) и ионизационного (И) манометров, термопарных манометров ($\text{М}_1$ и $\text{М}_2$), форвакуумного насоса (ФН) и соединительных кранов ($\text{K}_1, \text{K}_2,\; \ldots \;\text{K}_6$) (Рис. \ref{facility}). Кроме того, в состав установки входят: реостат и амперметр для регулирования тока нагревателя диффузионного насоса.
	
	\begin{wrapfigure}{l}{6cm}
		\centering
		\includegraphics[width=1\linewidth]{pump}
		\caption{Схема работы высоковакуумного насоса}\label{pump}
	\end{wrapfigure}
	
	Устройство масляного диффузионного насоса схематически показано на Рис. \ref{pump} (в лабораторной установке используется несколько откачивающих ступеней). Масло, налитое в сосуд, подогревается электрической печкой. Пары масла поднимаются по трубе и вырываются из сопла. Струя паров увлекает молекулы газа, которые поступают из откачиваемого сосуда через трубку. Дальше смесь попадает в вертикальную трубу. Здесь масло осаждается на стенках трубы и маслосборников после чего стекает вниз, а оставшийся газ откачивается форвакуумным насосом. 
	
	\section{Теоретические сведения}
	\subsection{Процесс откачки}
	
	Опишем процесс откачки математически: 
	Пусть W --- объем газа, удаляемого из сосуда при данном давлении за единицу времени, $Q_i$ для различных значений $i$ обозначим различные притоки газа в сосуд (в единицах $PV$), такие как течи извне $Q_\text{и}$, десорбция с поверхностей внутри сосуда $Q_\text{д}$, обратный ток через насос $Q_\text{н}$. Тогда имеем:
	\begin{equation}
		-VdP = \left(PW - \sum Q_i\right)dt
	\end{equation}
	При достижении предельного вакуума устанавливается $P_{\text{пр}}$, и $dP = 0$. В таком случае:
	\begin{equation}
		W = \biggl( \sum Q_i \biggr)\bigg/ P_{\text{пр}}
	\end{equation}
	Поскольку обычно $Q_\text{и}$ постоянно, а $Q_\text{н}$ и $Q_\text{д}$ слабо зависят от времени, также считая постоянной W, можем проинтегрировать (1) и получить:
	\begin{equation}
		P - P_{\text{пр}} = (P_0 - P_{\text{пр}})\exp\left(-\frac{W}{V}t\right)
		\label{exp}
	\end{equation}
	Полная скорость откачки $W$, собственная скорость откачки насоса $W_{\text{н}}$ и проводимости элементов системы $C_1, C_2,\;\ldots$ соотносятся согласно формуле (4), и это учтено в конструкции установки.
	\begin{equation}
		\frac{1}{W} = \frac{1}{W_\text{н}} + \frac{1}{C_1} + \frac{1}{C_2} + \ldots
	\end{equation}
	
	\subsection{Течение газа через трубу}
	
	Характер течения газа существенно зависит от соотношения между размерами системы и длиной свободного пробега молекул. При атмосферном и форвакуумном давлениях  длина свободного пробега меньше диаметра трубок, и течение газа определяется его вязкостью, т.е. взаимодействием молекул. При переходе к высокому вакууму столкновения молекул между собой начинают играть меньшую роль, чем соударения со стенками.
	
	Для количества газа, протекающего через трубу длины $l$ и радиуса $r$ в условиях высокого вакуума, справедлива формула:
	\begin{equation}
		\frac{d(PV)}{dt} = \frac{4}{3}r^3\sqrt{\frac{2\pi RT}{\mu}}\cdot\frac{P_2 - P_1}{l}
	\end{equation}
	Если труба соединяет установку с насосом, то давлением $P_1$ у его конца можно пренебречь. Давление в сосуде $P = P_2$. Тогда пропускная способность трубы:
	\begin{equation}
		C_\text{тр} = \left(\frac{dV}{dt}\right)_\text{тр} = \frac{4r^3}{3l}\sqrt{\frac{2\pi RT}{\mu}}
		\label{ty}
	\end{equation}

	\section{Ход работы}
	
	\subsection{Определение объемов форвакуумной и высоковакуумной частей установки}
	
	\begin{enumerate}
		
		\item Перед началом работы проверим, что все краны приведены в правильное положение. 
		
		\item Запустим воздух в систему (для этого нужно открыть кран $K_2$  и подождать пару минут пока воздух заполнит установку). 
		
		\item Запустим форвакуумный насос, чтобы он откачал воздух из установки. 
		
		Пронаблюдаем за тем, как давление в установке уменьшается и продолжим откачку до момента, пока давление не будет порядка $ 10^{-2}~ \text{торр}$.
		
		\item Отсоединим установку от форвакуумного насоса, а затем объем, заключенный в кранах и капиллярах форвакуумной части, откроем на всю форвакуумную часть. Тогда давление изменится
		
		\item Запишем показания масляного манометра, а именно высоту масла в обоих коленах: 
		\begin{align}
			h_1 = (37.8 \pm 0.1) ~\text{см} (\text{в правом колене}), && h_2 = (11.1 \pm 0.1) ~\text{см}(\text{в левом колене}),
		\end{align}
		\begin{align}
			\sigma_{\Delta h} = \sqrt{\sigma_{h1}^2 + \sigma_{h2}^2}=0.16 \text{см}  & &
			\Delta h_\text{фв} = h_1-h_2 (26.7 \pm 0.16) ~\text{см}.
		\end{align}
		
		\item Зная объем запертой части установки (вместе с кранами) $V_\text{кап} = 50 ~ \text{см}^3$, $P_\text{атм} = 735.9$ мм.рт.ст., $P_0=2.5\cdot10^{-2}$ мм.рт.ст. (начальное давление в установке), $\varrho=0.885 \text{г/см}^3$ (плотность масла), вычислим объем форвакуумной части установки.
		
		\begin{align}
			V_\text{фв} = \frac{P_\text{атм}-P_0}{\rho g \Delta h_\text{фв}} V_\text{кап} \approx 2117.7~\text{мл},&&
			\varepsilon_{V_\text{фв}} = \varepsilon_{\Delta h} \approx 0.6~\%.
		\end{align}
		
		\begin{equation}
			V_\text{фв} = (2.117 \pm 0.013)~ \text{л}
		\end{equation}
		
		
		\item Проведем те же самые измерения с высоковакуумной части и получим объем установки, из которой вычитанием объема форвакуумной части получается объем высоковакуумной части. 
		\begin{align}
			h_3 = (33.3 \pm 0.1) ~\text{см}, && h_4 = (16.2 \pm 0.1) ~\text{см},
		\end{align}
		\begin{align}
			\Delta h_\text{полн} = (17.1 \pm 0.16) ~\text{см}.
		\end{align}
		
		Погрешности высот определяются аналогично предыдущему пункту. Как и формула для полного объема установки, тогда:
		\begin{align}
			V_\text{полн} = \frac{P_\text{А}P_\text{атм}-P_0}{\rho g \Delta h_\text{полн}} V_\text{кап} \approx 3307~\text{мл},&&
			\varepsilon_{V_\text{полн}} = \varepsilon_{\Delta h} \approx 0.6~\%.
		\end{align}
	
	\begin{equation}
		V_\text{полн} = (3.310 \pm 0.02)~ \text{л}
	\end{equation}
		
		В результате искомая величина равна:
		\begin{align}
			V_\text{вв} = V_\text{полн} - V_\text{фв} = 1.193~ \text{л}, && \sigma_{V_\text{вв}} = \sqrt{\sigma_{V_\text{полн}}^2+ \sigma_{V_\text{фв}}^2} \approx 0.003 ~ \text{л},
		\end{align}
		\begin{align}
			V_\text{вв} = (1.193 \pm 0.003)~\text{л}.
		\end{align}
		
	\end{enumerate}
	
	\subsection{Получение высокого вакуума и измерение скорости откачки}
	
	\begin{enumerate}
		\setcounter{enumi}{7}
		
		\item Не выключая форвакуумного насоса убедимся в том, что в установке не осталось запертых объемов. 
		
		\item Откачав установку до давления порядка $ 10^{-4}~ \text{торр}$, приступим к откачке ВБ с помощью диффузионного насоса. 
		
		\item С помощью ионизационного манометра измерим значение предельного давления в системе со стороны высоковакуумной части: $$P_\text{пр} = (4,7 \pm 0.1)  \cdot 10^{-5} ~\text{торр}.$$
		
		
		\item Найдем скорость откачки по ухудшению и улучшению вакуума, для этого открывая и закрывая кран $K_3$ будем то подключать насос к объему, то отключать его, при этом на видео зафиксируем показания манометра от времени и построим графики необходимых  зависимостей (каких именно подробнее описано в соответствующих пунктах ниже), для которых определим коэффициенты наклона прямых и их погрешности (с помощью МНК).
		\begin{align}
			\sigma_{ln(P-P_\text{пр})}=2 \frac{\sigma_P}{P-P_\text{пр}}, && \sigma_t=0.5~c \text{ (минимальное время обновления прибора)}  
		\end{align}
		\begin{equation}
			\sigma_k^\text{сист} \approx -2k 
			\frac{\sigma_P}{(P-P_\text{пр})ln(P-P_\text{пр})} =0.006 ~ 1/c
		\end{equation}
		\begin{align}
			\varepsilon_W=\sqrt{\varepsilon_V^2+\varepsilon_k^2}=0.01 \text{ л/с}, && 
			\varepsilon_k=\frac{k_1-k_2}{k_\text{ср}} =0.2
		\end{align}
		Для случая улучшения вакуума воспользуемся формулой \eqref{exp} и построим график зависимости $(ln((P-P_\text{пр})/P_1))$ от $t$. По виду графику видно, что наша модель работает хорошо только для $t<15$с, поэтому при расчёте коэффициентов использовались только они. (первые 3 точки второго опыта так же не учитывались) При построении такого графика из МНК получим коэффициент наклона --- $k$, с помощью которого можно найти $W = -kV_{вв}$.
		\begin{figure}[H]
			\includegraphics[scale=1]{graph_bet}
			\caption{Улучшение вакуума 1}
			\label{graph2}
		\end{figure}
		\bgroup
		\def\arraystretch{1.3}%
		\begin{table}[H]
			\begin{center}
				\begin{tabular}{|c|c|c|c|c|c|}
					\hline
					$k,~\frac{1}{\text{с}}$ & $\sigma_k^\text{сл},~\frac{1}{\text{с}}$ &$\sigma_k,~\frac{1}{\text{с}}$& $k_{\text{ср}},~\frac{1}{\text{с}}$ & $W,~\frac{\text{л}}{\text{с}}$ & $\sigma_W,~\frac{\text{л}}{\text{с}}$\\
					\hline
					-0.171& 0.005 & 0.02 & \multirow{2}{*}{-0.19}& \multirow{2}{*}{0.23}& \multirow{2}{*}{0.02}\\
					\cline{1-3}
					-0.212&0.006&0.04&&&\\
					\hline
				\end{tabular}
			\end{center}
			\caption{Коэффициенты наклона при улучшении вакуума}
			\label{tab1}
		\end{table}
		Как можно видеть, основная погрешность кроется именно в случайности начальных условий опыта, так как при небольшой погрешности в рамках одного опыта, различные опыты дают достаточно отличающиеся значения. Так-же предложенная модель насоса работает лишь при достаточно большой разнице концентраций, что хорошо видно из графика
		\egroup
		\item Оценим величину потока газа  $Q_\text{Н}$. Для этого воспользуемся данными, полученными при ухудшении вакуума. А именно построим графики зависимости $P(t)$ и определим для них коэффициенты угла наклона прямой.
		\begin{equation}
			\sigma_k^\text{сист}\approx k\sigma_p/p=0.1 \cdot 10^{-6},~\frac{\text{торр}}{\text{с}}
		\end{equation}
		Поскольку $V_{\text{вв}}dP = (Q_\text{Д} + Q_\text{И}) dt$ получим $(Q_\text{Д} + Q_\text{И}) = kV_\text{вв}$. По графикам получаем:
		\bgroup
		\def\arraystretch{1.3}%
				\begin{figure}[H]
			\includegraphics[scale=1]{graph_wor}
			\caption{Ухудшение вакуума 1}
			\label{graphup1}
		\end{figure}
		\begin{table}[h!]
			\begin{center}
				\begin{tabular}{|c|c|c|c|c|c|}
					\hline
					$k\cdot 10^{6},~\frac{\text{торр}}{\text{с}}$ & $\sigma_k^\text{сл}\cdot 10^{6},~\frac{\text{торр}}{\text{с}}$ &$\sigma_k\cdot^{6},~\frac{\text{торр}}{\text{с}}$& $k_{\text{ср}}\cdot10^{6},~\frac{\text{торр}}{\text{с}}$ & $Q_\text{д}+Q_\text{и},~\text{торр}\cdot\frac{\text{л}}{\text{с}}$ & $\sigma_{Q_\text{д}+Q_\text{и}},~\text{торр}\cdot\frac{\text{л}}{\text{с}}$\\
					\hline
					5.2& 0.04 & 0.35 & \multirow{2}{*}{9.8}& \multirow{2}{*}{$5.55\cdot10^{-6}$}& \multirow{2}{*}{$0.4\cdot10^{-6}$}\\
					\cline{1-3}
					5.9&0.02&0.35&&&\\
					\hline
				\end{tabular}
			\end{center}
			\caption{Коэффициенты наклона при ухудшении вакуума}
			\label{tab2}
		\end{table}
		\egroup

		Используя формулу $Q_\text{Н} = P_\text{пр}W - (Q_\text{Д} + Q_\text{И})$, а значит $\sigma_{Q_\text{Н}} =  \sqrt{\sigma_{P_\text{пр}W}^2 + \sigma_{Q_\text{Д} + Q_\text{И}}^2} \approx 10^{-6}~\text{торр}\cdot\frac{\text{л}}{\text{с}} $, получим, что: $Q_\text{Н} = (5.26 \pm 1) \cdot 10^{-6} ~ \text{торр} \cdot \text{л} /$ с.
		
		
		
		\item Оценим пропускную способность трубки по формуле \eqref{ty}:
		\begin{align}
			L = (10.8 \pm 0.1)~ \text{см}; &&   d = (0.8 \pm 0.1) ~ \text{см},
		\end{align}
		
		\begin{equation}
			C_{\text{тр}} = (0.58 \pm 0.2)~\text{л} / \text{с}.
		\end{equation}
		
		Погрешность $C_{\text{тр}}$  оценена как корень из суммы квадратов погрешностей длинны и диаметра (которые явным образом не указаны на установке, оценка довольно грубая).
	
		
		\item Введем в систему искусственную течь и запишем значение  установившегося при этом давления и давления $P_{\text{фв}}$: 
		
		\begin{align}
			P_{\text{уст}} = (6.7 \pm 0.1) \cdot 10^{-5} ~ \text{торр}. && P_{\text{фв}} = (1 \pm 0.1) \cdot 10^{-4} ~ \text{торр}.
		\end{align}
		
		
		\item Поскольку
		$$P_{\text{пр}} W = Q_1, \quad P_{\text{уст}} W = Q_1 + \frac{d(PV)_\text{кап}}{dt},$$
		то с учетом \eqref{ty}, получаем:
		\begin{equation}
			W = \frac{P_\text{фв}}{P_\text{уст}-P_\text{пр}}\frac{4r^3}{3L}\sqrt{\frac{2\pi RT}{\mu}} \approx 0.21~\frac{\text{л}}{\text{с}}
		\end{equation}
	\begin{align}
		\sigma_W\approx3W\frac{\sigma_d}{d}=0.06 ~\frac{\text{л}}{\text{с}}, && W=(0.21 \pm 0.06)~\frac{\text{л}}{\text{с}}
	\end{align}
		
		(Поскольку погрешность $d$ была оценена очень грубо, то и погрешность итогового значения можно взять такой же по относительной величине)
		
		Полученное значение удивительно хорошо совпадает с предложенным ранее значением, что говорит о том, что предложенная модель течи работает.
		\item Следуя указаниям выключаем установку.
		
	\end{enumerate}
	
	\section{Вывод}
		В ходе работы были проверены предложенные в теоретической части зависимости. Все из них были подтверждены, но, как было описано в ходе работы, предложенная модель диффузионного насоса работает не при всех давлениях
		
		
		

	
	\newpage
	
	\section{Приложение}

$$\begin{tabular}{||c||c|c||c|c||c|c||c|c||}
	\hline\hline
	&  & Ухудшение &  & Откачка & & Откачка & & Ухудшение \\ \hline
	\hline\hline
	№ & $t,  \text{c}$ & $P, \text{торр} \cdot 10^{-5}$ & $t,  \text{c}$ & $P, \text{торр} \cdot 10^{-5}$ & $t,  \text{c}$ & $P, \text{торр} \cdot 10^{-5}$ & $t,  \text{c}$ & $P, \text{торр} \cdot 10^{-4}$\\ \hline \hline
	1 & 4 & 5.1 & 3 & 61 & 0&17 & 5 & 1.6\\ \hline
	2 & 5 & 5.8 & 4 & 58 & 2& 13& 8 & 1.8\\ \hline
	3 & 6 & 6.3 & 5 & 55 & 4& 10& 12 & 2.0\\ \hline
	4 & 7 & 6.8 & 7 & 41 & 6& 8.0& 16 & 2.3\\ \hline
	5 & 8 & 7.3 & 8 & 31 & 8& 7.2& 22 & 2.6\\ \hline
	6 & 9 & 7.7 & 9 & 26 & 10& 6.3& 26 & 2.9\\ \hline
	7 & 10 & 8.2 & 10 & 22 & 12& 5.9& 32 & 3.2\\ \hline
	8 & 11 & 8.9 & 11 & 18 & 14& 5.5& 37 & 3.5\\ \hline
	9 & 12 & 9.2 & 12 & 16 & 15& 5.4& 41 & 3.8\\ \hline
	10 & 13 & 10 & 13 & 14 & 17& 5.2& 46 & 4.1\\ \hline
	11 & 14 & 11 & 14 & 12 & 21& 5.0& 51 & 4.4\\ \hline
	12 & 20 & 13 & 15 & 11 & 29& 4.8 & 56 & 4.7\\ \hline
	13 & 30 & 18 & 16 & 9.4 & 48& 4.6& 61 & 5.0\\ \hline
	14 & 40 & 23 & 17 & 8.6 & 1& 15& 67 & 5.3\\ \hline
	15 & 50 & 28 & 18 & 7.9 & 5& 9.2& 73 & 5.6\\ \hline
	16 & 60 & 33 & 19 & 7.3 & 6& 8.4& 77 & 5.9\\ \hline
	17 & 70 & 39 & 20 & 6.8 & 7& 7.7& 83 & 6.2\\ \hline
	18 & 80 & 44 & 21 & 6.3 & 9& 6.8& 88 & 6.5\\ \hline
	19 & 90 & 50 & 22 & 6.0 & 13& 5.7& 93 & 6.8\\ \hline
	20 & 100 & 56 & 23 & 5.8 & & & 99 & 7.1\\ \hline
	21 &     &    & 25 & 5.5 & & & 103 & 7.4\\ \hline
	22 &     &    & 30 & 5.1 & & & & \\ \hline
	23 &     &    & 35 & 4.8 & & & & \\ \hline
	24 &     &    & 40 & 4.7 & & & & \\ \hline
\end{tabular}$$
	
	
\end{document}
