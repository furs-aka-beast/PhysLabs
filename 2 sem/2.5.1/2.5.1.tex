 \documentclass[a4paper,12pt]{article}
\usepackage[a4paper,top=1.3cm,bottom=2cm,left=1.5cm,right=1.5cm,marginparwidth=0.75cm]{geometry}
\usepackage{setspace}
\usepackage{cmap}		
\usepackage{amsmath,amsfonts,amssymb,amsthm,mathtools} 			

\usepackage[T2A]{fontenc}			
\usepackage[utf8]{inputenc}			
\usepackage[english,russian]{babel}
\usepackage{multirow}
\usepackage{graphicx}
\usepackage{wrapfig}
\usepackage{tabularx}
\usepackage{float}
\usepackage{longtable}
\usepackage{hyperref}
\hypersetup{colorlinks=true,urlcolor=blue}
\usepackage[rgb]{xcolor}
\usepackage{icomma} 
\usepackage{euscript}


\DeclareMathOperator{\sgn}{\mathop{sgn}}
\newcommand*{\hm}[1]{#1\nobreak\discretionary{}
	{\hbox{$\mathsurround=0pt #1$}}{}}


\title{\textbf{Измерение коэффициента поверхностного натяжения жидкости (2.5.1)}}
\author{Лавыгин Кирилл}
\date{\today}


\begin{document}
	
	\maketitle
	\section{Аннотация}
	В данной работе при помощи прибора Ребиндера, термостата, манометра и микроскопа мы найдем коэффициент поверхностного натяжения дистиллированной воды, зная коэффициент для спирта, определим полную поверхностную энергию.
	\section{Теоретические сведения}
	
	Наличие поверхностного слоя приводит к различию давлений по разные стороны от искривленной границы раздела двух сред.  Для сферического пузырька с воздухом  внутри жидкости избыточное давление дается формулой Лапласа:
	
	\begin{equation}
		\Delta P = P_{int} - P_{ext} = \frac{2\sigma}{r},
		\label{key}
	\end{equation}
	где $ \sigma $ -- коэффициент поверхностного натяжения, $ P_{int} $ и $ P_{ext} $ -- давление внутри пузырька и снаружи, $ r $ -- радиус кривизны поверхности раздела двух фаз. Эта формула лежит в основе предлагаемого метода определения коэффициента поверхностного натяжения жидкости. Измеряется давление $ \Delta P $, необходимое для выталкивания в жидкость пузырька воздуха.
	
	\section{Экспериментальная установка}
	
	\begin{wrapfigure}{R}{10cm}
		%\vspace{-61pt}
		\begin{center}
			\includegraphics[width=10cm]{fac}
			\caption{Рисунок экспериментальной установки}
			\label{img:ust}
		\end{center}
	\end{wrapfigure}
	
	Исследуемая жидкость (дистиллированная вода) наливается в сосуд (колбу) $ B $ (рис. \eqref{img:ust}). Тестовая жидкость  (этиловый спирт) наливается  в сосуд $ E $.  При измерениях  колбы герметично закрываются  пробками. Через одну из двух пробок  проходит полая металлическая игла $ С $. Этой пробкой закрывается сосуд, в котором  проводятся измерения. Верхний конец иглы открыт в атмосферу, а нижний погружен в жидкость. Другой сосуд герметично закрывается второй пробкой. При создании достаточного  разряжения воздуха в колбе с иглой пузырьки воздуха начинают пробулькивать через жидкость. Поверхностное натяжение можно определить по величине разряжения $ \Delta P $ \eqref{key}, необходимого для прохождения пузырьков (при известном радиусе иглы).
	
	Разряжение в системе создается с помощью аспиратора $ A $. Кран $ K_2 $ разделяет две полости аспиратора. Верхняя полость при закрытом кране $ K_2 $ заполняется водой. Затем кран $ K_2 $ открывают и заполняют водой  нижнюю полость  аспиратора.  Разряжение воздуха создается в нижней полости  при открывании крана $ K_1 $, когда  вода вытекает из неё по каплям. В колбах $ B $ и $ C $, соединённых трубками с нижней полостью аспиратора, создается такое же пониженное давление. Разность давлений в полостях с разряженным воздухом и атмосферой измеряется спиртовым микроманометром М. 
	
	Для стабилизации температуры исследуемой жидкости через рубашку $ D $ колбы $ В $ непрерывно прогоняется вода из термостата.
	
	Обычно кончик иглы лишь касается поверхности жидкости, чтобы исключить влияние гидростатического давления столба жидкости. Однако при измерении температурной зависимости коэффициента поверхностного натяжения возникает ряд сложностей. Во-первых, большая теплопроводность металлической трубки приводит к тому, что температура на конце трубки заметно ниже, чем в глубине жидкости. Во-вторых, тепловое расширение поднимает уровень жидкости при увеличении температуры.
	
	Обе погрешности можно устранить, погрузив кончик трубки до самого дна. Полное давление, измеренное при этом микроманометром, равно \[ P = \Delta P + \rho g h.\] Заметим, что $ \rho gh $ от температуры практически не зависит, так как подъём уровня жидкости компенсируется уменьшением её плотности (произведение $ \rho g $ определяется массой всей жидкости и поэтому постоянно). Величину  $ \rho g h $ следует измерить двумя способами.
	
	Во-первых, замерить величину $ P_1= \Delta P' $, когда кончик трубки только касается поверхности жидкости. Затем при этой же температуре опустить иглу до дна и замерить $ P_2= \rho gh + \Delta P'' $ ($ \Delta P' $, $ \Delta P'' $ -- давление Лапласа). Из-за  несжимаемости  жидкости можно положить $ \Delta P' = \Delta P'' $ и тогда \[ \rho gh= P_2 - P_1. \]
	
	Во-вторых, при измерениях $ P_1 $ и $ P_2 $ замерить линейкой  глубину погружения иглы $ h $. Это можно сделать, замеряя расстояние между верхним концом иглы и любой неподвижной частью прибора при положении иглы на поверхности и в глубине колбы.
	
	\section{Ход работы}
	Плотность используемого в барометре спирта $\varrho_\text{м}\approx 785 \text{ кг}/\text{м}^3$ Погрешности измеряемых величин: 
	\begin{align*}
		\sigma_p=0.2\text{ мм.сп.ст.} && \sigma_l=1 \text{ мм} && \sigma_T=0.1 ~ ^\circ C
	\end{align*}
	\subsection{Измерение диаметра иглы}
	Измерим максимальное давление при пробулькивании пузырьков воздуха через спирт ($P'$)):
	\begin{table}[H]
		\begin{center}
			\begin{tabular}{|c|c|c|c|c|}
				\hline
				$P',$ мм.сп.ст &8.6&8.6&8.8&8.6\\
				\hline
				$P',$ дел. &68&68&70&68\\
				\hline
			\end{tabular}
		\end{center}
		\caption{Результаты измерений в спирте}
		\label{tab1}
	\end{table}
	\begin{align*}
		P'=69 \text{Па} && \sigma_{P'}^\text{случ}=0.5 \text{Па} && \sigma_{P'}=2 Па
	\end{align*}
	
	По формуле \eqref{key} найдем диаметр иглы:
	\begin{equation*}
		d = \frac{4\sigma_{\text{с}}}{P_\text{макс}} = (1.34\pm 0.04)\text{ мм}.
	\end{equation*}
	
	Результат полученный под микроскопом: $D = (1,00\pm0.05)$ мм, как можно видеть, имеет место достаточно существенная ошибка. Она может крыться как в неверном указанном значении плотности спирта в манометра, так и в неверном значении коэффициента поверхностного натяжения. В любом случае это плохо скажется на результатах нашего эксперимента. В дальнейшем будем использовать показание полученное на микроскопе, так как в его точности мы уверены.
	
	\subsection{Измерение температурной зависимости коэффициента поверхностного натяжения}
	
	Проведем одно измерение с иглой у поверхности, далее измерения будем проводить с погруженной до дна иглой, чтобы устранить погрешности, вызванные расширением воды при нагреве и теплопроводностью.
	
	Глубина погружения измеренная линейкой: $\Delta h = (1.2\pm0.1)$ см. Глубина погружения по разнице давлений после опускания (по 3 измерениям) (комнатная температура: $T_\text{к}=25.2 ^\circ C$): 
	\begin{align*}
	\Delta P = (37.0-24.7)*765*9.81/10^3 = 92 \text{ Па}, && \sigma_{\Delta P}=4 \text{ Па} && \Delta h = \dfrac{\Delta P}{\rho g} = 9.4 \text{мм} && \sigma_{\Delta h}=0.4 \text{мм}  
	\end{align*}
	
	Далее перейдем к основной серии измерений с погруженной иглой (все измерения проводились 3 раза, но получилось так, что эти значения совпали)
	
	Рассчитывать коэффициент поверхностного натяжения будем по формуле:
	\begin{equation*}
		\sigma = \frac{\Delta P d}{4}.
	\end{equation*}
	Ввиду погруженности иглы $\Delta P = P - P_0, ~ P_0=109 \pm 9$ Па.
	
	Получаем таблицы:
	\begin{table}[h]
		\centering
		\begin{tabular}{|c|c|c|c|c|c|c|c|c|}
			\hline
			$T,\: ^\circ$С & 25.0 & 30.0 & 35.0 & 40.0 & 45.0 & 50.0 & 55.0 & 60.0 \\
			\cline{1-9}
			$P,$ мм.сп.ст. & 37.0 & 36.8 & 36.4 & 36.2 & 36.0 & 35.6 & 35.4 & 35.2 \\
			\cline{1-9}
			$\Delta P,$ Па & 168 & 167 & 164 & 162 & 161 & 158 & 156 & 155 \\
			\cline{1-9}
			$\sigma,\: \frac{\text{мН}}{\text{м}}$ & 42 & 41 & 41 & 41 & 40 & 40 & 39 & 39 \\
			\cline{1-9}
		\end{tabular}
		
	\end{table}
	
	\begin{align*}
		\sigma_{\Delta P}=\sqrt{\sigma_P^2+\sigma_{P_0}^2}=9 ~\text{Па}, && \sigma_\sigma=\sigma \varepsilon_{\Delta P}=2 ~ \frac{\text{мН}}{\text{м}}
	\end{align*}
	
	Построим график зависимости $\sigma(T)$:

	
\begin{figure}[H]
	\centering
	\includegraphics[width=1\linewidth]{main}
\end{figure}
	Температурный коэффициент :
	\begin{align*}
		k=\left(\dfrac{d\sigma}{dT}\right)=0.099 ~ \frac{\text{мН}}{\text{м}\cdot \text{К}}, && \sigma_k^\text{случ}=0.003 ~ \frac{\text{мН}}{\text{м}\cdot \text{К}}, && \sigma_k^\text{сист}\approx k \varepsilon_\sigma=0.005 ~ \frac{\text{мН}}{\text{м}\cdot \text{К}}, && \sigma_k=0.006 ~ \frac{\text{мН}}{\text{м}\cdot \text{К}}
	\end{align*}
	
	\subsection{Графики других величин}
	
	Окончательно, с помощью полученных данных построим графики теплоты образования единицы поверхности жидкости: $q = - T\cdot\dfrac{d\sigma}{dT}$ и поверхностной энергии $U$ единицы площади $F$: $\dfrac{U}{F} = \left(\sigma - T\cdot\dfrac{d\sigma}{dT}\right)$.
	
\begin{figure} [H]
	\centering
	\includegraphics{q}
	\caption{График $q$}
\end{figure}
	
\begin{figure} [H]
	\centering
	\includegraphics{u}
	\caption{График $\dfrac{U}{F}$}
\end{figure}
	\section{Вывод}
	В ходе работы:
	\begin{enumerate}
		\item Был экспериментально измерен диаметр иглы при помощи коэффициента поверхностного натяжения спирта и при помощи микроскопа. Полученные результаты сильно расходились, что обсуждалось в тай части работы. В итоге: $d = (1.0\pm 0.05)\text{ мм}$ .
		\item Различными способами была измерена глубина погружения иглы.
		\item Получены коэффициенты поверхностного натяжения воды при различных ее температурах.
		\item Получено значение $\left(\dfrac{d\sigma}{dT}\right)=0.099 \pm 0.006 ~ \frac{\text{мН}}{\text{м}\cdot \text{К}}$
		\item Построены требуемые дополнительные графики, из которых видно, например, что поверхностная энергия $\dfrac{U}{F}=44.65 ~ \frac{\text{мН}}{\text{м}\cdot \text{К}}$.
	\end{enumerate}
	
	
	
	
	
	
	
	
\end{document}
