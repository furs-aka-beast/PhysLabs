\documentclass[16pt]{article}

\usepackage{graphicx}

\usepackage[fontsize=16pt]{fontsize}
\usepackage[T2A]{fontenc}
\usepackage[utf8]{inputenc}
\usepackage[russian]{babel}
\usepackage{amsmath}
\usepackage{mathtools}
\usepackage{biblatex}
\usepackage{csquotes}
\usepackage{geometry}
\usepackage{float}
\usepackage{authblk}
\usepackage{physics}

\usepackage[neverdecrease]{paralist}

\geometry{legalpaper, margin = 0.5in}

\title{Изучение явлений физической адсорбции молекул воздуха в ионизационной лампе. Теория}
\author[*]{Лунев Дмитрий}
\affil[*]{Московский физико-технический институт}

\begin{document}
	
	\maketitle
	
	\section{Основы}
		
		Адсорбция - явления накопления вещества на поверхности другого. Обратный процесс называется диссорбцией. Является активационным процессом. Теплота этого процесса называется теплотой адсорбции.
		
		Адсорбция бывает химической и физической. При химической адсорбции будут проявляться валентные (химические) силы, при физической - Ван-дер-Ваальсовы силы (электромагнитные).
	
	
\end{document}