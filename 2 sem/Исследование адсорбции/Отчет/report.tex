 \documentclass[a4paper,12pt]{article}
\usepackage[a4paper,top=1.3cm,bottom=2cm,left=1.5cm,right=1.5cm,marginparwidth=0.75cm]{geometry}
\usepackage{setspace}
\usepackage{cmap}		
\usepackage{amsmath,amsfonts,amssymb,amsthm,mathtools} 			

\usepackage[T2A]{fontenc}			
\usepackage[utf8]{inputenc}			
\usepackage[english,russian]{babel}
\usepackage{multirow}
\usepackage{graphicx}
\usepackage{wrapfig}
\usepackage{tabularx}
\usepackage{float}
\usepackage{longtable}
\usepackage{hyperref}
\hypersetup{colorlinks=true,urlcolor=blue}
\usepackage[rgb]{xcolor}
\usepackage{icomma} 
\usepackage{euscript}



\title{\textbf{Изучение явлений физической адсорбции молекул воздуха в ионизационной лампе}}
\author{Лавыгин Кирилл Дмитриевич Б02-213}
\date{\today}

\begin{document}
	
\maketitle

\section{Аннотация}
В данной работе были исследовано явление адсорбции молекул воздуха (азота) на поверхность коллектора в ионизационной лампе ПМИ-2.

\paragraph{Цели работы}
\begin{itemize}
	\item Познакомиться с явлениями адсорбции-десорбции молекул воздуха, ионно-адсорбционной откачки и теплового баланса в условиях высокого вакуума. 
	\item По результатам измерений оценить энергию физической адсорбции молекул воздуха на металле (коллекторе), 
	\item Проверить использованную модель адсорбции
	\item Определить механизм установления теплового баланса 
\end{itemize}

\section{Принцип работы ПМИ-2}

\begin{wrapfigure}{l}{0.42\textwidth}
	\includegraphics[width=0.42\textwidth]{images/PMI2_ustr}
\end{wrapfigure}

В стеклянной колбе находятся три электрода: катод, анод и коллектор. Катод прямого накала необходим для создания потока электронов путем термоэлектронной эмиссии. Анод - проволока с большим расстоянием между витками, так что значительная часть электронов пролетает сквозь анод. Электроны, сталкиваясь с молекулами воздуха, ионизируют их. Положительные ионы разгоняются и попадают на коллектор, где они мгновенно становятся нейтральными, забирая электроны у коллектора. Получается, что на коллекторе возникает ток, который называется ионным током.

Если ток эмиссии (ток на катоде) удерживать постоянным, то ионный ток будет пропорционален давлению газа в колбе.

$$P = C \cdot I_\text{ион}$$

Паспортное значение $C$ при токе эмиссии $I_\text{э} = 0.5 \text{мА}$ равно $(87 \pm 17) \: \frac{\text{Торр}}{\text{A}}$

Катод и анод сделаны из вольфрама, коллектор - из никеля. Объем колбы $V \approx 85 \text{см}^3$, Площадь коллектора $S \approx 23,6 \text{см}^3$.

Коллектор соединен с колбой теплопроводящими стержнями. Это значит, что при в термодинамическом равновесии можно считать температуры колбы и коллектора постоянными.

\section{Теоретические сведения}

\subsection{Адсорбция}

Адсорбция - накопление одного вещества на поверхности другого. Обратный процесс называют десорбцией. 
При низких давлениях на поверхности адсорбента (твердого тела, коллектор) образуется мономолекулярный слой (слой, толщина которого не превышает размер молекулы).

Адсорбционные силы слагаются из валентных сил взаимодействия (химических) и более слабых ван-дер-ваальсовых (физических). В нашем случае имеет место физическая адсорбция.

При физической адсорбции адсорбционные силы имеют ту же природу, что и межмолекулярное взаимодействие в газах, жидкостях и твёрдых телах. Адсорбированные молекулы газа принимают участие в тепловом (колебательном) движении атомов твёрдого тела и, следовательно, при достаточно большой амплитуде этого движения могут испаряться (десорбироваться) с поверхности твёрдого тела. Минимальным временем адсорбции можно считать период колебания молекулы в потенциальной яме, в которой молекула совершает гармонические колебания. Приближенно это время можно считать независящим от температуры и одинаковым для различных молекул газа: $\tau_0=10^{-13} c$. По порядку величины она равна периоду колебаний молекул в твердом теле. Процесс десорбции является активационным,  т.е. для испарения молекула должна иметь определенную энергию. Эту энергию  называют теплотой адсорбции $Q_a$.

Общее время адсорбции можно оценить по формуле Френкеля:

$$\tau = \tau_0 \exp \left( \frac{Q_a}{RT}\right)$$

\subsection{Модель адсорбции Ленгмюра}

Рассмотрим адсорбцию молекул газа на поверхность твердого тела. Будем предполагать, что адсорбционный слой мономолекулярный, т.е. его толщина не больше толщины одной молекулы, а также молекулы занимают определенные ячейки фиксированной площади. Для азота эта площадь равна $w = 16.2 \cdot 10^{-16}\:\text{см}^2$.

\subsection{Степень заполнения коллектора ионами}

Обозначим отношение занятой площади к всей площади $\theta = \frac{S_\text{зн}}{S}$. Тогда свободная площадь равна $S_\text{св} = (1 - \theta) S$. Оценим максимально возможное число ионов на коллекторе: $N_s = \frac{S}{w}$. При достаточно большой температуре коллектор будет обезгажен, т.е. все $N_s$ молекул будут находится в колбе.

Оценим число частиц на коллекторе при температуре $T_0$. При остаточном давлении $P_0$ в колбе будет $N_0 = V \cdot \frac{P_0}{kT_0}$. Тогда занятую площадь можно оценить так:

$$S_\text{зн} = w(N_s - N_0)$$

%написать про оценку площади (99%)

Также можно оценить степень заполнения при любой температуре. 
\begin{equation}\label{n}
	n(T) = n_0 \cdot \frac{I_\text{ион}(T)}{I_0}
\end{equation}

Найдем отношение занятой площади к свободной c учетом того, что $N \ll N_s$:

$$\Theta = \frac{S_\text{зн}}{S_\text{св}} \approx \frac{N_s}{N(T)} \sim \frac{1}{n(T)}$$

\subsection{Ионизация. Уравнение ионизационного преобразователя}

Уравнение ионизации молекулы $M$:

$$e^- + M \rightarrow M^+ + 2e^-$$

Пусть $\lambda$ - длина свободного пробега электронов. Тогда число ионов, которые ионизирует один электрон, равна $n \sigma \lambda$ где $n$ - концентрация нейтральных молекул, $\sigma$ - сечение ионизации (эффективная площадь сечения нейтральной частицы, необходимая для 100\% вероятностью). Тогда общее число ионов будет равно:

$$N_\text{ион} = n \sigma \lambda_\text{эл} N_\text{эл}$$

Т.к. ионный ток и электронный ток (то же самое, что ток эмиссии) пропорциональны числу частиц получаем:

$$I_\text{ион} = n\sigma \lambda_\text{эл} I_\text{э}$$.

Значит при постоянном токе эмиссии $P \sim I_{ион}$. Коэффициент пропорциональности:

$$C = \frac{\sigma \lambda I_\text{э}}{kT}$$

\subsection{Адсорбционные и десорбционные потоки в лампе}

Адсорбционный поток определяется двумя состовляющими: поток нейтральных молекул и потоком ионов.

\begin{align*}
	J_\text{ад, м} &= \frac{1}{4} S_\text{св} n v_\text{т} \\
	J_\text{ад, ион} &= S_\text{св} n_\text{ион} v_\text{ион}
\end{align*}

где $v_\text{т}$ - тепловая скорость молекул воздуха, $v_\text{ион}$ - скорость ионов.

Концентрацию ионов можно получить, используя то, что они рождаются из-за налетания электрона на молекулу:

$$n_\text{ион} = n \sigma \lambda n_\text{эл}$$

Концентрацию электронов можно оценить через поток электронов на анод, который должен быть равен потоку электронов с катода:

$$J_\text{эл} = S_\text{анод} n_\text{эл} v_\text{эл} = e I_\text{э}$$

где $S_\text{анод}$ - площадь поверхности анода, $e$ - заряд электрона, $I_\text{э}$ - сила тока эмиссии.
Значение $\sigma \lambda$ можно определить через коэффициент $C$, который является параметром лампы, и потому известен
\begin{equation*}
	\sigma \lambda=\frac{CkT}{I_\text{э}} \approx 7.7 \cdot 10^{-6} ~ \text{см}^3
\end{equation*}

 Ионы разгоняются в потенциале $250 \text{В}$. Тогда их скорость можно оценить $v_\text{ион} = \sqrt{\frac{2eU}{m}} = 5.8 \cdot 10^4 \: \frac{\text{м}}{\text{с}}$. Энергия ионизации азота равна $15 \text{эВ}$. После окончания серии ионизации энергия электронов будет меньше $15 \text{eВ}$. То есть скорость электронов будем меньше $2.3 \cdot 10^8 \: \frac{\text{см}}{\text{c}}$. Можно считать, что электроны попадают на анод с этой скоростью. Площадь анода: $S_\text{анод} = 0.44 см^2$. Тогда $n_\text{эл} = 4 \cdot 10^8 ~ \text{см}^{-3}$

Используя выражения из этого раздела, получаем что множитель $v_\text{ион} n_\text{эл} \sigma \lambda \approx 7 \frac{\text{см}}{\text{c}}$

% А это надо вообще?



\subsection{Зависимость силы ионного тока от температуры}

Полный адсорбционный поток равен:

$$J_\text{ад} = S_\text{св} n \left(\frac{1}{4}v_\text{т} + n_\text{эл}\sigma \lambda_{эл} v_\text{ион}\right)$$

Расчет показывает, что $n_\text{эл}\sigma \lambda_{эл} v_\text{ион} \ll v_\text{т}$. Значит полный поток равен:

$$J_\text{ад} \approx J_\text{ад, м} = \frac{1}{4} S_\text{св} n v_\text{т}$$

Десорбционный поток зависит от времени адсорбции и энергии адсорбции.

$$J_\text{дес} = \frac{N_\text{ад}}{\tau} = \frac{S_\text{зн}}{w \tau_0} \exp(-\frac{Qa}{kT})$$

При адсорбционном равновесии поток адсорбции и десорбции равны:

$$\frac{1}{4} S_\text{св} n v_\text{т} = \frac{S_\text{зн}}{w \tau_0} \exp(-\frac{Q_a}{kT})$$

$$n \sim \frac{S_\text{зн}}{S_\text{св}}\exp(-\frac{Q_a}{kT})$$

С учетом того, что $\frac{S_\text{зн}}{S_\text{св}} = \frac{\theta}{1 - \theta} \sim \frac{1}{n}$ и $n \sim I$ получаем:

\begin{align} \label{Q_a}
	I^2 \sim \exp(-\frac{Q_a}{kT}) && \ln(I/I_0)=\frac{Q_a}{2k}\frac{1}{T} && (I_0=const)
\end{align}

Это соотношение позволяет найти энергию адсорбции.
\section{Эксперементальная часть}

\subsection{Описание эксперимента}

\begin{figure}[H]
	\centering
	\includegraphics[width=0.45\textwidth]{images/PMI_photo.png}
	\includegraphics[width=0.45\textwidth]{images/VIT_photo.png}
\end{figure}

Для эксперимента использовалась запаянная лампа ПМИ-2. Ее рабочая температура составляет $40-50 ^\circ C$. В опыте лампа прогревалась до температуры вдвое больше ($\sim 100 ^\circ C$), ток эмиссии поддерживался равным $I_\text{э}=0.5$ мА с помощью вакууметра ВИТ-2. затем с помощью датчика температуры и амперметра, встроенного в вакуумметр, измеряется зависимость температуры коллектора и силы ионного тока от времени в процессе охлаждения. Датчики температуры крепились с внешней стороны лампы на уровне коллектора. Так как коллектор крепится к лампе через теплопроводящие стержни, можно считать, что температуры коллектора и стекла равны.

\subsection{Энергия адсорбции}

Для начала снимем зависимость ионного тока от температуры и используя зависимость \eqref{Q_a} – вычислим энергию адсорбции
\begin{figure}[H]
	\begin{minipage}{0.5\linewidth}
		\center 
		\includegraphics[width = \textwidth]{images/Activation.png}
	\end{minipage}
\begin{minipage}{0.5\linewidth}
		\center 
		\includegraphics[width = \textwidth]{images/Activation2.png}
	\end{minipage}
	
	\caption{Зависимость логарифма силы тока от обратной температуры $(I_0=1$ мкА) }
\end{figure}

Согласно теоретической модели -- в установившемся режиме зависимость в данных координатах должна оказаться прямой, поэтому находим этот самый прямолинейный участок (на правом графике), и по нему определяем коэффициент пропорциональности.
\begin{align*}
	\frac{Q_a}{2k}=(-1.782 \pm 0.011)\cdot 10^3~ \text{К} && Q_a = (4.92 \pm 0.03)\cdot10^{-20}\:\text{Дж} = (29.6 \pm 0.2)\:\frac{\text{кДж}}{\text{моль}}
\end{align*}


\subsection{Теплообмен в колбе}

Рассмотрим зависимость концентрации молекул в колбе от времени

\begin{figure}[H]
	\includegraphics[width = \textwidth]{images/concentrate_by_time.png}
\end{figure}

Для ультраразреженного газа коэффициент теплопроводности можно оценить как:

$$\kappa = \frac{1}{6}nm \overline{v^2}$$

уже через 40 секунд концентрация частиц составляет $10^{13} \frac{1}{\text{см}^3}$. Тогда коэффициент теплопроводности при таких значениях будет порядка $\kappa \approx 10^{-5}\frac{\text{Вт}}{\text{м}\cdot\text{К}}$. 

Характерный размер лампы порядка 1 см, температура $T = 2060 ~ \text{К}$, значит примерная мощность такой теплопередачи: $P\approx 10^{-4}$ Вт.

Оценим мощность излучения катода по закону Стефана-Больцмана: $P = \epsilon \sigma T^4$.
Длина нити катода $70 \: \text{мм}$, диаметр $100 \: \text{мкм}$, $T = 2060 \: \text{К}$.
Площадь катода = $21,98 \cdot 10^{-6}\:\text{м}^2$, $\epsilon$ для вольфрама $0.3$.
$$\text{Р} \approx 7\:\text{Вт}$$

В итоге получаем, что теплоплообмен излучением является основным для подобного вида ламп и при необходимость теплопроводностью можно принебречь.  

\subsection{Проверка модели}

В работе мы предполагали, что адсорбция проходит по модели Ленгмюра, т.е. молекулы образуют мономолекулярный слой, где каждая молекула занимает свою площадь. В такой модели коэффициент Ленгмюра $\frac{\theta}{1 - \theta}$ 
должен быть обратно пропорционален концентрации. Для проверки этого построим график $ \frac{\theta}{1-\theta} \left( \frac{1}{n} )\right $, в
оспользовавшись формулами \eqref{n} и $\theta=1-\frac{nV}{\omega S}$


\begin{figure}[H]
	\includegraphics[width = \textwidth]{images/lengmure_koef.png}
\end{figure}

Линейный вид графика говорит о том, что в действительности, использованное приближение о мономолекулярности слоя молекул выполнено достаточно точно

\section{Выводы}

\begin{itemize}
	\item Мы убедились в применимости приближения Ленгмюра в рамках данной задачи.
	\item Мы нашли энергию адсорбции молекул воздуха на поверхность коллектора. Она оказалась равна $Q_a = (29.6 \pm 0.2)\:\frac{\text{кДж}}{\text{моль}}$.
	\item Проверили, что теплообмен в колбе в данном режим производится засчет излучения.
\end{itemize}

\end{document}