\documentclass[a4paper,12pt]{article} % добавить leqno в [] для нумерации слева
\usepackage[a4paper,top=1.3cm,bottom=2cm,left=1.5cm,right=1.5cm,marginparwidth=0.75cm]{geometry}
%%% Работа с русским языком
\usepackage{cmap}					% поиск в PDF
\usepackage{mathtext} 				% русские буквы в фомулах
\usepackage[T2A]{fontenc}			% кодировка
\usepackage[utf8]{inputenc}			% кодировка исходного текста
\usepackage[english,russian]{babel}	% локализация и переносы
\usepackage{float}
\usepackage{graphicx}

\usepackage{wrapfig}
\usepackage{tabularx}

\usepackage{hyperref}
\usepackage[rgb]{xcolor}
\hypersetup{
colorlinks=true,urlcolor=blue
}

%%% Дополнительная работа с математикой
\usepackage{amsmath,amsfonts,amssymb,amsthm,mathtools} % AMS
\usepackage{icomma} % "Умная" запятая: $0,2$ --- число, $0, 2$ --- перечисление

% Номера формул
\mathtoolsset{showonlyrefs=true} % Показывать номера только у тех формул, на которые есть \eqref{} в тексте.

%% Шрифты
\usepackage{euscript}	 % Шрифт Евклид
\usepackage{mathrsfs} % Красивый матшрифт

%% Свои команды
\DeclareMathOperator{\sgn}{\mathop{sgn}}

%% Перенос знаков в формулах (по Львовскому)
\newcommand*{\hm}[1]{#1\nobreak\discretionary{}
{\hbox{$\mathsurround=0pt #1$}}{}}


\begin{document}

\begin{titlepage}
	\begin{center}
		{\large МОСКОВСКИЙ ФИЗИКО-ТЕХНИЧЕСКИЙ ИНСТИТУТ (НАЦИОНАЛЬНЫЙ ИССЛЕДОВАТЕЛЬСКИЙ УНИВЕРСИТЕТ)}
	\end{center}
	\begin{center}
		{\large Физтех-школа физики и исследований им. Ландау}
	\end{center}
	
	
	\vspace{4.5cm}
	{\huge
		\begin{center}
			{\bf Лабораторная работа 3.2.4(5)}\\
			Свободные и вынужденные колебания в электрическом контуре
		\end{center}
	}
	\vspace{2cm}
	\begin{flushright}
		{\LARGE Выполнили:\\ Мельников Антон \\
                                 Лавыгин Кирилл\\
			\vspace{0.2cm}
			Б02-213}
	\end{flushright}
	\vspace{8cm}
	\begin{center}
		Долгопрудный 2023
	\end{center}
\end{titlepage}

\section{Аннотация:}
В ходе работы 6 различными способами были получены значения добротности для двух последовательных RLC цепей. Проверена теоретическая модель колебаний.
\section{Теоретическая справка:}
Для RLC контура применяя 2 правило Кирхгофа:
\begin{equation}\label{eq:main}
    RI + U_C + L\frac{dI}{dt} = 0.
\end{equation}
Подставляя в уравнение \eqref{eq:main} выражение для тока через 1-ое правило Кирхгофа, и разделив обе части уравнения на $CL$, получаем:
\begin{equation}\label{2}
    \frac{d^2U_C}{dt^2} + \frac{R}{L} \frac{dU_C}{dt} + \frac{U_C}{CL}.
\end{equation}
Произведём замены $\gamma = \frac{R}{2L}$ -- коэффициент затухания, $\omega_0^2 = \frac{1}{LC}$ -- собственная круговая частота, $T_0 = \frac{2\pi}{\omega_0} = 2\pi \sqrt{LC}$ -- период собственных колебаний. Тогда уравнение \eqref{2} примет вид:
\begin{equation}
    \ddot{U_C} + 2 \gamma \dot{U_C} + \omega_0^2U_C = 0,
\end{equation}
где точкой обозначено дифференцирование по времени. Решение данного дифференциального уравнения:
$$U_C(t) = U(t)e^{- \gamma t}.$$
Получим:
\begin{equation}
    \ddot{U} + \omega_1^2 U = 0,
\end{equation}
где
\begin{equation}
    \omega_1^2 = \omega_0^2-\gamma^2
\end{equation}
Для случая $\gamma < \omega_0$ в силу того, что $\omega_1 > 0$, получим:
\begin{equation}\label{U}
    U_C(t) = U_0 \cdot e^{-\gamma t} \text{cos}(\omega_1 t + \varphi_0).
\end{equation}
Для получения фазовой траектории представим формулу \eqref{U} в другом виде:
\begin{equation}
    U_C(t) = e^{-\gamma t}(a \text{cos} \omega_1 t + b \text{sin} \omega_1 t),
\end{equation}
где $a$ и $b$ получаются по формулам:
$$a = U_0 \text{cos} \varphi_0, \qquad b = - U_0 \text{sin} \varphi_0.$$
Преобразовав в более удобном виде выражения для напряжения на конденсаторе и токе через катушку:
\begin{equation}
    U_C (t) = U_{C0} \cdot e^{-\gamma t} (\text{cos} \omega_1 t + \frac{\gamma}{\omega_1} \text{sin} \omega_1 t),
\end{equation}
\begin{equation}
    I(t) = C\dot{U_C}= - \frac{U_{C0}}{\rho} \frac{\omega_0}{\omega_1} e^{-\gamma t} \text{sin} \omega_1 t.
\end{equation}
Введём некоторые характеристики колебательного движения:
\begin{equation}
    \tau = \frac{1}{\gamma} = \frac{2L}{R},
\end{equation}
где $\tau$ -- время затухания (время, за которое амплитуда колебаний уменьшается в $e$ раз).
\begin{equation}
    \Theta = \text{ln} \frac{U_k}{U_{k+1}} = \gamma T_1 = \frac{1}{N_\tau} = \frac{1}{n} \text{ln} \frac{U_k}{U_{k+n}}, 
\end{equation}
где $\Theta$ -- логарифмический декремент затухания, $U_k$ и $U_{k+1}$ -- два последовательных максимальных отклонения величины в одну сторону, $N_\tau$ -- число полных колебаний за время затухания $\tau$.

Рассматривая случай \textit{вынужденных колебаний} под действием внешнего синусоидального источника, используя метод \textit{комплексных амплитуд}:
\begin{equation}
    \ddot{I} + 2 \gamma \dot{I} + \omega^2 I = - \varepsilon \frac{\Omega}{L} e^{i\Omega t}.
\end{equation}
Решая данное дифференциальное уравнение получим решение:
\begin{equation}
    I = B\cdot e^{-\gamma t} \text{sin}(\omega t - \Theta) + \frac{\varepsilon_0 \Omega}{L \phi_0} \text{sin} (\Omega t - \varphi).
\end{equation}
Очевидно, что частота резонанса будет определяться формулой:
\begin{equation}\label{LC}
    \omega_0 = \frac{1}{2 \pi \sqrt{LC}}.
\end{equation}

Способы измерения \textbf{добротности}:
\begin{enumerate}
    \item с помощью потери амплитуды свободных колебаний: 
    \begin{equation}
        \theta = \frac{1}{n} \text{ln}\frac{U_k}{U_{k+n}},
    \end{equation}
    \item с помощью среза АЧХ на уровне 0.7 от максимальной амплитуды, тогда <<дисперсия>> ($\Delta \Omega$) будет численно равна коэффициенту $\gamma$, то есть $Q = \frac{\nu_0}{2 \Delta \Omega}$.
    \item с помощью среза ФЧХ, отразив график относительно оси $\pi/2$ по формуле $Q=\omega_0/\Delta\omega$
    \item с помощью нарастания амплитуд в вынужденных колебаниях:
    \begin{equation}
        Q = \frac{\nu_0 n}{2\text{ln} \frac{U_0 - U_k}{U_0 - U_{k+n}}}.
    \end{equation}
\end{enumerate}
%\subsection{Затухающие колебания}
%	Частота свободных колебаний LC контура: 
%	\begin{equation} \label{LC}
%		\nu_{0}=\frac{1}{2 \pi \sqrt{L C}}=\frac{1}{T} 
%	\end{equation}
%
%Критическое сопротивление, разделяющее периодический и апериодический режимы колебаний: 
%\begin{align}
%	R_{c r}=2 \sqrt{L / C}
%\end{align}
%
%Условие реализации затухающих колебаний в $R L C$ контуре: $$0<R<R_{c r}$$
%Логарифмический декримент затухания:
%\begin{equation}
%	\Theta=\frac{1}{n} \ln \frac{U_m}{U_{m+n}}
%\end{equation}
\section{Экспериментальная установка:}
\begin{figure}[h!]

	\center
	\includegraphics[width=\textwidth]{ust}
\caption{Конфигурация установки для изучения затухаюших колебаний}
\label{ust}
\end{figure}

Схема установки для исследования колебаний приведена на \figurename \ref{ust}. Отметим, что сигнал с генератора поступает через конденсатор $C_1$ на вход колебательного контура. Данная емкость необходима чтобы выходной импеданс генератора был много меньше импеданса колебательного контура и не влиял на процессы, проходящие в контуре.

Установка предназначена для исследования не только возбужденных, но и свободных колебаний в электрической цепи. При изучении свободно затухающих колебаний генератор специальных сигналов на вход колебательного контура подает пери- одические короткие импульсы, которые заряжают конденсатор $C$. За время между последовательными импульсами происходит разрядка конденсатора через резистор и катушку индуктивности. Для наблюдения фазовой картины затухающих колебаний на канал 2(Y) подается напряжение с резистора $R$ (пунктирная линия на схеме установки), которое пропорционально току $I$ ($I \propto  dU_C /dt$).

\begin{figure}[h!]

	\center
	\includegraphics[width=\textwidth]{ust2}
\caption{Конфигурация установки для изучения возбужденных колебаний}
\label{ust}
\end{figure}

При изучении возбужденных колебаний на вход колебательного контура подается синусоидальный сигнал. С помощью осциллографа возможно измерить зависимость амплитуды возбужденных колебаний в зависимости от частоты внешнего сигнала, из которого возможно определить добротность колебательного контура. Альтернативным способом расчета добротности контура является определение декремента затухания по картине установления возбужденных колебаний. В этом случае генератор сигналов используется для подачи цугов синусоидальной формы.
\section{Свободные колебания}

Определим $C_0$ (нулевую емкость колебательного контура), которую в дальнейшем необходимо будет прибавлять к значению $C$ магазина емкостей. При установленных $R=0.01$ Ом, $L=100$ мГн, $C=0$ (емкость оставим неизменной в течении всей работы)получено значение периода (мы измерили пять периодов, чтобы повысить точность):
\begin{align}
	T=0.0654~\text{мс} && C_0= 1.07~ \text{пФ}
\end{align}
Далее проверим уравнение для периода свободных колебаний и установленное значение $C_0$, построив график зависимости экспериментального значения T от теоретического полученного по формуле \eqref{LC}

\begin{figure}[H]

	\center
	\includegraphics[width=\textwidth]{free}

\end{figure}

Этот график показывает достаточно хорошую точность наших измерений и предложенной модели, погрешность свободного члена составила  $\Delta T=0.0005$ мс. 
\subsection{Критическое сопротивление}
Взяв целевую частоту собственных колебаний за $\nu_0=6.5$ кГц получаем следующие значения необходимого сопротивления $C^*$ и критического сопротивления $R_\text{кр}^{th}$. $R_\text{кр}^{exp}$ -- значение критического сопротивления, определенное по смене колебательного режима (достаточно неточное из-за гистерезиса):
\begin{align}
	C^*=6~\text{пФ} && R_\text{кр}=8 ~\text{кОм} && R_\text{кр}^{exp}=6~\text{кОм}
\end{align}
Далее рассчитаем $\theta$ по потере амплитуды для различных $R$ и построим следующий график:
\begin{figure}[H]

	\center
	\includegraphics[width=\textwidth]{theta}
\label{theta}
\end{figure}

По данному графику можно рассчитать значение $R_\text{кр}$ как: 
\begin{align}
	R_\text{кр}^{gr}=2\pi \sqrt{k}=0.858R_\text{кр}^{th}\approx 6.8 ~\text{кОм} && \Delta R_\text{кр}^{gr}=0.012R_\text{кр}^{th}\approx 0.1 ~\text{кОм}
\end{align}
Из всех предложенных способов наиболее точным является последний. 
\subsection{Добротность}
В дальнейшей работе за $R_\text{кр}$ принимается $R_\text{кр}=6$ кОм.

Расчитаем добротности по полученным значениям $\theta$ для $0.05 R_\text{кр}$ и $0.25 R_\text{кр}$
\begin{align}
	Q(0.05R_{\text{кр}})=8.16 && Q(0.25R_{\text{кр}})=1.70
\end{align}

Теперь рассчитаем добротности по значениям $\theta$ полученым их фазовой диаграммы затухающих колебаний:
\begin{align}
	Q(0.05R_{\text{кр}})=7.87 && Q(0.25R_{\text{кр}})=1.7
\end{align}
\section{Вынужденные колебания}
\subsection {АЧХ}
\begin{figure}[H]

	\center
	\includegraphics[width=\textwidth]{ARC}
\end{figure}

С помощью среза АЧХ на уровне 0.7 рассчитаем добротности:
\begin{align}
	Q(0.05R_{\text{кр}})=7.5 && Q(0.25R_{\text{кр}})=1.4
\end{align}
Стоит заметить, что нашего диапазона измерений для $0.25R_\text{кр}$ оказалось недостаточно, поэтому полученное значение является лишь оценочным
\subsection{ФЧХ}
\begin{figure}[H]
	\center
	\includegraphics[width=\textwidth]{FRC}
\end{figure}
Здесь при измерениях 0.25 случилась какая-то ошибка. Исходно получаемое резонансное значение не совпадало с полученным ранее. Скорее всего была ошибка при выставлении кривых относительно друг-друга. После исправления позиционирования оказалось, что тут нам так-же не хватило диапазона в сторону больших частот, из-за чего полученное знаение для 0ю25 так-же является лишь оценкой.

Расчитанные значения:
\begin{align}
	Q(0.05R_{\text{кр}})=7.7 && Q(0.25R_{\text{кр}})=1.3
\end{align}
\subsection{Нарастание и затухание колебаний}
При измерении затухающих колебаний были получены следующие значения (1,2 для раскачки, з - для затухания):
\begin{align}
	Q_1(0.05R_{\text{кр}})=7.5 && Q_2(0.05R_{\text{кр}})=7.3 && Q_\text{з}(0.05R_{\text{кр}})=8.2 \\
	 Q_1(0.25R_{\text{кр}})=1.3 && Q_\text{з}(0.25R_{\text{кр}})=1.8
\end{align}

\section{Выводы}
Итоговая таблица:
\begin{center}
	\begin{tabular}{|l|c|c|c|c|c|c|c|}
\hline \multirow{$R$} & \multicolumn{3}{|c|}{ Свободные колебания } & \multicolumn{4}{c|}{ Вынужденные колебания } \\
\cline { 2 - 8 } & $f(L, C, R)$ & $f(\Theta)$ & Спираль & АЧХ & ФЧХ & Нарастание & Затухание \\
\hline$0.05R_{\text{кр}}$ &12.6 & 8.2 & 7.9 & 7.5 & 7.7 & 7.4 & 8.2\\
\hline$0.25R_{\text{кр}}$ & 2.5& 1.7 & 1.7 & 1.4 & 1.3 & 1.3 & 1.8\\
\hline
\end{tabular}
\end{center}


Полученные значения добротности, с которыми не возникло проблем, неплохо совпадают. Наиболее точными являются значения полученные из АЧХ и ФЧХ, так как данные методы опираются на существенно больший набор данных и попутно позволяют проверить верность предложенной модели явления и обнаружить возможную ошибку при расчетах или измерениях.

 Теоретические значения ощутимо отличаются от экспериментальных. Мы связываем это с тем, что предложенная модель не позволяет учесть сопротивления всех элементов цепи, и их индуктивности, поскольку сам зависимости, предсказанные нашей теорией, реализуются на практике.
\end{document}