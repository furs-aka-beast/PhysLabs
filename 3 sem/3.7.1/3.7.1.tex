\documentclass[a4paper,12pt]{article} % добавить leqno в [] для нумерации слева
\usepackage[a4paper,top=1.3cm,bottom=2cm,left=1.5cm,right=1.5cm,marginparwidth=0.75cm]{geometry}
%%% Работа с русским языком
\usepackage{cmap}					% поиск в PDF
\usepackage{mathtext} 				% русские буквы в фомулах
\usepackage[T2A]{fontenc}			% кодировка
\usepackage[utf8]{inputenc}			% кодировка исходного текста
\usepackage[english,russian]{babel}	% локализация и переносы
\usepackage{float}
\usepackage{graphicx}

\usepackage{wrapfig}
\usepackage{tabularx}

\usepackage{hyperref}
\usepackage[rgb]{xcolor}
\hypersetup{
colorlinks=true,urlcolor=blue
}
\usepackage[arrowdel]{physics}
%%% Дополнительная работа с математикой
\usepackage{amsmath,amsfonts,amssymb,amsthm,mathtools} % AMS
\usepackage{icomma} % "Умная" запятая: $0,2$ --- число, $0, 2$ --- перечисление

% Номера формул
\mathtoolsset{showonlyrefs=true} % Показывать номера только у тех формул, на которые есть \eqref{} в тексте.

%% Шрифты
\usepackage{euscript}	 % Шрифт Евклид
\usepackage{mathrsfs} % Красивый матшрифт

%% Свои команды
\DeclareMathOperator{\sgn}{\mathop{sgn}}

%% Перенос знаков в формулах (по Львовскому)
\newcommand*{\hm}[1]{#1\nobreak\discretionary{}
{\hbox{$\mathsurround=0pt #1$}}{}}


\begin{document}

\begin{titlepage}
	\begin{center}
		{\large МОСКОВСКИЙ ФИЗИКО-ТЕХНИЧЕСКИЙ ИНСТИТУТ (НАЦИОНАЛЬНЫЙ ИССЛЕДОВАТЕЛЬСКИЙ УНИВЕРСИТЕТ)}
	\end{center}
	\begin{center}
		{\large Физтех-школа физики и исследований им. Ландау}
	\end{center}
	
	
	\vspace{4.5cm}
	{\huge
		\begin{center}
			{\bf Лабораторная работа 3.7.1}\\
			Скин-эффект в полом цилиндре
		\end{center}
	}
	\vspace{2cm}
	\begin{flushright}
		{\LARGE Выполнили:\\ Мельников Антон \\
                                 Лавыгин Кирилл\\
			\vspace{0.2cm}
			Б02-213}
	\end{flushright}
	\vspace{8cm}
	\begin{center}
		Долгопрудный 2023
	\end{center}
\end{titlepage}

\section{Аннотация:}
В ходе работы будет исследовано проникновение магнитного поля в полый цилиндр, измерена его проводимость различными способами, основанными на этом эффекте. 
\section{Теоретическая часть}
\subsection{Скин-эффект для полупрастранства}
\vspace{1cm}
\begin{wrapfigure}{l}{0.3\textwidth}
  \begin{center}
    \includegraphics[width=0.28\textwidth]{poluprostranstvo}
  \end{center}
  \caption{Скин-эффект в полупространстве}\label{fig:poluprostranstvo}
\end{wrapfigure}

Рассмотрим квазистационарное поле внутри проводящей среды в простейшем плоском случае.
Пусть вектор $\vb*{E}$ направлен всюду вдоль оси $y$ (рис.\ref{fig:poluprostranstvo}) 
и зависит только от координаты $x$, т. е. ${E_x} = {E_z} \equiv 0$, $E_y=E_y(x,t)$.
В квазистационарном приближении 
\begin{align}
    \grad \times \vb*{H} = \sigma \vb*{E}
\end{align}
Берем ротор обоих частей
\begin{align}
    \grad \times (\grad \times \vb*{H}) = \grad(\grad \cdot \vb*{H}) - \grad^2\vb*{H} = \sigma \grad \times \vb*{E}
\end{align}
Испоьзуя ур-е Максвелла для ротора $\vb*{E}$ и для дивергенции $\vb*{H}$ получаем
\begin{align}
    \grad^2 \vb*{H} = \sigma\mu\mu_0\frac{\partial\vb*{H}}{\partial t} 
                      + \grad(\grad \cdot \vb*{H}) = \sigma\mu\mu_0\frac{\partial\vb*{H}}{\partial t} 
    \label{eq:laplacian_H}
\end{align}
Берем ротор еще раз
\begin{align}
    \grad \times (\grad^2\vb*{H}) = \grad^2 (\grad \times \vb*{H}) =
    \sigma\mu\mu_0 \frac{\partial (\grad \times \vb*{H})}{\partial t}
\end{align}
Осталось подставить первое ур-е, и воспользоватся уравнением Максвелла
\begin{align}\label{eq:diffusion}
    \grad^2\vb*{E}=\sigma\mu\mu_0 \frac{\partial \vb*{E}}{\partial t}
\end{align}

Подставляем в \eqref{eq:diffusion} наше электрическое поле $E_y=E_y(x,t)$
\begin{align}
    \frac{\partial^2 E_y}{\partial x^2} = \sigma\mu\mu_0\frac{\partial E_y}{\partial t}
    \label{eq:diffusion_chastni}
\end{align}
Если $E_y(0,t)=E_0 e^{i\omega t}$ то решением \eqref{eq:diffusion_chastni} будет функция вида
\begin{align}
    E_y(x,t)=E_0 e^{-x/\delta} e^{i(\omega t - x/\delta)}
    \label{eq:skin_effect_poluprostranstvo}
\end{align}
где
\begin{align}
    \delta = \sqrt{\frac{2}{\omega\sigma\mu\mu_0}}
    \label{eq:delta}
\end{align}

 
\subsection{Скин-эффект в тонокм полом цилиндре}
\vspace{1cm}
\begin{wrapfigure}[36]{l}{0.3\textwidth}
  \begin{center}
    \includegraphics[width=0.28\textwidth]{cilindr}
  \end{center}
  \caption{Эл-магнитные поля в цилиндре}\label{fig:cilindr}

  \begin{center}
    \includegraphics[width=0.28\textwidth]{stenka}
  \end{center}
  \caption{Стенка цилиндра}\label{fig:stenka}
\end{wrapfigure}

Перейдем теперь к описанию теории в нашей работе. Из соображении симметрии и 
непрерывности соответствующих компонет векторов $\vb*{E}$ и $\vb*{H}$ можем сказать что
\begin{align}
    H_z = H(r)e^{i\omega t} \text{, } E_\varphi = E(r)e^{i\omega t}
\end{align}
и при этом функции $H(r)$ и $E(r)$ непрерывны.

Внутри цилиндра токов нет, следовательно $H(r)=H_1=\text{const}$ внутри цилиндра.
По теореме об электромагнитной индукции
\begin{align}
    E(r) = -\frac{1}{2}\mu_0 r \cdot i \omega H_1
\end{align}
откуда мы получаем граничное условие
\begin{align}
    E_1=E(a)= -\frac{1}{2}\mu_0 a \cdot i \omega H_1
    \label{eq:granichnoe_uslovie_E}
\end{align}

В прближении $h \ll a$ можем пренебречь кривизной стенки и смоделировать 
его бесконечной полосой. Тогда, надо решить уравнение \eqref{eq:laplacian_H}
с граничными условиями. Решая уравнение получим связь полей $H_1$ 
(поле внутри цилиндра которое мы будем измерять) и $H_2$, которое колебается с частотой
$\omega$

\begin{align}
    H_1 = \frac{H_0}{\ch(\alpha h) + \frac{1}{2} \alpha a \sh(\alpha h)} 
    \text{\ \ \ }
    \alpha = \sqrt{i\omega \sigma \mu_0} = \frac{\sqrt{2}}{\delta}e^{i\pi/4}
    \label{eq:svyaz_poley}
\end{align}

из этой формулы получим сколько по фазе отстает поле $H_1$ от $H_0$. При $\delta \ll h$
(высокачастотная область)

\begin{align}
    \psi \approx \frac{\pi}{4} + \frac{h}{\delta} = 
    \frac{\pi}{4} + h \sqrt{\frac{\omega \sigma \mu_0}{2}}
    \label{eq:faza_high_freq}
\end{align}

При $\delta \gg h$ (низкочастотная область)

\begin{align}
    \tan \psi \approx \frac{ah}{\delta^2} = \pi a h \sigma \mu \mu_0 \nu
    \label{eq:faza_low_freq}
\end{align}
 

\subsection{Процесс измерения}
\begin{wrapfigure}{l}{0.5\textwidth}
  \begin{center}
    \includegraphics[width=0.48\textwidth]{ustanovka}
  \end{center}
  \caption{Установка}\label{fig:ustanovka}
\end{wrapfigure}

Мангнитное поле внутри цилиндра измеряется катушкой 3. Напряжение на катушке
пропорционалньна производной $\dot{B_1}(t)$
\begin{align}
    U(t) \propto \dot{B_1}(t) = -i\omega H_1 e^{i\omega t}
\end{align}
Поле внутри цилиндра пропорциональна току через соленоид
\begin{align}
    B_0(t) \propto I(t)
\end{align}
Отсюда несложно увидеть, что
\begin{align}
    \frac{\abs{H_1}}{\abs{H_0}} = \xi_0^{-1} \cdot \frac{U}{\nu I} = \xi \xi_0
    \label{eq:otnoshenie_amplitud}
\end{align}
\vspace{0.3cm}

где константу можно определить из условия $\abs{H_1}/\abs{H_2} \rightarrow 1$ при
$\nu \rightarrow 0$.

\vspace{0.3cm}

При измерениях разности фаз нужно учесть, что первый сигнал на осциллографе
пропорционален магнитному полю снаружи, а второй пропорционален производному
поля внутри цилиндра по времени. Вследствии этого набегает дополнительная фаза $\pi/2$,
которую надо вычесть при измерениях.

 
\section{Ход работы}

Параметры нашей установки $2a = 45$ мм, $h=1.5$ мм. Проводимость порядка
$\sigma \sim 5\cdot 10^7$ См/м. Получаем оценку для частоты, при которой
глубина проникновения равна толщине стенок цилиндра $\nu_h = $2250 Гц.

\subsection{Измерение проводимости через отношение амплитуд}
В области частот $\nu \ll \nu_h$ $\alpha h \ll 1$, и из \eqref{eq:svyaz_poley} получаем
\begin{align}
    {(\xi/\xi_0)}^2 \approx \frac{1}{1+A^2\nu^2} && A=\pi a h \sigma \mu_0
\end{align}
или, эквивалентно
\begin{align}
    \frac{1}{\xi^2}=B^2\nu^2 + 1/\xi_0^2 && B=\pi a h \sigma \mu_0 /\xi_0
    \label{eq:liniya_dlya_c}
\end{align}

\begin{figure}[H]
    \center{\includegraphics[width=\textwidth]{xi_nu_low_freq_linearized}}
    \caption{График зависимости $1/\xi^2(\nu)$}\label{fig:xi_nu_low_freq_linearized}
     
\end{figure}

Из графика получаем значение $\nu_0$, а так же проводимость меди $\sigma$
\begin{align}
    \xi_0=(0.014533 \pm 0.000006) ~ (\text{В/А/Гц})^2 && \sigma = (4.432 \pm 0.002) \cdot 10^7 ~ \text{См/м}
\end{align}

 

\subsection{Измерение проводимости через разность фаз в низкочастотном диапазоне}
Согласно формуле \eqref{eq:faza_low_freq}, при $\delta \gg h$
\begin{align}
    \tan \psi = k \nu \ \text{; } k = \pi a h \sigma \mu_0 \ \ (\mu = 1)
\end{align}

\begin{figure}[H]
    \center{\includegraphics[width=\textwidth]{tg_psi_nu_line}}
    \caption{График зависимости $\tan \psi (\nu)$ }\label{fig:tg_psi_nu_line}
     
\end{figure}

\vspace{1cm}
Из коэффициента наклона прямой находим проводимость
\begin{align}
    \sigma = (4.5 \pm 1.2) \cdot 10^7 \text{См/м}
\end{align}

 

\begin{figure}[H]
    \center{\includegraphics[width=\textwidth]{tg_psi_nu_no_line}}
    \caption{График зависимости $\tan \psi (\nu)$ (нелинейная часть)}\label{fig:tg_psi_nu_no_line}
     
\end{figure}

\subsection{Измерение проводимости через разность фаз в высокачастотном диапазоне}
Согласно формуле \eqref{eq:faza_high_freq}, при $\delta \ll h$
\begin{align}
    \psi - \pi/4 = k\cdot \sqrt{\nu}; \ k = h\sqrt{\pi\mu_0\sigma}
\end{align}

Из графика получаем следующее значение проводимости

\begin{align}
    \sigma = (4.6 \pm 0.6) \cdot 10^7 \text{См/м}
\end{align}

 

\begin{figure}[H]
    \center{\includegraphics[width=0.9\textwidth]{psi_sqrt_nu}}
    \caption{График зависимости $(\psi - \pi/4)(\sqrt{\nu})$}\label{fig:psi_sqrt_nu}
     
\end{figure}

\subsection{Измерение проводимости через изменение индуктивности}

Из за наличия цилиндра внутри, индуктивность внешней катушки зависит от катушки
следующим образом

\begin{align}
    \frac{L_{\max} - L}{L - L_{\min}} = \pi ^2 a^2 h^2 {\mu_0}^2 \sigma^2 \nu^2
\end{align}

\begin{figure}[H]
    \center{\includegraphics[width=\textwidth]{L_nu}}
    \caption{График зависимости $\frac{L_{\max} - L}{L - L_{\min}} (\nu^2)$}\label{fig:L_nu}
     
\end{figure}

 

\begin{figure}[H]
    \center{\includegraphics[width=\textwidth]{L_nu_linearized}}
    \caption{График зависимости $\frac{L_{\max} - L}{L - L_{\min}} (\nu^2)$}\label{fig:L_nu_linearized}
     
\end{figure}

\vspace{1cm}
$L_{\max}$ и $L_{\min}$ ищем в области монотонности. Далее, линеаризуя данные по формуле
выше получаем линейную зависимость при малых $\nu$. По наклону кривой находим

\begin{align}
    \sigma = (4.240 \pm 0.007) \cdot 10^7 ~\text{См/м}
\end{align}

 

\subsection{Отношение магнитных полей}
Отношение $\abs{H_1}/\abs{H_0}$ можем посчитать двумя способами. Первый способ - через
формулу \eqref{eq:otnoshenie_amplitud},использовав значение $\xi_0$ из пункта (2.1).
Второй способ - через теоретическую формулу \eqref{eq:svyaz_poley}, использовав значение
$\sigma$ из пункта (2.1). Посмотрим на их различие с помощью графиков зависимости
$\abs{H_1}/\abs{H_0} (\nu)$

\begin{figure}[H]
    \center{\includegraphics[width=\textwidth]{all_freq_ratio}}
    \caption{Отношение полей}\label{fig:all_freq_ratio}
     
\end{figure}

Как можно видеть ошибка теоретической модели достаточно небольшая. Предлагаемая модель описывает зависимость во всем рассмотренном диапазоне, что позволяет сделать вывод о ее корректности.
\section{Вывод}
Мы измерили проводимость материала цилиндра 4 разными способами. Сравним эти данные
между собой

\begin{table}[!h]
\begin{center}
\begin{tabular}{|l|r|r|r|}
\hline
Метод измерения & $\sigma, 10^{7} \text{См/м}$ & $\Delta\sigma, 10^{7} \text{См/м}$ & $\varepsilon_{\sigma}$\\
\hline

Отношение амплитуд & 4.432 & 0.002 & 0.05\%\\
\hline

Разности фаз (низкие частоты) & 4.5 & 26 & \%\\
\hline

Разности фаз (высокие частоты) & 4.6 & 0.6 & 13\%\\
\hline

Индуктивность & 4.240 & 0.007 & 0.17\%\\
\hline


\end{tabular}
\end{center}
    \caption{Сравнение результатов различных методов}\label{}
\end{table}

Для меди проводимость состовляет $\sigma_{обыч} = (5.55-5.80)\cdot10^{7}$ См/м.
Учитывая высокую точность измерения первым методом, значения очевидно не совпадают. У этого может быть две причины: пренебрежение краевыми эффектами или неидеальность меди трубы (различные примеси и вкрапление, неидеальная структура из-за способа изготовления). Второе, как мне кажется, вероятнее. 

	Измерение через сдвиг фазы на низкой частоте получились очень неточным. Это произошло из-за большой погрешности тангенса (т.к. тета это разница двух близких величин) и небольшого количества точек в линейной области
	
	Измерения через сдвиг на высоких частотах получились лучше, но погрешность все еще выше чем в первом методе.
	
	Измерения через индуктивность получились достаточно точными. Нелинейная часть присутствует по той причине, что при почти нулевом проникании поля в трубу основной вклад вносит поле в теле трубы, которое мы не учитываем.
	Результат не совпал с первым пунктом. Это может быть связано с ошибкой измерительного прибора (во время измерений на некоторых частотах его показания были абсурдны, поэтому доверия он не вызывает) или неучтенным полем в теле проводника
	
	Как уже было сказано в п 3.5, предложенная зависимость описывает явление в рассматриваемом диапазоне

\end{document}