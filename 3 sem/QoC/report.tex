\documentclass[a4paper,12pt]{article}
\usepackage[a4paper,top=1.3cm,bottom=2cm,left=1.5cm,right=1.5cm,marginparwidth=0.75cm]{geometry}
\usepackage{setspace}
\usepackage{cmap}		
\usepackage{amsmath,amsfonts,amssymb,amsthm,mathtools} 			
 				
\usepackage[T2A]{fontenc}			
\usepackage[utf8]{inputenc}			
\usepackage[english,russian]{babel}
\usepackage{multirow}
\usepackage{graphicx}
\usepackage{wrapfig}
\usepackage{tabularx}
\usepackage{float}
\usepackage{longtable}
\usepackage{hyperref}
\hypersetup{colorlinks=true,urlcolor=blue}
\usepackage[rgb]{xcolor}
\usepackage{icomma} 
\usepackage{euscript}


\DeclareMathOperator{\sgn}{\mathop{sgn}}
\newcommand*{\hm}[1]{#1\nobreak\discretionary{}
	{\hbox{$\mathsurround=0pt #1$}}{}}


\title{\textbf{Разработка stepped-
impedance СВЧ фильтра и
notch-port $\lambda/4$ резонатора}}
\author{Лавыгин Кирилл}
\date{\today}


\begin{document}
	
	\maketitle
	
	\section{Введение}
	При работе с различными сверхпроводящими системами, в частности с джозефсоновскими переходами, характерная частота подаваемых и излучаемых импульсов - гигагерцы. Как и в любых системах возникают задачи по фильтрации различных частот входящего и выходящего сигналов. Использование классических LC фильтров на таких частотах невозможно ввиду наличия паразитной индуктивности у емкости. Из-за этой индуктивности опущенной на землю обычные LC фильтры почти не пропускают высокие частоты.
	
	Эти проблемы решают микрополосковые (microstrip) фильтры.
	
	В ходе работы будет создана модель stepped-impedance (микрополоскового) фильтра НЧ и notch-port $\lambda/4$ резонатора (изначально его планировалось изготовить для проверки работы фильтра. В общем они тоже довольно часто используются в работе с сверхпроводниками). Данные модели будут проверены при помощи симуляции в среде sonnet.
	
	Изначально планировалось изготовить данные структуры, но к сожалению на это не хватило времени
	\section{Stepped impedance filter}
	Stepped-impedance фильтр низких частот имеет следующий вид (чередующиеся полоски с высоким и низким импедансами:
	\begin{figure}[H]
    \center{\includegraphics[width=\textwidth]{SiF}}
     
\end{figure}
	Справа представлена структура материала в разрезе -- заземленный слой проводника отделенный диэлектриком от металлического слоя с изображенной слева структурой.
	\subsection{Теоретическая модель}
	Сразу оговоримся что в данной части мы пренебрегаем всеми потерями. Моделирование покажет, что это приближение работает с хорошей точностью.
	
	Короткую проводящую линию без потерь можно представить в следующем виде:
	\begin{figure}[H]
    \center{\includegraphics[width=0.8\textwidth]{eqscheme}}
     
	\end{figure}
	Рассмотрим одну полоску фильтра (здесь и в дальнейшем $\gamma=\alpha+i\beta$):
	$$
\left[\begin{array}{l}
V_{1} \\
I_{1}
\end{array}\right]=\left[\begin{array}{ll}
A & B \\
C & D
\end{array}\right]\left[\begin{array}{l}
V_{2} \\
I_{2}
\end{array}\right] \quad \begin{array}{ll}
A=\cos \beta \ell & B=j Z_{0} \sin \beta \ell \\
C=j Y_{0} \sin \beta \ell & D=\cos \beta \ell
\end{array}
$$
$$
\left[\begin{array}{l}
V_{1} \\
V_{2}
\end{array}\right]=\left[\begin{array}{ll}
Z_{11} & Z_{12} \\
Z_{21} & Z_{22}
\end{array}\right]\left[\begin{array}{l}
I_{1} \\
I_{2}
\end{array}\right]
$$
$$
\begin{aligned}
& Z_{11}=Z_{22}=\frac{A}{C}=-j Z_{0} \cot \beta \ell \\
& Z_{12}=Z_{21}=\frac{1}{C}=-j Z_{0} \csc \beta \ell .
\end{aligned}
$$
$$
Z_{\text {series }}=Z_{11}-Z_{12}=-j Z_{0}\left(\frac{\cos \beta \ell-1}{\sin \beta \ell}\right)=j Z_{0} \tan \left(\frac{\beta \ell}{2}\right)
$$
$$
\mathrm{Z}_{\text {shunt }}=Z_{12}=-j Z_{0} / \sin \beta l
$$
$$
\begin{aligned}
\frac{X}{2} =Z_{0} \tan \left(\frac{\beta \ell}{2}\right) &&
B =\frac{1}{Z_{0}} \sin \beta \ell
\end{aligned}
$$
Теперь воспользуемся тем, что $\beta l<\pi / 4$ :
	\begin{figure}[H]
    \center{\includegraphics[width=0.45\textwidth]{hi-lo}}
     
	\end{figure}
Нормализованные величины (их принято использовать для расчета подобного рода цепей):
$$
C=B R_{0}, L=X / R_{0} \quad R_{0} \text { - импеданс фильтра }
$$
$$\beta \ell=\frac{L R_{0}}{Z_{h}} \quad\text{(inductor)} \quad \beta \ell=\frac{C Z_{\ell}}{R_{0}} \quad\text{ (capacitor)}$$
 	\begin{figure}[H]
    \center{\includegraphics[width=0.65\textwidth]{genvalues.png}}
    
    \center{\includegraphics[width=0.5\textwidth]{genvalues2.png}}
     
	\end{figure}
\subsection{Параметры изготовления}
Для изготовления были выбраны следующие материалы (выбор сделан ввиду их распространенности, так как это обычные печатные платы)

Диэлектрик - текстолит FR4 tg$135$ 1.5мм $\epsilon_{r} \approx 4,58, \tan (\delta) \approx 0.02$

Металлизация - медь 18 мкм, $W_{\min }=500$ мкм, $W_{\max }=10000$ мкм

Для прехода от пластины с диэлектриком к пространству с диэлектриком введем эффективное значение $\epsilon$:
	\begin{figure}[H]
    \center{\includegraphics[width=0.7\textwidth]{eps_eff}}
     
	\end{figure}
$$\epsilon_{e}=\frac{\epsilon_{r}+1}{2}+\frac{\epsilon_{r}-1}{2} \frac{1}{\sqrt{1+12 d / W}}$$
$$Z_{\text {min }}=\frac{120 \pi}{\sqrt{\epsilon_{e}}\left[W_{\text {max }} / d+1.393+0.667 \ln \left(W_{\text {max }} / d+1.444\right)\right]}=20.30 \text{ Ом}$$
$$
\begin{align}
Z_{\text {max }}=\frac{60}{\sqrt{\epsilon_{e}}} \ln \left(\frac{8 d}{W_{\text {min }}}+\frac{W_{\text {min }}}{4 d}\right)=108.7 \text { Ом } &&
\beta=\frac{2 \pi f}{v_{p}}=\frac{2 \pi f \sqrt{\epsilon_{e}}}{c}
\end{align}
$$
Желаемая частота пропускания: $\omega_{\text {cut-off }}=3 \mathrm{GHz}$. Фильтр будет состоять из 6 полосок потому что это дает достаточно хороший результат при небольших размерах.
\subsection{Итоговый дизайн}
	Для создания чертежа была написана программа на python с использованием api klayout которая генерирует чертеж по заданным параметрам материала и фильтра.
	\begin{figure}[H]
    \center{\includegraphics[width=\textwidth]{finalfilter}}
     
	\end{figure}
\subsection{Симуляция с учетом потерь в Sonnet}
	В итоге наша теория дала хороший график S-параметров не смотря на пренебрежение потерями.
	\begin{figure}[H]
    \center{\includegraphics[width=\textwidth]{graph}}
     
	\end{figure}
\section{Notch-port $\lambda/4$ резонатор}
	Notch-port означает что резонатор емкостно связан с копланаром (основной проводящей линией). $\lambda/4$ - его длина
	
	Синим на чертеже обозначено via -- металлизированное отверстие в плате, соединяющее верхний слой с нижним (заземленным) 
	
		\begin{figure}[H]
    \center{\includegraphics[width=\textwidth]{resonator1}}
     
	\end{figure}
	\subsection{Добротность}
	Добротность по определению:
	$$
Q \stackrel{\text { def }}{=} 2 \pi \times \frac{\text { energy stored }}{\text { energy dissipated per cycle }}
$$
В действительности у нас есть две причины потерь: при передаче энергии в резонатор (capacitive) и в замом резонаторе непосредственно (inernal)
\begin{figure}[H]
    \center{\includegraphics[width=0.7\textwidth]{quality1}}
     
	\end{figure}
	Связаны они получаются таким соотношением:
$$\frac{1}{Q_{r}}=\frac{1}{Q_{c}}+\frac{1}{Q_{i}}$$

Мне была поставлена задача сделать резонатор с FWHM(шириной на полувысоте) порядка 10 МГц при резонансной частоте в 1500 Мгц, тогда желаемая добротность:
$$
Q \approx \frac{f_{r}}{\Delta f}=\frac{1500 \mathrm{MHz}}{10 \mathrm{MHz}}=150
$$
\begin{figure}[H]
    \center{\includegraphics[width=0.6\textwidth]{fwhm}}
     
	\end{figure}


\subsection{Внутренняя добротность}
Эквивалентная схема и ее импеданс:
\begin{figure}[H]
    \center{\includegraphics[width=0.7\textwidth]{eqsheme}}
     
	\end{figure}
	\begin{align*}
Z_{i n}=\left(\frac{1}{R}+\frac{1}{j \omega L}+j \omega C\right)^{-1} \approx \frac{1}{(1 / R)+2 j \Delta \omega C}, \quad \Delta \omega \ll \omega_{0},  && Q_{i}=\frac{R}{\omega_{0} L}=\omega_{0} R C
\end{align*}
А для реальной проводящей линии:
$$Z_{\text {in }}=Z_{0} \tanh (\alpha+j \beta) \ell \quad ~~~ \beta \ell=\frac{\omega_{0} \ell}{v_{p}}+\frac{\Delta \omega \ell}{v_{p}}=\frac{\pi}{2}+\frac{\pi \Delta \omega}{2 \omega_{0}}$$
Откуда можно получить параметры эквивалентных элементов:
$$Z_{i n} \approx \frac{Z_{0}}{\alpha \ell+j \pi \Delta \omega /\left(2 \omega_{0}\right)} \quad R=\frac{Z_{0}}{\alpha \ell} \quad C=\frac{\pi}{4 \omega_{0} Z_{0}} \quad L=\frac{1}{\omega_{0}^{2} C}$$
Откуда добротность:
$$
Q_{i}=\frac{\beta}{2 \alpha}, \text { т.к. } \ell=\pi / 2 \beta \quad \beta=\frac{2 \pi f}{v}=\frac{2 \pi f \sqrt{\epsilon_{e}}}{c}
$$
Источников потерь у нас два: в диэлектрике и в проводнике
$$
\alpha=\alpha_{c}+\alpha_{d} \quad \alpha_{c}=\sqrt{\frac{\pi f \mu_{0}}{\sigma}} /\left(Z_{0} W\right) \quad \alpha_{d}=\frac{\pi f \epsilon_{r}\left(\epsilon_{e}-1\right) \tan \delta}{c\left(\epsilon_{r}-1\right)}
$$
\subsection{Внешняя добротность}
Для расчета добротности можно представить наш копланар и емкостную связь резонатора как трехпортовую систему к третьему порту которой подключается резонатор.
\begin{figure}[H]
    \center{\includegraphics[width=0.4\textwidth]{q2}}
     
	\end{figure}
Тогда S-матрица такой системы имеет следующий вид:
$$
\begin{aligned}
& \delta_{0}=\omega C_{c} Z_{0} \quad \delta_{r}=\omega C_{c} Z_{r} \quad \delta_{0}, \delta_{r} \ll 1 \\
& S=\left[\begin{array}{lll}
-j \delta_{0} / 2 & 1-j \delta_{0} / 2 & j \sqrt{\delta_{0} \delta_{r}} \\
1-j \delta_{0} / 2 & -j \delta_{0} / 2 & j \sqrt{\delta_{0} \delta_{r}} \\
j \sqrt{\delta_{0} \delta_{r}} & j \sqrt{\delta_{0} \delta_{r}} & 1-2 j \delta_{r}-2 \delta_{r}^{2}-\delta_{r} \delta_{0}
\end{array}\right]\end{aligned}$$

Откуда добротность:
$$
Q_{c}=2 \pi \frac{\text { energy stored in the resonator }}{\text { energy leak from port } 3 \text { to port } 1 \text { and } 2 \text { per cycle }}=\frac{\pi}{2\left|S_{31}\right|^{2}}=\frac{\pi}{2\omega^{2} C_{c}^{2} Z_{0} Z_{r}}
$$
Полученный результат не подходит для действительного расчета добротности так как не учитывает потери в диэлектрике, но позволяет понять, от каких параметров она зависит и как.
\subsection{Расчет для различных подложек}
Изначально планировалось использование такой-же подложки что и для фильтра, но на ней физически невозможно достичь желаемой добротности из-за большой величины тангенса потерь. (внутренняя добротность получалась уже порядка 30) Чтобы получить желаемую добротность был выбран специализированный СВЧ композит Rogers на основе смолы с тефлоном и керамикой, который обеспечивает меньше потери (и так-же доступен на производстве)
\begin{figure}[H]
    \center{\includegraphics[width=\textwidth]{calc}}
     
	\end{figure}
	По итогу была выбрана максимально тонкая линия на подложке rogers для достижения максимальной добротности.
\subsection{Дизайн}
	Для создания чертежа была написана программа на python с использованием api klayout которая генерирует чертеж по заданным параметрам геомтрии резонатора.
	\begin{figure}[H]
    \center{\includegraphics[width=\textwidth]{res2}}
     
	\end{figure}
\subsection{Симуляция с учетом потерь в Sonnet}
Так как не было получено точного выражения для внешней добротности, то было сгенреировано и исследовано несколько моделей с различными параметрами ($l$ - длина связи, а $d$ - расстояние от участка связывания до копланара)
	\begin{figure}[H]
    \center{\includegraphics[width=\textwidth]{ressim}}
     
	\end{figure}
	В итогу получилось достичь желаемых значений, притом основной вклад в потери вносит внутренняя добротнойсть, которая определяется свойствами подложки, поэтому получить больше на нашей подложке мы не сможем 
\end{document}
